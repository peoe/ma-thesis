%        Math imports        %

%        Symbols        %
\usepackage{mathtools}
\usepackage{amssymb}
\usepackage{marvosym} % lightning symbol
\usepackage{textcomp} % interrobang
\usepackage{aligned-overset} % alignment in overset commands

%        Theorems        %
\usepackage{amsthm}
\makeatletter
\@ifclassloaded{beamer}{}{%
    \@ifclassloaded{book}{%
        \newtheorem{theorem}{Theorem}[chapter]
    }{%
        \@ifclassloaded{article}{%
            \newtheorem{theorem}{Theorem}
        }{}
    }

    \newtheorem{lemma}[theorem]{Lemma}
    \newtheorem{corollary}[theorem]{Corollary}

    \theoremstyle{definition}
    \newtheorem{definition}[theorem]{Definition}
}

\theoremstyle{definition}
\newtheorem{assumption}[theorem]{Assumption}

\theoremstyle{remark}
\newtheorem*{remark}{Remark}
\newtheorem{example}[theorem]{Example}

\makeatother

%        Commands        %
\usepackage{interval}
\intervalconfig{soft open fences}

%        Paired Delimiters        %
\DeclarePairedDelimiterXPP{\abs}[2]{}{\lvert}{\rvert}{_{#2}}{#1}
\DeclarePairedDelimiterXPP{\norm}[2]{}{\lVert}{\rVert}{_{#2}}{#1}
\DeclarePairedDelimiterXPP{\inner}[3]{}{(}{)}{_{#3}}{#1,#2}

%        Set Commands        %
\newcommand{\sset}[1]{\left\{#1\right\}}
\newcommand{\iset}[2]{\left\{#1\mid#2\right\}}
\newcommand{\ISET}[2]{\iset{#1}{\text{#2}}}

%        Matrix Commands        %
\usepackage{xparse}
\newcommand{\trans}{^{\top}}
\newcommand{\herm}{^{\mkern1mu \mathsf{H}}}
\NewDocumentCommand{\inv}{O{1}}{^{-#1}}
\newcommand{\id}[1][]{\text{id}_{#1}}%\mkern2mu #1}}
\newcommand{\diag}[1]{\mathop{}\!\mathrm{diag}\left(#1\right)}
\newcommand{\sksym}[1]{\mathop{}\!\mathrm{skew}\left(#1\right)}
\newcommand{\psd}[1]{\mathop{}\!\mcl{P}_{\succcurlyeq 0}\left(#1\right)}
\newcommand{\kron}[1]{\mathop{}\!\mathrm{kron}\left(#1\right)}

%        Typesetting Commands        %
\newcommand{\pd}[2][x]{\partial_{#1}^{#2}}
\newcommand{\bb}[1]{\mathbb{#1}}
\newcommand{\fk}[1]{\mathfrak{#1}}
\newcommand{\mbf}[1]{\mathbf{#1}}
\newcommand{\mcl}[1]{\mathcal{#1}}
\newcommand{\msc}[1]{\textnormal{\scshape #1}}
\newcommand{\dif}{\mathop{}\!\mathrm{d}}
\newcommand{\dvg}[1]{\mathop{}\!\mathrm{div}\left(#1\right)}
\newcommand{\curl}[1]{\mathop{}\!\mathrm{curl}\left(#1\right)}
\newcommand{\lapl}[1][]{\mathop{}\!\Delta#1}
\newcommand{\sse}{\subseteq}
\newcommand{\res}[2]{{\left.\kern-\nulldelimiterspace#1\vphantom{|}\right|_{#2}}}
\newcommand{\eps}{\varepsilon}
\newcommand{\kolm}[1]{\mathop{}\!\mathrm{d}_N\left(#1\right)}
\newcommand{\tol}{\mathop{}\!\msc{tol}}
\newcommand{\lspn}[1]{\mathop{}\!\mathrm{span}\left(#1\right)}
\newcommand{\range}[1]{\mathop{}\!\mathrm{range}\left(#1\right)}
\newcommand{\knl}[1]{\mathop{}\!\mathrm{ker}\left(#1\right)}
\newcommand{\rank}[1]{\mathop{}\!\mathrm{rank}\left(#1\right)}
\newcommand{\trace}[1]{\mathop{}\!\mathrm{tr}\left(#1\right)}
\newcommand{\iu}{{i\mkern1mu}}
\newcommand{\re}[1]{\mathop{}\!\mathfrak{Re}\left(#1\right)}
\newcommand{\im}[1]{\mathop{}\!\mathfrak{Im}\left(#1\right)}
\newcommand{\conj}[1]{\overline{#1}}
\newcommand{\vecmat}{\mathop{}\!\mathrm{v}2\mathrm{m}}
\newcommand{\vecupper}{\mathop{}\!\mathrm{v}2\mathrm{u}}
\newcommand{\vecstrict}{\mathop{}\!\mathrm{v}2\mathrm{s}}
\newcommand{\matvec}{\mathop{}\!\mathrm{m}2\mathrm{v}}
\newcommand{\uppervec}{\mathop{}\!\mathrm{u}2\mathrm{v}}
\newcommand{\strictvec}{\mathop{}\!\mathrm{s}2\mathrm{v}}
\newcommand{\eqinf}{\mathop{\mathrm{inf}\vphantom{\mathrm{sup}}}}
\newcommand{\tfunc}{\zeta}

%        Operators        %
\DeclareMathOperator*{\argmax}{arg\,max}
\DeclareMathOperator*{\argmin}{arg\,min}
\DeclareMathOperator*{\suchthat}{s.\,t.}

%        Algorithms        %
\usepackage[algoruled, noline, linesnumbered, noend]{algorithm2e}
