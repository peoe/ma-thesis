\appendix

{
    \setbeamertemplate{footline}{
        \color{white}
        \small
        \begin{tabularx}{\textwidth}{XX}
            \insertshorttitle{} & \hfill\insertauthor{} \\
        \end{tabularx}
        \vspace{4pt}
    }

    \begin{frame}[standout]
        \begin{columns}
            \begin{column}{.4\textwidth}
                Recap
                {
                    \normalsize
                    \begin{itemize}
                        \item MOR with Quadratically Embedded Manifolds
                        \item Port-Hamiltonian Systems (Overview)
                        \item pH DMD
                        \item Quadratically Embedded Manifolds for pH Systems
                    \end{itemize}
                }
            \end{column}
            \uncover<2>{
                \begin{column}{.4\textwidth}
                    What's next?
                    {
                        \normalsize
                        \begin{itemize}
                            \item Examples
                            \item Other Operator Inference Frameworks
                            \item Error Analysis
                            \item Polynomially Embedded Manifolds
                            \item xcvb
                        \end{itemize}
                    }
                \end{column}
            }
        \end{columns}
    \end{frame}
}

{
    \setbeamertemplate{footline}{
        \color{petrol}
        \small
        \begin{tabularx}{\textwidth}{XX}
            \insertshorttitle{} & \hfill\insertauthor{} \\
        \end{tabularx}
        \vspace{4pt}
    }

    \begin{frame}{Structure-Preserving Interconnection of pH Systems, cf.~\cite{VanDerSchaft2014, Mehrmann2022}}
        Multiple port-Hamiltonian systems
        \begin{align*}
            H_i \dot{x}_i &= (J_i - R_i) x_i + (G_i - P_i) u_i \\
            y_i &= (G_i + P_I)\trans x_i + (S_i - N_i) u_i
        \end{align*}
        can be interconnected
        \begin{itemize}
            \item through theoretical framework of \emph{Dirac structures}, cf.~\cite[Section~6.2]{VanDerSchaft2014}, or
            \item through the matrix relation (cf.~\cite[Section~6.4]{Mehrmann2022})
                \begin{equation*}
                    \begin{bmatrix}
                        M_{11} & M_{12} \\
                        M_{21} & M_{22}
                    \end{bmatrix} \begin{bmatrix}
                        u_1 \\
                        u_2
                    \end{bmatrix} + \begin{bmatrix}
                        L_{11} & L_{12} \\
                        L_{21} & L_{22}
                    \end{bmatrix} \begin{bmatrix}
                        y_1 \\
                        y_2
                    \end{bmatrix} = \begin{bmatrix}
                        0 \\
                        0
                    \end{bmatrix}.
                \end{equation*}
        \end{itemize}
    \end{frame}

    \begin{frame}{Power Balance Equation, cf.~\cite[Section~6.1]{Mehrmann2022}}
        \begin{block}{Power Balance}
            \begin{equation*}
                \frac{\text{d}}{\text{d}t} \mcl{H}(t, x) = - \begin{bmatrix}
                    \eta(t, x) \\
                    u
                \end{bmatrix}\trans \begin{bmatrix}
                    R & P \\
                    P\trans & S
                \end{bmatrix}(t, x) \begin{bmatrix}
                    \eta(t, x) \\
                    u
                \end{bmatrix} + y\trans u
            \end{equation*}
        \end{block}

        \begin{block}{Dissipation Inequality}
            \begin{align*}
                \begin{bmatrix}
                    R & P \\
                    P\trans & S
                \end{bmatrix}(t, x) \geq 0 &\implies \frac{\text{d}}{\text{d}t} \mcl{H}(t, x) \leq y\trans u \\
                 &\implies \mcl{H}(t_2, x(t_2)) - \mcl{H}(t_1, x(t_2)) \leq \int\limits_{t_1}^{t_2} y(s)\trans u(s) ds
            \end{align*}
        \end{block}
    \end{frame}

    % TODO: MOR for pH

    % TODO: iterative solution algorithm from phDMD?

    % TODO: more references on Quad MOR, pH and DMD?
}