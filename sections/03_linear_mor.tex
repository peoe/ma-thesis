\chapter{\acl{MOR}}\label{chap:linear-mor}

This chapter introduces the concept of \acf{MOR}.
The aim of MOR is to create models of smaller dimension from \acfp{FOM} such that these \acfp{ROM} provide the best approximation of the \acp{FOM} possible.
Ultimately, the motivation behind creating low-dimensional ROMs is to speed up repeated computations in multi-query situations such as parameter optimization, model validation, or time-sensistive on-the-fly (online) computations for use within real-world applications.
As the central perspective for this thesis, we introduce two of the most prominent linear reduction frameworks for \ac{LTI} systems in Section~\ref{sec:system-mor}.
In order to effectively reduce \ac{PH} systems, we also have to take a look at preserving matrix structures during the reduction process.
We discuss two such structure-preserving algorithms in Section~\ref{sec:structure-preserving-mor}.
Afterwards, we demonstrate the drawbacks of linear model reduction by considering an abstract benchmark for aspects of the approximation quality of linear reduced models in Section~\ref{sec:limitations-linear-mor}.
Finally, we discuss strategies to mitigate the downsides of linear \ac{MOR} through nonlinear procedures in Sections~\ref{sec:nn-mor} and~\ref{sec:mor-quadratically-embedded-manifolds}.

\section{System Theoretic Linear MOR}\label{sec:system-mor}

\acl{MOR} for Systems Theory offers two main perspectives: reduction resulting from the system matrices of \ac{LTI} systems as in Definition~\ref{def:lti}, and reduction by means of the associated transfer function from Definition~\ref{def:transfer-function}.
We give an idea of truncation in Subsection~\ref{subsec:balanced-truncation} by explaining the method of \acf{BT}, a reduction method focussed on the system's matrices.
Afterwards, we discuss reduction by means of interpolating the transfer function, which leads us to the \acf{IRKA} in Subsection~\ref{subsec:interpolation-reduction}.

\subsection{Balanced Truncation}\label{subsec:balanced-truncation}

\ac{BT} is a two step procedure consisting of balancing and truncation.
Firstly, balancing an \ac{LTI} system in a heuristic way means transforming the system such that hard-to-reach states are simultaneously hard to observe.
Secondly, truncation of the balanced system constitutes itself by omission of the matrix blocks that correspond to hard-to-reach and hard-to-observe system states.
In this manner, we create a system that prioritizes easily controllable and easily observable states.

This procedure was originally described in~\cite{Mullis1976} and later extended by~\cite{Moore1981, Enns1984}.
Within the \ac{BT} methodology, several balancing options are available to cater to different system requirements.
Standard Lyapunov balancing often requires large computational times as~\cite{Gugercin2007} points out, because during this procedure one has to compute expensive dense matrix factorizations.
To mitigate this drawback, approximate balanced truncation methods have been developed.
These include, among others, stochastic balancing~\cite{Desai1984, Green1988}, frequency weighted balancing~\cite{Enns1984, Wang1999}, bounded real balancing~\cite{Opdenacker1988, Reis2010}, or positive real balancing~\cite{Desai1984, Ober1991}.
For our purposes, it suffices to consider Lyapunov \ac{BT} and \ac{PRBT} since we are not primarily interested in the theory of balancing \ac{MOR} methods.
\ac{PRBT} in particular plays an important role in the context of this thesis because it produces passive systems that are closely linked to \ac{PH} systems but we cover this connection later on in Subsection~\ref{subsec:prbt}.

We commence with standard Lyapunov \ac{BT} as described in~\cite{BB2017}.
Consider $(A, B, C, D, E)$ to be a realization of an \ac{LTI} system as in Definition~\ref{def:lti}.
It is important to note that some \ac{BT} literature employs \ac{LTI} systems that have the trivial matrix $E = \id$.
If $E$ is regular, then any system $(A, B, C, D, E)$ can easily be transformed into a system of the form $(\tilde{A}, \tilde{B}, \tilde{C}, \tilde{D}, \id)$ by applying $E\inv$ to the state equation in~\eqref{eq:lti}.
Henceforth, we refer to our \ac{LTI} system $\Sigma_\msc{lti}$ by the matrices $(A, B, C, D, \id)$.

\begin{definition}[{Adapted from~\cite[Definition~6.2]{BB2017}}]
    The system $(A, B, C, D, \id)$ is controllable if for all states $x_0, x_1 \in \bb{R}^n$ and a final time $T \geq 0$ there exists an admissible control $u \colon \interval{0}{T} \to \bb{R}^q$ such that the system's solution $x(t)$ satisfies $x(0) = x_0$ and $x(T) = x_1$.
\end{definition}

\begin{definition}[{Adapted from~\cite[Definition~6.3]{BB2017}}]
    The state $x_1 \in \bb{R}^n$ of a system $(A, B, C, D, \id)$ is unobservable if the output of the system with the initial condition $x(0) = x_1$ is constant $y(t) \equiv 0$ for all points in time.
    In particular, a state $x_1$ being unobservable means that $x_1$ cannot be distinguished from the zero state no matter the control applied to the system.
    A system is observable if the set of all unobservable states consists of only the trivial zero state.
\end{definition}

To characterize which system states are observable and which system states are controllable, we define the system's infinite Controllability Gramian as given in~\cite[Equation~6.8]{BB2017}
\begin{equation*}
    P = \int\limits_0^\infty \exp{(A t)} B B\trans \exp \left( A\trans t \right) \dif t \in \bb{R}^{n \times n}
\end{equation*}
alongside the corresponding infinite Observability Gramian
\begin{equation*}
    Q = \int\limits_0^\infty \exp \left( A\trans t \right) C\trans C \exp{(A t)} \dif t \in \bb{R}^{n \times n}.
\end{equation*}
For asymptotically stable and minimal systems as defined in Definitions~\ref{def:minimal-system} and~\ref{def:lti-stability}, these matrices satisfy the Lyapunov equalities
\begin{equation}\label{eq:lyapunov-equations}
    \begin{aligned}
        A P + P A\trans + B B\trans &= 0, \\
        A\trans Q + Q A + C\trans C &= 0
    \end{aligned}
\end{equation}
as has been shown in~\cite{Antoulas2005, Hinrichsen2005}.

The act of balancing now reduces to finding a transformation matrix $T \in \bb{R}^{n \times n}$ such that the two matrices $T P T\trans$ and $T\inv[T] Q T\inv$ are diagonal matrices.
One such balancing transformation can be constructed by computing the Cholesky decompositions
\begin{equation}\label{eq:balancing}
    P = S\trans S,\quad Q = R\trans R,\quad S R\trans = U \Sigma V\trans
\end{equation}
and then defining the transformation matrix as $T \coloneqq \Sigma\inv[\frac{1}{2}] V\trans R$.
With this $T$ we can transform the original \ac{LTI} system $(A, B, C, D, \id)$ into its balanced form $(T A T\inv, T B, C T\inv, D, \id)$.
To finally truncate the system, that is, to obtain a reduced order system, we split the balanced system into matrix blocks
\begin{equation*}
    T A T\inv = \begin{pmatrix}
        A_1 & A_2 \\
        A_3 & A_4
    \end{pmatrix},\quad T B = \begin{pmatrix}
        B_1 \\
        B_2
    \end{pmatrix},\quad C T\inv = \begin{pmatrix}
        C_1 & C_2
    \end{pmatrix}.
\end{equation*}
Thereafter, we define the balanced \ac{ROM} as $(A_1, B_1, C_1, D, \id)$.
This reduction even yields the following estimate of the $\mcl{H}_\infty$ projection error
\begin{equation*}
	\norm{\tfunc - \hat{\tfunc}}{\mcl{H}_\infty} \coloneqq \sup\limits_{\omega \in \bb{R}} \sigma_{\max} \left( \tfunc(\iu \omega) - \hat{\tfunc}(\iu \omega) \right).
\end{equation*}

\begin{theorem}[{Adapted from~\cite[Theorem~6.4]{BB2017}}]
    Let $(A_\msc{b}, B_\msc{b}, C_\msc{b}, D_\msc{b}, \id)$ be a balanced realization of an asymptotically stable and minimal system as defined in Definitions~\ref{def:minimal-system} and~\ref{def:lti-stability} with an ordered sequence of real numbers $\sigma_1 > \cdots > \sigma_k > 0$ such that the Gramians read $P = Q = \diag{\sigma_1 \id, \dots, \sigma_k \id}$.
    Then the reduced system $\hat{\Sigma}_\msc{lti}$ obtained by truncating $\Sigma_\msc{lti}$ is asymptotically stable, minimal, balanced, and fulfills
    \begin{equation*}
        \norm{\tfunc - \hat{\tfunc}}{\mcl{H}_\infty} \leq 2 \sum\limits_{i = r + 1}^k \sigma_i
    \end{equation*}
    for the transfer functions $\tfunc$ and $\hat{\tfunc}$ corresponding to $\Sigma_\msc{lti}$ and $\hat{\Sigma}_\msc{lti}$.
\end{theorem}

\subsection{Interpolation-Based Reduction}\label{subsec:interpolation-reduction}

Whereas balanced \acp{ROM} try to minimize the $\mcl{H}_\infty$-norm, interpolation-based reduced models aim to find bestapproximations with respect to the $\mcl{H}_2$-norm as defined in~\cite[Section~3]{Gugercin2008} by
\begin{equation*}
	\norm{\tfunc}{\mcl{H}_2}^2 \coloneqq \frac{1}{2 \pi} \int\limits_{- \infty}^\infty \norm{\tfunc(\iu \omega)}{}^2 \dif \omega.
\end{equation*}
The problem with this type of optimization problem, as indicated in~\cite[Section~3.1]{Gugercin2008}, is that the set of stable \acp{LTI} does not form a subspace of $\mcl{H}_2$, hence the $\mcl{H}_2$ optimal approximation is not easy to characterize.
As a replacement for this optimization set, we instead consider the set of all proper rational transfer functions as defined in~\cite[Section~1.3.1]{Corless2003} with the simple poles $\mu \coloneqq {\sset{\mu_i}}_{i = 1}^r \subseteq \bb{C}^r$ in the open left half plane $\bb{C}_{-}$
\begin{equation*}
    \mcl{M}(\mu) \coloneqq \iset{\tfunc \text{ proper rational transfer function}}{\tfunc \text{ has simple poles at } \mu_i, i = 1, \dots, r}.
\end{equation*}
Following~\cite[Theorem~3.1]{Gugercin2008}, this results in an subset of $\mcl{H}_2$ with the properties that
\begin{enumerate}
    \item $H \in \mcl{M}(\mu)$ is the transfer function of a stable \ac{LTI} system with $\dim{(H)} = r$,
    \item $\mcl{M}(\mu)$ is an $(r - 1)$-dimensional subspace of $\mcl{H}_2$,
    \item $\tfunc_r \in \mcl{M}(\mu)$ is the unique bestapproximation of $\tfunc \in \mcl{H}_2$ if and only if it holds that $\inner{\tfunc - \tfunc_r}{H}{\mcl{H}_2} = 0$ for all other transfer functions $H \in \mcl{M}(\mu)$.
\end{enumerate}
As a consequence of~\cite[Theorem~3.1]{Gugercin2008}, if we interpolate the \ac{FOM} transfer function $\tfunc$ at the mirror images $\hat{\sigma} = -a + \iu b$ of its poles $\sigma = a + \iu b$, the subspace transfer function $\tfunc_\msc{r} \in \mcl{M}(\mu)$ is $\mcl{H}_2$-optimal among all \acp{ROM} with the same reduced poles $\mu$.

For any local minimizer we can formulate the following necessary optimality conditions of the $\mcl{H}_2$ bestapproximation problem with the interpolation points $\sigma_i \in \bb{C}$ and the corresponding tangential directions $b_i, c_i \in \bb{C}^{m}$ for an \ac{LTI} system with $m \in \bb{N}$ inputs and outputs
\begin{equation*}
    \tfunc_\msc{r}(- \sigma_i) b_i = \tfunc(- \sigma_i) b_i,\quad c_i\trans \tfunc_\msc{r}(- \sigma_i) = c_i\trans \tfunc(- \sigma_i),\quad c_i\trans \tfunc'_\msc{r}(- \sigma_i) b_i = c_i\trans \tfunc(- \sigma_i) b_i.
\end{equation*}

While these conditions are very useful, solving the optimal approximation problem remains complicated.
To combat this difficulty, the authors of~\cite{Gugercin2008} propose an algorithm derived from the standard Newton update formula.
Let $\sigma \coloneqq {\sset{\sigma_i}}_{i = 1}^r \subset \bb{C}$ denote the set of interpolation points of the reduced model $\hat{\Sigma}_\msc{lti}$.
From this model we compute the transfer function $\tfunc_r$ and its poles $\lambda(\sigma) = {\sset{\lambda_i}}_{i = 1}^r$.
We now want to minimize the difference between the mirror images of the interpolation points and the poles $\lambda(\sigma)$.
To this end, we define the objective function $g(\sigma) \coloneqq \lambda(\sigma) + \sigma$, aiming to ultimately obtain an optimal set of interpolation points $\sigma_i$ such that $g\left( {\sset{\sigma_i}}_{i = 1}^r \right) = 0$.
This optimality condition directly corresponds to $\lambda(\sigma) = -\sigma$, thus resulting in the interpolation points matching the mirror images of the poles if we consider complex conjugate pairs of interpolation points.
The Newton step for this optimization in its full form is
\begin{equation}\label{eq:interpolation-points-newton-step}
    \sigma^{(k + 1)} = \sigma^{(k)} - (\id + J)\inv (\sigma^{(k)} + \lambda(\sigma^{(k)})),
\end{equation}
where $J$ is the Jacobian matrix in the sense that $J_{i, j} = \pd[\sigma_j]{} \lambda_i$.
In practice, the explicit compution of this Jacobian quickly becomes prohibitively expensive.
Therefore, we make the suitable substitution $J \approx 0$ because the Jacobian will be close to zero when the set of interpolation points is close to an optimal set.
Applying this assumption in~\eqref{eq:interpolation-points-newton-step}, the Newton update reduces to
\begin{equation*}
    \sigma^{(k + 1)} = - \lambda(\sigma^{(k)}).
\end{equation*}
Algorithm~\ref{alg:irka} shows how this Newton step is used in the \acf{IRKA} as described in~\cite[Algorithm~4.1]{Gugercin2008}.
This algorithm heavily relies on a realization-based formulation; however, there are some variants independent of specific realizations using the Loewner framework such as~\cite[Algorithm~7.2]{Beattie2017}.
It is further notable, that, due to the complex conjugates in the interpolation points, the updated subspaces $V_\msc{r}$ and $W_\msc{r}$ can be chosen as real-valued as explained in~\cite[Remark~4]{Gugercin2012}.

\begin{algorithm}
    \caption{\ac{IRKA}, adapted from~\cite[Algorithm~7.1]{Beattie2017}}\label{alg:irka}
    \KwData{Initial interpolation points ${\sset{\sigma_i^{(0)}}}_{i = 1}^r$ and initial tangent directions ${\sset{r_i^{(0)}}}_{i = 1}^r, {\sset{\ell_i^{(0)}}}_{i = 1}^r$ closed under complex conjugation, full order model $\Sigma$}
    \tcc{Initialize reduced subspaces}
    $V_r^{(0)} \coloneqq {\left( {(\sigma_i^{(0)} E - A)}\inv B r_i^{(0)} \right)}_{i = 1}^r,\quad W_r^{(0)} \coloneqq {\left( {(\sigma_i^{(0)} E - A\trans)}\inv C\trans \ell_i^{(0)} \right)}_{i = 1}^r$\;
    $k \coloneqq 0$\;
    \While{not converged}{
        $k \coloneqq k + 1,\quad \tilde{A} \coloneqq W\trans A V,\quad \tilde{E} \coloneqq W\trans E V,\quad \tilde{B} \coloneqq W\trans B,\quad \tilde{C} = C V$\;
        \tcc{Compute pole-residue expansion}
        $\tfunc_r(s) = \tilde{C} {(s \tilde{E} - \tilde{A})}\inv \tilde{B} = \sum\limits_{i = 1}^r \frac{\tilde{\ell}_i \tilde{r}_i\trans}{s - {\lambda(\sigma^{(k)})}_i}$\;
        \tcc{Update interpolation points and tangential directions}
        $\sigma_i^{(k)} \coloneqq - \lambda_i(\sigma^{(k - 1)}),\quad r_i^{(k)} \coloneqq \tilde{r}_i,\quad \ell_i^{(k)} \coloneqq \tilde{\ell}_i$\;
        \tcc{Update reduced subspaces}
        $V_\msc{r}^{(k)} \coloneqq {\left( {(\sigma_i^{(k)} E - A)}\inv B r_i^{(k)}\right)}_{i = 1}^r,\quad W_\msc{r}^{(k)} \coloneqq {\left( {(\sigma_i^{(k)} E - A\trans)}\inv C\trans \ell_i^{(k)} \right)}_{i = 1}^r$\;
    }
\end{algorithm}

\section{Structure-Preserving Linear MOR}\label{sec:structure-preserving-mor}

In Section~\ref{sec:system-mor}, we covered some linear \ac{MOR} frameworks for \ac{LTI} systems.
Now, we want to extend these linear methods to \acp{PH} systems by introducing structure preservation into the \ac{BT} and \ac{IRKA} methods.
This is by no means easy because the restriction from the set of unstructured matrices to more restrictive sets like the family of skew-symmetric or symmetric positive definite matrices causes even more complexity in the optimization step.
However, these algorithms are the foundation for the linear \acp{ROM} we need when constructing more involved models.
To start off this section, we expand the standard \ac{BT} formulation to positive real \ac{LTI} systems and demonstrate how these are related to \ac{PH} systems in Subsection~\ref{subsec:prbt}.
Afterwards, we quickly illustrate \ac{PHIRKA}, an extension of the usual \ac{IRKA} algorithm, in Subsection~\ref{subsec:ph-irka}.
Lastly, we mention \ac{DEIM} and effort to apply it to \ac{PH} systems in Subsection~\ref{subsec:ph-deim}.

\subsection{Positive Real Balanced Truncation}\label{subsec:prbt}

The first algorithm we take a look at is \acf{PRBT}.
This method relies on the fact that positive real systems are closely related to \ac{PH} systems by the fact that we can transform behaviourally controllable and behaviourally observable positive real systems into \ac{PH} systems as we will see at the end of this subsection in Equation~\eqref{eq:prbt-ph-system}.
First of all, we require three additional definitions which relate to observability and controllability of the \ac{LTI} system, as well as the positive realness property.

\begin{definition}[{Adapted from~\cite[Theorem~6.2]{Freund2004} and~\cite[Equation~(8)]{CGH2022}}]\label{def:behaviourally-controllable-observable}
    An \ac{LTI} system with the realization $(A, B, C, D, E)$ is behaviourally controllable if for all $\lambda \in \bb{C}$ the following condition holds
    \begin{equation}\label{eq:behaviourally-controllable}
        \rank{\begin{matrix}
            \lambda E - A & B
        \end{matrix}} = n.
    \end{equation}
    Analogously, the system further is behaviourally observable if for all $\lambda \in \bb{C}$
    \begin{equation}\label{eq:behaviourally-observable}
        \rank{\begin{matrix}
            \lambda {(E - A)}\trans & C\trans
        \end{matrix}} = n
    \end{equation}
    is satisfied.
\end{definition}

\begin{definition}[{Adapted from~\cite[Definition~2.1]{Freund2004} and~\cite[Property~(PR)]{CGH2022}}]\label{def:positive-real}
    An \ac{LTI} system $\Sigma_\msc{lti}$ is said to be positive real if its transfer function $\tfunc$ has no poles in the right complex halfplane $\bb{C}_+$, for all complex numbers $\sigma \in \bb{C}$ it holds that $\conj{\tfunc(\sigma)} = \tfunc(\conj{\sigma})$, and the the matrix $\tfunc(\sigma) + \tfunc(\sigma)\herm$ is positive semidefinite for all complex numbers $\sigma \in \bb{C}, \re{\sigma} > 0$.
    Analogously, the transfer function is called strictly positive real if, in addition to the other two conditions, the matrix sum is positive definite.
\end{definition}

To extend the standard balancing procedure to result in positive real systems we relate the passivity of an \ac{LTI} system and its accompanying realization $(A, B, C, D, E)$ with its positive real transfer function $\tfunc$ through the result of~\cite[Corollary~2.7]{CGH2022}.
In addition, we can also formulate the condition $D + D\trans \succcurlyeq 0$ on the system's feedthrough term following~\cite[Definition~5]{Gugercin2007}.
Similarly to the Lyapunov equations~\eqref{eq:lyapunov-equations} it can be shown that an \ac{LTI} $(A, B, C, D, \id)$ is strictly positive real if and only if there exist symmetric positive definite matrices $K, L \in \bb{R}^{n \times n}$ such that the following \acp{ARE} are satisfied
\begin{equation}\label{eq:riccati-equations}
    \begin{aligned}
        A\trans K + K A + (K B - C\trans) (D + D\trans)\inv (K B - C\trans)\trans &= 0, \\
        A L + L A\trans + (L C\trans - B) (D + D\trans)\inv (L C\trans - B)\trans &= 0.
    \end{aligned}
\end{equation}
For positive real systems, all solutions $K, L$ can be bounded by minimal and maximal solutions
\begin{equation*}
    K_{\max} \succcurlyeq K \succcurlyeq K_{\min} \succcurlyeq 0,\quad L_{\max} \succcurlyeq L \succcurlyeq L_{\min} \succcurlyeq 0,
\end{equation*}
as shown in~\cite[Proposition~5.1]{Ober1991} by the connection to bounded real systems.
The solutions $K, L$ of the \acp{ARE}~\eqref{eq:riccati-equations} are related by $K = L\inv$, thus further implying that
\begin{equation*}
    K_{\min} = L_{\max}\inv,\quad K_{\max} = L_{\min}\inv.
\end{equation*}
To balance the strictly positive real system, we then perform the same Cholesky decomposition and following calculations as in~\eqref{eq:balancing} on the minimal solutions $K_{\min}$ and $L_{\min}$.
Such a system is then called strictly positive real balanced if, after the transformation has been applied, $K_{\min} = L_{\min}$ can be written as the diagonal matrix $\diag{\pi_1 \id[s_1], \dots, \pi_q \id[s_q]}$ with the $\pi_i$ ordered such that $0 < \pi_q < \cdots < \pi_1 \leq 1$, and $s_i$ the corresponding multiplicities that sum up to $\sum s_i = n$.

For the rest of this subsection, we focus on considering the necessary conditions such that a positive real \ac{LTI} admits a \ac{PH} form.

\begin{lemma}[{Adapted from~\cite{Verghese1981, Dai1989}}]\label{lem:minimality-conditions}
    An \ac{LTI} system with a realization $(A, B, C, D, E)$ is minimal in the sense of Definition~\ref{def:minimal-system} if and only if it fulfills the following two conditions
    \begin{equation*}
        \rank{\begin{matrix}
            E & B
        \end{matrix}} = \rank{\begin{matrix}
            E \\
            C
        \end{matrix}} = n,\quad A \knl{E} \subseteq \range{E}
    \end{equation*}
    in addition to Equations~\eqref{eq:behaviourally-controllable} and~\eqref{eq:behaviourally-observable}.
\end{lemma}

For the next step, we need to compute a specific form of the transfer function's Laurent series.
In the same manner as in~\cite[Section~5]{CGH2022} we construct the Laurent expansion
\begin{equation}\label{eq:laurent-series}
    \tfunc(s) = \sum\limits_{i = -\infty}^{k - 1} M_i s^i.
\end{equation}
The $M_i$ within the Laurent series are the moments of the \ac{LTI} system.
These moments allow us to give the following Lemma which connects positive real systems' transfer functions to the terms of the Laurent expansion.

\begin{lemma}[{Adapted from~\cite[Lemma~5.1]{CGH2022}}]\label{lem:lti-laurent-series}
    Let $(A, B, C, D, E)$ be a realization of a positive real \ac{LTI} system $\Sigma_\msc{lti}$.
    We denote the transfer function of $\Sigma_\msc{lti}$ by $\tfunc$, and compute the underlying Laurent series as defined in~\eqref{eq:laurent-series}.
    Then we can write the transfer function as the sum
    \begin{equation}\label{eq:pr-lti-laurent-series}
        \tfunc(s) = \tfunc_\msc{p}(s) + M_1 s,
    \end{equation}
    where $\tfunc_{\msc{p}}$ is a proper rational function fulfilling $\lim\limits_{s \to \infty} \tfunc_\msc{p}(s) = M_0$, and $M_0$ and $M_1$ are the corresponding moment matrices from~\eqref{eq:laurent-series}.
\end{lemma}

We now construct a minimal realization $(A_\msc{p}, B_\msc{p}, C_\msc{p}, D_\msc{p}, E_\msc{p})$ with $D_\msc{p} = M_0$ from Lemma~\ref{lem:lti-laurent-series}.
This realization contains an invertible matrix $E_\msc{p}$ because, by Lemma~\ref{lem:minimality-conditions}, the realization satisfies the assumptions of~\cite[Theorem~6.3]{Freund2004}, and by extension from the positive realness of the complete system, $(A_\msc{p}, B_\msc{p}, C_\msc{p}, D_\msc{p}, E_\msc{p})$ is also positive real.
We can apply~\cite[Proposition~5.4]{CGH2022} because we have defined the proper realization in such a way that $D_\msc{p} + D_\msc{p}\trans \succcurlyeq M_0 + M_0\trans$ is always true.
Therefore there exists a matrix $Q_\msc{p} \in \bb{C}^{n \times n}$ that solves the following \acp{KYP}
\begin{equation}\label{eq:kyp-lmi}
    \begin{pmatrix}
        -A_\msc{p}\trans Q - Q\trans A_\msc{p} & C_\msc{p}\trans - Q\trans B_\msc{p} \\
        C_\msc{p} - B_\msc{p}\trans Q & D_\msc{p} + D_\msc{p}\trans
    \end{pmatrix} \succcurlyeq 0,\quad E_\msc{p}\trans Q = Q\trans E_\msc{p} \succcurlyeq 0
\end{equation}
for the proper system $(A_\msc{p}, B_\msc{p}, C_\msc{p}, M_0, E_\msc{p})$.

\begin{lemma}[{Adapted from~\cite[Proposition~3.1]{CGH2022}}]\label{lem:kyp-invertible-solution}
    If the \ac{LTI} system $\Sigma_\msc{lti}$ in the form~\eqref{eq:lti} has an invertible matrix $E$, is behaviourally observable and there exists a matrix $Q \in \bb{C}^{n \times n}$ solving the \acp{KYP}
    \begin{equation*}
        \begin{pmatrix}
            -A\trans Q - Q\trans A & C\trans - Q\trans B \\
            C - B\trans Q & D + D\trans
        \end{pmatrix} \succcurlyeq 0,\quad E\trans Q = Q\trans E,
    \end{equation*}
    then $Q$ is invertible.
\end{lemma}

Thus, by means of Lemma~\ref{lem:kyp-invertible-solution}, the matrix $Q_\msc{p}$ that solves~\eqref{eq:kyp-lmi} is invertible, and by solving~\eqref{eq:kyp-lmi} it also satisfies $Q_\msc{p}\trans E_\msc{p} \succcurlyeq 0$.
This allows us to obtain the proper part of the final \ac{PH} system
\begin{equation}\label{eq:pr-proper-ph-part}
    J_\msc{p} - R_\msc{p} \coloneqq A_\msc{p} Q_\msc{p}\inv,\quad G_\msc{p} - P_\msc{p} \coloneqq B_\msc{p},\quad {(G_\msc{p} + P_\msc{p})}\trans \coloneqq C_\msc{p} Q_\msc{p}\inv,\quad D_\msc{p} \coloneqq D_\msc{p}.
\end{equation}
Additionally, we can form a minimal realization conforming to Definition~\ref{def:minimal-system} by creating the block matrices
\begin{equation*}
    E_\infty = \begin{pmatrix}
        M_1 & 0 \\
        0 & 0
    \end{pmatrix},\quad A_\infty = \begin{pmatrix}
        0 & -\id[m] \\
        \id[m] & 0
    \end{pmatrix},\quad B_\infty = \begin{pmatrix}
        0 \\
        \id[m]
    \end{pmatrix} = C_\infty\trans,\quad D_\infty = 0.
\end{equation*}
These matrices indeed define a realization when we compute
\begin{equation*}
    s M_1 = C_\infty {(s E_\infty - A_\infty)}\inv B_\infty = C_\infty \begin{pmatrix}
        0 & \id[m] \\
        -\id[m] & s M_1
    \end{pmatrix} B_\infty.
\end{equation*}
We apply the same arguments as we have used with the proper rational part of the sum~\eqref{eq:pr-lti-laurent-series} to see that these matrices together with $Q_\infty = \id[2m]$ form a \ac{PH} realization
\begin{equation}\label{eq:pr-improper-ph-part}
    J_\infty - R_\infty \coloneqq A_\infty,\quad G_\infty - P_\infty \coloneqq B_\infty,\quad {(G_\infty + P_\infty)}\trans \coloneqq C_\infty,\quad D_\infty \coloneqq D_\infty.
\end{equation}
Lastly, we combine the both \ac{PH} systems from~\eqref{eq:pr-proper-ph-part} and~\eqref{eq:pr-improper-ph-part} by using the canonical interconnection
\begin{equation}\label{eq:prbt-ph-system}
    \begin{aligned}
        E_\msc{ph} \coloneqq \begin{pmatrix}
            E_\msc{p} & 0 \\
            0 & E_\infty
        \end{pmatrix},\quad A_\msc{ph} \coloneqq \begin{pmatrix}
            A_\msc{p} & 0 \\
            0 & A_\infty
        \end{pmatrix},\quad B_\msc{ph} \coloneqq \begin{pmatrix}
            B_\msc{p} \\
            B_\infty
        \end{pmatrix}, \\
        C_\msc{ph} \coloneqq \begin{pmatrix}
            C_\msc{p} & C_\infty
        \end{pmatrix},\quad D_\msc{ph} \coloneqq M_0,\quad Q_\msc{ph} \coloneqq \begin{pmatrix}
            Q_\msc{p} & 0 \\
            0 & Q_\infty
        \end{pmatrix}
    \end{aligned}
\end{equation}
from~\cite[Lemma~5.6]{CGH2022}.

\subsection{\acl{PHIRKA}}\label{subsec:ph-irka}

In Subsection~\ref{subsec:interpolation-reduction}, we introduced the \ac{IRKA} algorithm.
In~\cite{Gugercin2012}, the authors propose \acf{PHIRKA}, an extended variant of \ac{IRKA} that produces reduced \ac{PH} realizations of the transfer function $\tfunc$.
A representation of \ac{PHIRKA} is shown in Algorithm~\ref{alg:ph-irka}.
When considering Algorithm~\ref{alg:irka}, it is not inherently clear that the constructed subspaces $V_r$ and $W_r$ create a passive system.
However choosing $W_r = Q V_r {(V_r\trans Q V_r)}\inv$ generates a subspace which produces a passive system, and thus realizing a \ac{PH} system as outlined in~\cite[Section~2.4]{Breiten2022}.

In order to understand why the specific choice of $V_r^{(k)}$ and $W_r^{(k)}$ creates a \ac{PH} model, we need the following statements.

\begin{lemma}[{Adapted from~\cite[Theorem~7]{Gugercin2012}}]\label{lem:ph-irka-subspace}
    Let $\Sigma_\msc{ph}$ be an \ac{LTI} \ac{PH} system with the realization $(E, Q, J, R, G, P, N, S)$, and consider the following sets of sets of interpolation points and tangential directions closed under complex conjugation ${\sset{\sigma_i}}_{i = 1}^r \subseteq \bb{C}, {\sset{r_i}}_{i = 1}^r \subseteq \bb{C}^m$.
    If we construct the reduced subspaces $V_r \coloneqq {\left( {(\sigma_i E - (J - R) Q)}\inv (G - P) r_i \right)}_{i = 1}^r$ and $W_r = Q V_r {(V_r\trans Q V_r)}\inv$, and define the reduced system matrices $E_r = W_r\trans E W_r, Q_r = V_r\trans Q V_r, J_r = W_r\trans J W_r, R_r = W_r\trans R W_r, G_r = W_r\trans G, P_r = W_r\trans P, N_r = N, S_r = S$, then $(E_r, Q_r, J_r, R_r, G_r, P_r, N_r, S_r)$ is the realization of a passive \ac{PH} system.
    Additionally, the reduced transfer function $\tfunc_r$ interpolates the \ac{FOM} transfer function $\tfunc$ at all interpolation points $\sigma_i$ and all tangential directions $r_i$.
\end{lemma}

\begin{theorem}[{Adapted from~\cite[Theorem~11 and Remark~13]{Gugercin2012}}]\label{thm:ph-irka-subspace}
    Let $\Sigma_\msc{ph}$ be an asymptotically stable \ac{PH} system with the transfer function $\tfunc$.
    If \ac{PHIRKA} converges to a reduced system $\hat{\Sigma}_\msc{ph}$ with the transfer function $\tfunc_r$, and $\tfunc_r$ admits the decomposition
    \begin{equation*}
        \tfunc_r(s) \coloneqq \sum\limits_{i = 1}^r \frac{l_i r_i\trans}{s - \lambda_i}
    \end{equation*}
    with $r$ distinct poles ${\sset{\lambda_i}}_{i = 1}^r$, then $\tfunc_r$ represents an asymptotically stable and passive \ac{PH} system and it interpolates the full order transfer function $\tfunc$ at all interpolation points $\sigma_i$ and tangential directions $r_i$.

    If in addition to the subspace construction in Lemma~\ref{lem:ph-irka-subspace} the ranges of the subspaces fulfill
    \begin{equation*}
        \range{{\left( {(\lambda_i E + (J - R) Q)}\inv (G - P) r_i \right)}_{i = 1}^r} = \range{{\left( {(\lambda_i E + {(J - R)}\trans Q)}\inv (G + P) l_i \right)}_{i = 1}^r},
    \end{equation*}
    then $\hat{\Sigma}_\msc{ph}$ fulfills all necessary conditions for the $\mcl{H}_2$ optimal approximation of $\Sigma_\msc{ph}$.
\end{theorem}

\begin{remark}
    While we have so far only presented \ac{PRBT} and \ac{PHIRKA} in this section, other \ac{PH} reduction methods are available.
    Among them are moment matching as detailed in~\cite{Polyuga2010}, or algorithms based on \ac{OI} such as~\cite{BGD2020} and~\cite{Lee2022}.
    We discuss some more optimization-based \ac{OI} procedures in Chapter~\ref{chap:inferring-models}, however the focus for these will be on how we can adapt them to allow the application of quadratically embedded manifolds.
\end{remark}

\begin{algorithm}\label{alg:ph-irka}
    \caption{\ac{PHIRKA}, adapted from~\cite[Algorithm~1]{Gugercin2012}}
    \KwData{Initial interpolation points $\sigma$ closed and initial tangent directions $r_1^{(0)}, \dots, r_r^{(0)}$ under complex conjugation, full order model $\Sigma$}
    \tcc{Initialize reduced subspaces}
    $V_r^{(0)} \coloneqq {\left( {(\sigma_i^{(0)} E - (J - R) Q)}\inv (G - P) r_i^{(0)} \right)}_{i = 1}^r,\quad W_r^{(0)} \coloneqq Q V_r^{(0)} \left( {V_r^{(0)}}\trans Q V_r^{(0)} \right)\inv$\;
    $k \coloneqq 0$\;
    \While{not converged}{
        $k \coloneqq k + 1$\;
        \tcc{Update system matrices}
        $J_r^{(k)} \coloneqq {W_r^{(k - 1)}}\trans J W_r^{(k - 1)},\quad R_r^{(k)} \coloneqq {W_r^{(k - 1)}}\trans R W_r^{(k - 1)},\quad Q_r^{(k)} \coloneqq {V_r^{(k - 1)}}\trans Q V_r^{(k - 1)}$\;
        $G_r^{(k)} \coloneqq {W_r^{(k - 1)}}\trans G,\quad P_r^{(k)} \coloneqq {W_r^{(k - 1)}}\trans P,\quad A_r^{(k)} \coloneqq (J_r^{(k)} - R_r^{(k)}) Q_r^{(k)}$\;
        \tcc{Compute left and right eigenpairs}
        Find $(\lambda_i, x_i)$ and $(\lambda_i, y_i)$ satisfying $A_r^{(k)} x_i = \lambda_i x_i,\quad y_i\herm A_r^{(k)} = \lambda_i y_i\herm,\quad y_i\herm x_i = \delta_{i, j}$\;
        \tcc{Update interpolation points and tangential directions}
        $\sigma_i^{(k)} \coloneqq -\lambda_i,\quad {r_i^{(k)}}\trans \coloneqq y_i\herm B_r^{(k)}$\;
        \tcc{Update reduced subspaces}
        $V_r^{(k)} \coloneqq {\left( {(\sigma_i^{(k)} E - (J - R) Q)}\inv (G - P) r_i^{(k)} \right)}_{i = 1}^r,\quad W_r^{(k)} \coloneqq Q V_r^{(k)} {\left( {V_r^{(k)}}\trans Q V_r^{(k)} \right)}\inv$\;
    }
    \tcc{Compute final realization}
    $J_r^{(k)} \coloneqq {W_r^{(k - 1)}}\trans J W_r^{(k - 1)},\quad R_r^{(k)} \coloneqq {W_r^{(k - 1)}}\trans R W_r^{(k - 1)},\quad Q_r^{(k)} \coloneqq {V_r^{(k - 1)}}\trans Q V_r^{(k - 1)}$\;
    $G_r^{(k)} \coloneqq {W_r^{(k - 1)}}\trans G,\quad P_r^{(k)} \coloneqq {W_r^{(k - 1)}}\trans P$\;
\end{algorithm}

\subsection{\texorpdfstring{\ac{PH}}{PH} Discrete Empirical Interpolation Method}\label{subsec:ph-deim}

Besides the \ac{LTI}-based \ac{MOR} algorithms we can also consider systems with state-dependent couplings akin to the following system under the usual conditions~\eqref{eq:ph-matrix-structure}
\begin{equation}\label{eq:nonquadratic-ph-system}
    \begin{aligned}
        \dot{x} &= (J - R) Q(x) + (G - P) u, \\
        y &= {(G + P)}\trans Q(x) + (S - N) u.
    \end{aligned}
\end{equation}
In essence this can reflect e.g.\ non-quadratic Hamiltonians such as $\mcl{H}(x) = x^3$ in the scalar case, or even more general nonlinear storage functions.
To reduce a system such as~\eqref{eq:nonquadratic-ph-system}, we apply the \acf{DEIM}.

In~\cite{Chaturantabut2010}, the nonlinear hyper-reduction technique \ac{DEIM} was introduced.
For generic nonlinear functions $f \colon \bb{R}^n \to \bb{R}^n$, \ac{DEIM} computes a projection matrix $\bb{P} \in \bb{R}^{n \times n}$ such that for all $x \in \bb{R}^n$ we can define $\hat{f}(x) = \bb{P} f(x)$ with $f(x), \hat{f}(x) \in \bb{R}^n$.
The projection matrix $\bb{P}$ is computed explicitly from $m \in \bb{N}$ linearly independent vectors ${\sset{u_i}}_{i = 1}^m \subseteq \bb{R}^n$ and a set of unit vectors ${\sset{e_{\gamma_i}}}_{i = 1}^m \subseteq \bb{R}^n$ selected from indices ${\sset{\gamma_i}}_{i = 1}^m \subseteq \bb{N}$.
In order to compute the indices $\gamma_i$, we apply Algorithm~\ref{alg:deim}.

\begin{algorithm}\label{alg:deim}
    \caption{\ac{DEIM}, adapted from~\cite[Algorithm~1]{Chaturantabut2010} and~\cite[Algorithm~4]{Chaturantabut2016}}
    \KwData{${\sset{}}_{i = 1}^m \subseteq \bb{R}^n$ linearly independent}
    \tcc{Compute initial index}
    $\gamma_1 \coloneqq \min\limits_{\gamma \in \bb{N}}{\abs{u_1^{(\gamma)}}{}},\quad U \coloneqq (u_1),\quad E \coloneqq (e_{\gamma_1})\quad \gamma \coloneqq (\gamma_1)$\;
    \For{$i = 2, \dots, m$}{
        \tcc{Update vector}
        Solve the \ac{LSQ} problem $(E\trans U) c = E\trans u_i$ for $c$\;
        $r \coloneqq u_i - Uc$\;
        \tcc{Compute next index}
        $\gamma_i \coloneqq \min\limits_{\gamma \in \bb{N}}{\abs{r^{(\gamma)}}{}},\quad U \coloneqq \begin{pmatrix}
            U & u_i
        \end{pmatrix},\quad E \coloneqq \begin{pmatrix}
            E & E_{\gamma_i}
        \end{pmatrix},\quad \gamma = \begin{pmatrix}
            \gamma \\
            \gamma_i
        \end{pmatrix}$\;
    }
\end{algorithm}

When it comes to applying \ac{DEIM} to the \ac{PH} system~\eqref{eq:nonquadratic-ph-system}, we want to consider the state coupling $Q(x)$ as the nonlinear function, see~\cite[Section~3]{Chaturantabut2016}.
First, we identify the any linear components of $Q$ such that for a matrix $M \in \bb{R}^{n \times n}$ and a nonlinear function $q \colon \bb{R}^n \to \bb{R}^n$ it holds that
\begin{equation}\label{eq:deim-decomposition}
    Q(x) = M x + q(x).
\end{equation}
Thereafter we select a new modelling basis orthonormalized w.r.t.\ $M$ so that the nonlinear part $q$ satisfies $\nabla_x q(x(t)) \approx U g(t)$ and $U\trans M U = \id$ for some $g(t) \in \bb{R}^m$.
We then apply Algorithm~\ref{alg:deim}, compute $\bb{P} \in \bb{R}^{n \times n}$, and apply the projection to the nonlinear part $Q(x) \approx M x + \bb{P} q\left( \bb{P}\trans x \right)$.
The projection step on $q$ reduces the number of nonlinear evaluations to $m$ components of $q$ on $m$ entries of $x$ each, thus allowing for a reduction in overall computational effort.

Finally, state reduction takes place via \ac{POD}, a well known and researched \ac{MOR} technique, see e.g.~\cite{Pinnau2008}.
We compute a projection matrix $V_\msc{r} \in \bb{R}^{n \times r}$ for a reduced dimension $r \in \bb{N}$, and calculate the reduced state coupling
\begin{equation*}
    Q_\msc{r}(x_\msc{r}) = V_\msc{r}\trans M V_\msc{r} x_\msc{r} + V_\msc{r}\trans  \bb{P} q\left( \bb{P}\trans V_\msc{r} x_\msc{r} \right).
\end{equation*}
Because the evaluation of $V_\msc{r}\trans U$ can be precomputed, we only need to evaluate the $m$ \ac{DEIM} indices, thus further reducing the online computational load.

\section{Abstract Limitations of Linear MOR}\label{sec:limitations-linear-mor}

This section is not directly linked to the scope of this thesis; however, it demonstrates an important motivation for the next chapter.
We therefore think it reasonable to discuss the limitations of linear MOR for \acp{PDE}.
In effect, we want to carry this motivation onto the problem inherent in this manuscript; however, no similar results to the ones we present in this section are available at the time of writing.
We therefore commence with the description of a greedy reduction algorithm and afterwards discuss what implications we derive for the abstract Kolmogorov $N$-width.

Unlike the truncation-based or interpolation-based methods presented in Section~\ref{sec:system-mor}, greedy algorithms rely on a solution snapshot based linear subspace approximation, see e.g.~\cite{Grepl2005, Rozza2008, Buffa2012}.
In this setting, solution snapshots are defined as individual solutions $u(\mu_i) \in \bb{R}^n$ of the parametric \ac{PDE}
\begin{equation}\label{eq:parametric-pde}
    A(\mu) u = L(\mu),
\end{equation}
with the operators $A(\mu) \in \bb{R}^{n \times n}, L(\mu) \in \bb{R}^n$ defined on a parameter space $\mcl{P} \subseteq \bb{R}^P$ for differing parameters $\mu \in \mcl{S}$ in a sample set $\mcl{S} \subseteq \mcl{P}$ with $\abs{\mcl{S}}{} = r \in \bb{N}$ elements.
Using a greedy approach, we choose a momentarily optimal basis element to improve the approximation in each iteration instead of computing an optimal basis for the entire reduced model.

The idea is as follows: Given a sample set of parameters $\mcl{S}$ and an associated reduced linear subspace $V_r \subseteq \bb{R}^r$ spanned by the solution snapshots $u(\mu_i), i = 1, \dots, r$, we then want to choose the next parameter sample $\mu_{r + 1}$ such that the \ac{ROM} solution $u_\msc{r}(\mu_{r + 1})$ creates the largest error over all possible parameters with respect to the full order solution $u(\mu_{r + 1})$.
Afterwards, we add the basis element to our reduced subspace by computing the linear span $V_{r + 1} = \lspn{V_r, u(\mu_{r + 1})}$.
Because $u_\msc{r}(\mu_{r + 1})$ induces the largest error with respect to the previous basis $V_r$, we maximize the reduction in error over all selected sampling parameters in every iteration; however, this cannot be guaranteed to result in the optimal basis for every problem.
An algorithmic depiction of the greedy method is given in Algorithm~\ref{alg:greedy}.

\begin{algorithm}\label{alg:greedy}
    \caption{Greedy Algorithm, adapted from~\cite[Algorithm~1]{Buffa2012}}
    \KwData{Parameter domain $\mcl{P}$, full order solution map $u \colon \mcl{P} \to \mcl{X}$}
    \tcc{Initialize greedy reduced basis}
    $\mu_1 \coloneqq \argmax\limits_{\mu \in \mcl{P}} \norm{u(\mu)}{\mcl{X}},\quad V_1 \coloneqq \lspn{u(\mu_1)} \subseteq \mcl{X},\quad r \coloneqq 1$\;
    \While{$\argmax\limits_{\mu \in \mcl{P}} \norm{u(\mu) - u_\msc{r}(\mu)}{\mcl{X}} \geq \tol$}{
        $r \coloneqq r + 1$\;
        \tcc{Find parameter with largest error}
        $\mu_{r} \coloneqq \argmax\limits_{\mu \in \mcl{P}} \norm{u(\mu) - u_\msc{r}(\mu)}{\mcl{X}},\quad V_{r} \coloneqq \lspn{V_r, u(\mu_{r})}$\;
    }
\end{algorithm}

With the greedy basis ${\sset{u_i = u(\mu_i)}}_{i = 1}^r$ computed with Algorithm~\ref{alg:greedy} we form the projection matrix $U = \left( u_1, \dots, u_r \right) \in \bb{R}^{n \times r}$ spanning the reduced space $V_r$ through linear combinations of its columns.
Using this matrix to project the full order space $\bb{R}^n$ onto the linear subspace $\bb{R}^r$, we can transform an exemplary full order discretized \ac{PDE} such as~\eqref{eq:parametric-pde} to a \ac{ROM} by projecting the operators
\begin{equation*}
    \tilde{A} = U\trans A U,\quad \tilde{L} = U\trans L.
\end{equation*}
To bound the error of the reduced solutions, we have to consider so-called $N$-widths as defined in~\cite{Pinkus1985}.
For linear approximations of coercive elliptic problems defined on the parameter space $\mcl{P}$ we can provide an abstract measure of the approximation quality of the reduced space $V_\msc{r}, \dim{(V_\msc{r})} = r \in \bb{R}$ with respect to the full order space $V \subseteq \bb{R}^n$.
This is achieved by bounding the \ac{ROM} projection error with the Kolmogorov N-width of the solution manifold $\mcl{M} = \iset{u(\mu) \in \bb{R}^n}{\mu \in \mcl{P}}$
\begin{equation}\label{eq:kolmogorov-n-width}
    \kolm{\mcl{M}} \coloneqq \eqinf\limits_{\substack{V_\msc{r} \subseteq V,\\\dim{(V_\msc{r})} = r}} \sup\limits_{u \in \mcl{M}} \eqinf\limits_{u_\msc{r} \in V_N} \norm{u - u_\msc{r}}{}.
\end{equation}
For greedy algorithms applied to linear coercive elliptic \acp{PDE} with sufficiently smooth initial conditions, a fast decay of the Kolmogorov $N$-width~\eqref{eq:kolmogorov-n-width} is observed; for an introduction into the approximation theory see e.g.~\cite{Binev2011, DeVore2013}.
It can be shown that if the solutions $u(\mu)$ depend analytically on $\mu$, then there exist constants $\alpha, c > 0$ such that the Kolmogorov $N$-width is bounded from above by exponential decay
\begin{equation*}
    \kolm{\mcl{M}} \leq c \exp{(-N^\alpha)}.
\end{equation*}

However, not all problems allow for this kind of upper bound.
When considering advection dominated problems such as~\cite[Section~5.1]{Ohlberger2016}, the restriction on the Kolmogorov $N$-width~\eqref{eq:kolmogorov-n-width} of linear advection problems with jump discontinuities instead is a lower bound
\begin{equation*}
    \kolm{\mcl{M}} \geq \frac{1}{2} N^{- \frac{1}{2}}.
\end{equation*}
Similarly, as demonstrated in~\cite{Greif2019}, the Kolmogorov $N$-width~\eqref{eq:kolmogorov-n-width} of the hyperbolic wave equation on the unit interval with the initial condition $u_0(x) \coloneqq \begin{cases}
    1, & x < 0 \\
    -1, & x \geq 0
\end{cases}$ exhibits a decay bounded from below
\begin{equation*}
    \kolm{\mcl{M}} \geq \frac{1}{4} N^{- \frac{1}{2}}.
\end{equation*}

While the slow decay has an effect on the increase in approximation quality with respect to the \ac{ROM} dimension if the projection error can be bounded by the Kolmogorov $N$-width, the existence of such results is by no means guaranteed and can therefore only be understood as an indication.
Similarly, obtaining an upper bound on the projection error through~\eqref{eq:kolmogorov-n-width} does not guarantee that we can preemptively determine a reduced order for which a desired error tolerance is achieved; however, a slow decay of the $N$-width often results in the need for larger \acp{ROM}.
The drawback of these bases increasing in dimension is the simultaneous increase in computational cost, hence diminishing the (online) speedup obtained from reducing the model.
The consequence of these findings is immediate: to preserve the speedup obtained from \ac{MOR}, we need to incorporate nonlinear parts into the construction of the \acp{ROM}.

\section{MOR with \aclp{NN}}\label{sec:nn-mor}

In recent years, \acp{NN} have been introduced into a variety of numerical applications to allow for nonlinear solutions to many problems, among them \ac{MOR} methods.
This section aims to explain how structures from deep \acp{NN} can be used in reducing \acp{FOM}.
Thus, we inherently rely on some basic knowledge in the structure of \acp{NN} and \acp{DCAE}.
For references on \acp{NN} and Deep Learning we refer to sources such as~\cite{Goodfellow2016, Kubat2017, Sarker2021}, as well as the other citations therein.

To approximate a function $f \colon \bb{R}^n \to \bb{R}$ or, more generally, an operator $A \colon \bb{R}^n \to \bb{R}^m$ with \acp{NN}, the general procedure consists of first constructing a parametrized network $A_\theta$ intended to approximate the operator $A$ and then training the network by successively updating the parameter $\theta$.
We obtain these updates by evaluating an approximate gradient of a loss function which ressembles a measured error between the output of the original operator $A$ and its computed counterpart as the output of $A_\theta$.
By the choice of nonlinear activation functions in the network's hidden layers, this method can produce very accurate nonlinear models while being easily parallelizable.
However, it requires both large amounts of data as well as appropriate design choices in the structure of the \ac{NN} to generalize from restrictive input data to more complex verification data.

Besides the ability to mimic operator behaviour, \ac{NN} architectures can recreate effects akin to \ac{MOR}.
To this end, modern deep \ac{NN} techniques are employed to inherently produce \acp{NN} that in turn ressemble a low order network.
In recent publications such as~\cite{Lee2020, Salvador2021, Benner2022, Kim2022, Buchfink2023}, \acp{DCAE} are introduced.
These networks consist of two parametrized components: an encoder $e_\theta \colon \bb{R}^n \to \bb{R}^r$ and a decoder $d_\theta \colon \bb{R}^r \to \bb{R}^n$.
Here, the encoder serves as a means to reduce the state order by computing $x_\msc{r} = e_\theta(x) \in \bb{R}^r$, and the decoder acts as the reconstruction of the reduced state back to the \ac{FOM} state $\hat{x} = d_\theta(x_\msc{r})$.
To train the encoder-decoder pair, we typically choose a loss function that corresponds to the projection error over a sample set $\mcl{S}$ of full order states
\begin{equation}\label{eq:reduction-loss}
    \mcl{L}_{\msc{data}}(\theta) \coloneqq \frac{1}{\abs{\mcl{S}}{}} \sum\limits_{x \in \mcl{S}} \norm{x - d_\theta\left( e_\theta(x) \right)}{}^2.
\end{equation}
In addition to the loss function $\mcl{L}_{\msc{data}}$, we can also enforce systemic constraints on the components of the \acp{ROM}.
As an example, the approach highlighted in~\cite{Buchfink2023} introduces a second loss function $\mcl{L}_{\msc{sympl}}(\theta)$ that weakly enforces the final \ac{ROM}'s symplecticity, a property closely related to the physically motivated nature of \ac{PH} systems.
Then, the combined loss function $\mcl{L} \coloneqq \alpha \mcl{L}_{\msc{data}} + (1 - \alpha) \mcl{L}_{\msc{sympl}}$ not only generates a reduced order model but also weakly forces a symplectic structure of the \ac{ROM}.

When we consider \ac{NN}-based approximation, we arrive at a clear distinction between the offline and the online phases: we have to perform expensive calculations while training the model offline, and after the reduction we can efficiently evaluate the structure of the network to get results in the online evaluation phase.
Unfortunately, the training process conceptually obfuscates how the low order state representation $e_\theta(x)$ is connected to the full order state $x$.
This results in sophisticated nonlinear models that are entirely unexplainable.
We aim to circumnavigate the unexplainability introduced in the \ac{NN} setting, which means that we have to consider methods acting directly on the dynamical systems.

\section{MOR on Quadratically Embedded Manifolds}\label{sec:mor-quadratically-embedded-manifolds}

In recent years, advances have been made in computing nonlinear models without the aid of \acp{NN}.
These methods take many different forms: from explicitly substituting polynomial terms in~\cite{Gu2011} to modal derivatives such as in~\cite{Wu2016, Weeger2016, Jain2017, Rutzmoser2017}, or from fitting polynomial system components in various different forms such as through lifting transformations as described in~\cite{Kramer2019, Qian2022} to bestapproximating from polynomially embedded data terms as detailed in~\cite{Peherstorfer2016, BGK2020, Gosea2021, Barnett2022, Geelen2023}.
Most importantly, the inference approaches described in these publications allow for simple \ac{LSQ} solutions of the inference problems, resolving the explainability constraint of the \ac{NN} methods mentioned in Section~\ref{sec:nn-mor}.
For this section, we focus on the formulation introduced in~\cite{Geelen2023} as a foundational device to derive \ac{PH} applications later on.

In the linear MOR framework, the full state variables $x \in \bb{R}^n$ are reduced by linear projection onto the span of a low-dimensional subspace $V_\msc{r} \subseteq \bb{R}^r$ such that we can approximate
\begin{equation}\label{eq:linear-state-reduction}
    x \approx x_\msc{ref} + V x_r,\quad V \in \bb{R}^{n \times r},
\end{equation}
where $x_\msc{ref} \in \bb{R}^n$ serves as a reference state to allow for normalization of the state data, the choice of which is dependent on the problem under consideration.
There exist multiple options to construct the projection matrix $V \in \bb{R}^{n \times r}$, however, these methods are outside the scope of this thesis, which is why we refer the interested reader to~\cite{BOP2017, BCO2017} and the references contained therein.
We measure the reduction error for the state snapshot matrices $X \coloneqq (x_1, \dots, x_k), X_\msc{ref} \coloneqq {(x_\msc{ref})}_{i = 1}^k \in \bb{R}^{n \times k},k \in \bb{N}$, by computing the projection error of the reduced states
\begin{equation}
    \mcl{E} \coloneqq (\id - V V\trans) (X - X_\msc{ref}) \in \bb{R}^{n \times k}.
\end{equation}
In order to construct the polynomially embedded manifold model, we compare the full order and the reduced states at every sample point in time $t_i \in I \subseteq \bb{R}_{\geq 0}$ through the signed error
\begin{equation}\label{eq:linear-signed-error}
    \varepsilon(t_i) \coloneqq x(t_i) - x_\msc{ref} - V x_\msc{r}(t_i),\quad i = 1, \dots, k.
\end{equation}
Crucially, we now get to choose how we want to approximate the signed error $\varepsilon$.
Theoretically, we could choose an arbitrary linear combination of nonlinear transformations of the reduced state $x_\msc{r}$, but for simplicity we constrain ourselves to the quadratic terms $x_\msc{r} \odot x_\msc{r}$.
Here, the operator $\odot$ denotes the Khatri--Rao product, also known as the column-wise Kronecker product as described in~\cite{Slyusar1999, Shuangzhe2008, Favier2021}, which acts on a state variable $x_\msc{r} = {\left( \hat{x}_{(1)}, \dots, \hat{x}_{(r)} \right)}\trans$ in the following manner
\begin{equation}\label{eq:redundant-khatri-rao}
    x_\msc{r} \odot x_\msc{r} \coloneqq {\left( \hat{x}_{(1)}^2, \hat{x}_{(1)}^{\phantom{1}} \hat{x}_{(2)}^{\phantom{1}}, \dots, \hat{x}_{(1)}^{\phantom{1}} \hat{x}_{(r)}^{\phantom{1}}, \hat{x}_{(2)}^{\phantom{1}} \hat{x}_{(1)}^{\phantom{1}}, \hat{x}_{(2)}^2, \dots, \hat{x}_{(r)}^2 \right)}\trans \in \bb{R}^{r^2}.
\end{equation}
In practice, this formulation contains many duplicate entries such as $\hat{x}_{(1)}^{\phantom{1}} \hat{x}_{(2)}^{\phantom{1}}$ and $\hat{x}_{(2)}^{\phantom{1}} \hat{x}_{(1)}^{\phantom{1}}$, thus we can omit these terms without losing any information in the quadratic state.
This results in the reformulation of the Khatri--Rao product as the non-redundant version
\begin{equation}\label{eq:khatri-rao}
    x_\msc{r} \odot x_\msc{r} \coloneqq {\left( \hat{x}_{(1)}^2, \hat{x}_{(1)}^{\phantom{1}} \hat{x}_{(2)}^{\phantom{1}}, \dots, \hat{x}_{(1)}^{\phantom{1}} \hat{x}_{(r)}^{\phantom{1}}, \hat{x}_{(2)}^2, \hat{x}_{(2)} \hat{x}_{(3)}, \dots, \hat{x}_{(r)}^2 \right)}\trans \in \bb{R}^{\frac{r (r + 1)}{2}}.
\end{equation}
By abuse of notation, we will henceforth consider the ordinary Khatri--Rao product~\eqref{eq:redundant-khatri-rao} in theoretic discussions while using the non-redundant version~\eqref{eq:khatri-rao} in numerical applications unless otherwise specified.
This replacement results in no changes to those of the product's properties we would need to use later on.
To ease the handling of the different dimensions invoked by both~\eqref{eq:redundant-khatri-rao} and~\eqref{eq:khatri-rao}, we denote the quadratic data dimension as $q \in \bb{N}$ such that $x_\msc{r} \odot x_\msc{r} \in \bb{R}^q$ no matter which version of the product we use.

Next, we must compute the quadratic projection operator $W \in \bb{R}^{n \times q}$.
In view of the fact that most methods emplying polynomially embedded manifolds use non-intrusive modelling approaches, it is necessary to come up with a data-driven solution to finding $W$ without accessing the underlying system matrices of the \ac{FOM} or linear \ac{ROM}.
We take a cue from \ac{LSQ} data fits and similarly to the signed error $\varepsilon$ from~\eqref{eq:linear-signed-error} minimize the cost functional
\begin{equation}\label{eq:quadratic-least-squares-sum}
    \sum\limits_{i = 1}^k \norm{x(t_k) - x_\msc{ref} - V x_\msc{r}(t_k) - W (x_\msc{r}(t_k) \odot x_\msc{r}(t_k))}{2}^2.
\end{equation}
If we transpose all summands within the norm in~\eqref{eq:quadratic-least-squares-sum} and form the data matrix of quadratically embedded reduced states $Q \coloneqq {\big( x_\msc{r}(t_1) \odot x_\msc{r}(t_1), \dots, x_\msc{r}(t_k) \odot x_\msc{r}(t_k) \big)} \in \bb{R}^{q \times k}$, then we can equivalently solve the following simultaneous \ac{LSQ} problem instead of the $k$ individual problems in~\eqref{eq:quadratic-least-squares-sum}
\begin{equation}\label{eq:quadratic-least-squares}
    W = \argmin\limits_{W \in \bb{R}^{n \times q}} \frac{1}{2} \norm{Q\trans W\trans - \mcl{E}\trans}{F}^2.
\end{equation}
This problem is overdetermined if $Q$ has full row rank, meaning $\rank{Q} = q$, which means that as a necessary condition the quadratic data has to satisfy $k \geq q$.
Practically, we can separate problem~\eqref{eq:quadratic-least-squares} into $n$ vector-valued problems through~\eqref{eq:quadratic-least-squares-sum}, allowing for a parallel solution algorithm of each individual \ac{LSQ} problem.
In order to numerically stabilize the problem~\eqref{eq:quadratic-least-squares}, we can introduce additional regularization terms as demonstrated in~\cite[Equation~15]{Geelen2023}.
Some of the terms the authors discuss in~\cite{Geelen2023} use Frobenius norms, whereas others like the $2$-norm serve to induce sparsity by groups in the quadratic projection matrix, effectively creating column sparsity in $W$.

Similar proposals have been made in~\cite{Peherstorfer2016}, covering \ac{LTI} systems that, additionally to the formulation in Definition~\ref{def:lti}, depend on the quadratically embedded system states.
Furthermore, the authors prove in~\cite[Theorem~1, Corollary~1]{Peherstorfer2016} that the inferred operators converge to intrusively reduced operators for vanishing time steps, implying that data from stable systems eventually allows to infer stable systems.
Unfortunately, this direct approach cannot be applied to the inference of \ac{PH} systems because the structural constraints~\eqref{eq:ph-matrix-structure} on the system matrices do not allow for a straightforward solution of the \ac{LSQ} problem via the Moore-Penrose pseudo-inverse of the data matrix.

