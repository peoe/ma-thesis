\section{Inferring port-Hamiltonian Systems}\label{sec:inferring-ph-systems}

We introduce several methods to infer port-Hamiltonian systems in a nonintrusive manner throughout this section.
First, we commence by a hybrid approach in Subsection~\ref{subsec:realization-based-inference}: We apply a nonintrusive method to obtain an LTI system and afterwards compute a port-Hamiltonian realization from it.
Secondly, we discuss the interpolation-based \emph{Loewner Approach} in Subsection~\ref{subsec:loewner-approach} which acts on transfer function data similar to Subsection~\ref{subsec:interpolation-reduction}.
Lastly, there exist a number of optimization-based inference methods similar to the problems in Section~\ref{sec:mor-quadratically-embedded-manifolds} which we highlight in Subsection~\ref{subsec:optimization-based-inference}.

\subsection{Realization-Based Inference}\label{subsec:realization-based-inference}

Realization-based Operator Inference works as a two step procedure:
\begin{enumerate}
    \item Use the provided data $x_i, u_i, y_i, i = 1, \dots, k$ to infer an LTI realization $\Sigma_\msc{lti} \colon (A, B, C, D, E)$, and
    \item Compute a port-Hamiltonian realization $\Sigma_\msc{ph}$ from $\Sigma_\msc{lti}$.
\end{enumerate}
This approach thus is inherently hybrid, because the second step requires access to the system matrices, and cannot be considered purely nonintrusive.
There are also quite a few downsides to this way of approximating a system.
If the second step is not very robust with respect to the input data, then even small numerical errors or noise in the measurements can cause the result to be nonsensical.
We thus only consider this framework as a theoretical stepping stone to introduce later more direct ideas.

As first step~\cite{Miller2012, Gosea2021, Heiland2022, Peherstorfer2016, Peherstorfer2017}.
Sindy~\cite{Brunton2016, Kaiser2018, Kaheman2020, Lee2022}?
qwer.~\cite{Cherifi2019, Beattie2022}.

\subsection{The Loewner Approach}\label{subsec:loewner-approach}

asf.~\cite{Antoulas2019, BGD2020, Cherifi2022, Peherstorfer2017, Poussot2022, GKA2021}.

\subsection{Optimization-Based Inference}\label{subsec:optimization-based-inference}

zxcv.~\cite{Gillis2018, Günther2023, Morandin2022, Najnudel2021, Schwerdtner2021, SV2021, Schwerdtner2022, Schwerdtner2023}.
