\section{Numerical Examples}

\subsection{Burgers Equation}

For Burgers' equation, the state derivative of the Hamiltonian is $\pd \mcl{H}(x) = \frac{x^2}{2}$, cf.~\cite{Maschke2005}.
This term is clearly not derived from a quadratic Hamiltonian, so some different methods have to be used.
Alternatively, Linearly approximate first and then try to apply the quadratic data such that the system Hamiltonian state derivative turns into $\pd \mcl{H}(x) = \frac{x}{2}$.
This in a sense reflects the idea of lifting system coordinates, cf.~\cite{Qian2020, Kramer2019}.
Alternatively, one might also have a look at the Carleman linearization process, cf.~\cite{Goyal2015, Rugh1981}.
There is also the idea of QLMOR which transforms entire systems of nonlinearities into linear systems in which some variables represent the prior nonlinear terms, cf.~\cite{Gu2011}.
From the PDE theoretical perspective the Hopf-Cole transformation may be of interest to linearize the viscuous Burgers' equation, cf.~\cite{Hopf1950, Cole1951, Kannan2012, Chen2016}.

\subsection{Damped Wave Equation}

Wave equations can be modelled in different ways such that the resulting formulations are attractive to pH sytems.
Firstly, there is the Friedrichs' system perspective, cf.~\cite{Friedrichs1958, Antonic2014, Mifsud2016}.
Secondly, direct pH forms are given in~\cite{Cardoso2018, Serhani2019}.

Because the undampened wave equation is a linear equation, the Hamiltonian is always going to be quadratic.
With regards to dampening, it should be easy to use the control variable $u$ as dampening action on the system, however an inherent state dampening (boundary values?) might also be interesting.

\subsection{Friedrichs Systems}

The usual sources all use the usual model problems: scalar elliptic equations, heat, wave (homogeneous/inhomogeneous), Laplace --- cf.~\cite{Friedrichs1958, Burman2010, Antonic2013, Antonic2014, Mifsud2016, Erceg2022}.
There also exist formulations of the MHD equations, cf.~\cite{Ern2006}.

\itodo{choose specific problem}
