\section{Port-Hamiltonian Dynamic Mode Decomposition}

\begin{frame}{Port-Hamiltonian Dynamic Mode Decomposition, cf.~\cite{Morandin2022}}
    \begin{alertblock}{Question}
        How to compute $\mcl{J}, \mcl{R}$ from data $x_i, (\dot{x}_i), y_i, u_i$?
    \end{alertblock}

    \uncover<2->{Operator Inference!}

    \uncover<3->{$\rightarrow$ Port-Hamiltonian Dynamic Mode Decomposition (\cite{Morandin2022})}

    \uncover<4>{Nice side effect:\ The inferred system will always be passive.}
\end{frame}

\begin{frame}{Port-Hamiltonian Dynamic Mode Decomposition, cf.~\cite{Morandin2022}}
    \metroset{block=transparent}
    \begin{block}{Data}
        \begin{columns}[totalwidth=\linewidth]
            \begin{column}{.49\textwidth}
                \begin{itemize}
                    \item $k \in \bb{N}$ time steps
                    \item $X^- = \frac{1}{\Delta t} \left( x_1 - x_0, \dots, x_k - x_{k - 1} \right)$
                    \item $X^+ = \frac{1}{2} \left( x_1 + x_0, \dots, x_k + x_{k - 1} \right)$
                \end{itemize}
            \end{column}
            \begin{column}{.49\textwidth}
                \begin{itemize}
                    \item[] \phantom{}
                    \item $U = \frac{1}{2} \left( u_1 + u_0, \dots, u_k + u_{k - 1} \right)$
                    \item $Y = \frac{1}{2} \left( y_1 + y_0, \dots, y_k + y_{k - 1} \right)$
                \end{itemize}
            \end{column}
        \end{columns}
    \end{block}

    \uncover<2>{
        \metroset{block=fill}
        \begin{block}{\color{petrol}Port-Hamiltonian Dynamic Mode Decomposition}
            \begin{equation*}
                \mcl{J}, \mcl{R} = \argmin\limits_{\mcl{J} = -\mcl{J}\trans, \mcl{R} = \mcl{R}\trans \succcurlyeq 0} \norm[F]{\begin{bmatrix}
                    V\trans H V V\trans X^- \\
                    -Y
                \end{bmatrix} - \left( \mcl{J} - \mcl{R} \right) \begin{bmatrix}
                    V\trans X^+ \\
                    U
                \end{bmatrix}}
            \end{equation*}
        \end{block}

        Solution by an iterative algorithm, cf.~\cite{Morandin2022}.
    }
\end{frame}