\chapter*{Abstract}

This thesis describes an approach to combine recent advances in Model Order Reduction and Systems Theory.
In particular, we are going to examine quadratically embedded manifolds for reduced models and apply the ideas thereof to port-Hamiltonian systems.
The inherent goal of this combination is to couple physically motivated descriptions of dynamical systems to ideas mitigating certain restrictions in approximation theory.
We commence by describing aspects of the underlying theory in systems theory, highlighting reduced modelling and its constraints before closing with our proposed method in combination with numerical experiments.

The necessary background in systems theory will focus on the class of port-Hamiltonian systems, a subset of dynamical systems incorporating both energry preserving and energy dissipating elements.
This description is particularly interssting because port-Hamiltonian systems consider two aspects of a model: the internal \emph{states} and \emph{ports} through which the system and the outside environment can interact with each other by so called \emph{controls} and \emph{outputs}.
Modelling in such a manner makes it incredibly easy to combine multiple systems into a more complex large scale system.

Reducing port-Hamiltonian systems unfortunately is not as easy as reducing systems with less structure, however appropriate methods exist in literature.
Usually, they are more involved than the basic model reduction algorithms from whom they derive because more caution has to be taken to keep mathematical structures intact.
The benefit of choosing among these structure-preserving methods is that the individual components of the full order port-Hamiltonian system keep their properties: energy dissipating elements remain energy dissipating even after the model reduction, and similarly for the other components.

Most classical Model Order Reduction approaches suffer from the fact that their approximative qualities are directly linked to the type of full order model.
This can result in very good decays of the so-called \emph{Kolmogorov N-width} for elliptic parameter dependent problems, but also in very slow decay rates for more unforgiving models such as wave equations.
To bypass this constraint, recent publications have proposed methods taken from machine learning such as neural networks and autoencoders.
These deliver good results but are computationally less efficient during the offline (read: training) phases and lack the simple explainability of linear models.
To incorporate more explainable terms and sill imporve model error rates, polynomial terms have been included in usually linear model reduction methods.

The goal of this thesis is to extend reduced models with quadratic terms to port-Hamiltonian systems.
Along the way we make use of a few different frameworks and highlight how they all come together.
We end this work with numerical experiments showcasing a fww model problems and discussing the results attained.
