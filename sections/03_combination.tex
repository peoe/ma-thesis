\section{Combining Both Approaches}

When constructing a quadratic manifold in a pH system, many approaches might be taken:
\begin{enumerate}
    \item ``quadraticize'' the state variables $\varPhi\trans \mbf{X}$ and $\varPhi\trans H \varPhi \varPhi\trans \dot{\mbf{X}}$,
    \item ``quadraticize'' the control variables $\mbf{U}$,
    \item ``quadraticize'' the output variables $\mbf{Y}$, or
    \item any viable combination of these.
\end{enumerate}

For now we'll stick to quadratic state variables.
Quadratizising the control or the outputneabs having to select components of the output for combining linear and quadratic models.

\subsection{The Linear Model}\label{sec:lin-model}

\itodo{if needed estimate the linear model using op inf}

\subsection{The Quadratic Model}\label{sec:quad-model}

We start of with a linearly approximated model
\begin{equation*}
    \begin{bmatrix}
        \mbf{H} \dot{\mbf{X}} \\
        -\mbf{Y}
    \end{bmatrix} = \underbrace{(\mcl{J}^{(L)} - \mcl{R}^{(L)})}_{\coloneqq \mcl{L}} \begin{bmatrix}
        \mbf{X} \\
        \mbf{U}
    \end{bmatrix}, \mcl{L} = \begin{bmatrix}
        \mcl{L}_{11} & \mcl{L}_{12} \\
        \mcl{L}_{21} & \mcl{L}_{22}
    \end{bmatrix}
\end{equation*}
where $\mcl{L}$ has to be split into the various matrix blocks to match the individual variables.

We now proceed as follows:
\begin{enumerate}
    \item Compute a trajectory $\mbf{H} \mbf{X}_{\textsc{lin}}, \dot{\mbf{X}}_{\textsc{lin}}, \mbf{Y}_{\textsc{lin}}$ from the linear system.
    \item Compute quadratic data matrices $(\mbf{H} \mbf{X}_{\textsc{lin}}) \diamond (\mbf{H} \mbf{X}_{\textsc{lin}}), \mbf{X}_{\textsc{lin}} \diamond \mbf{X}_{\textsc{lin}}$.
    \item Identify the system matrices $\mcl{J}^{(Q)} - \mcl{R}^{(Q)} = \mcl{Q}$ of the quadratic pH system~\eqref{eq:quad-ph}.
\end{enumerate}

\begin{equation}\label{eq:quad-ph}
    \begin{bmatrix}
        (\mbf{H} \dot{\mbf{X}}_{\textsc{lin}}) \diamond (\mbf{H} \dot{\mbf{X}}_{\textsc{lin}}) \\
        - \mbf{Y}
    \end{bmatrix} = \mcl{Q} \begin{bmatrix}
        \mbf{X}_{\textsc{lin}} \diamond \mbf{X}_{\textsc{lin}} \\
        \mbf{U}
    \end{bmatrix}, \mcl{Q} = \begin{bmatrix}
        Q_{11} & Q_{12} \\
        Q_{21} & Q_{22}
    \end{bmatrix}
\end{equation}

To identify the system~\eqref{eq:quad-ph}, there are multiple approaches.
The usual idea is to somehow interpolate the underlying transfer function of the pH system.
In the simplest way, direct frequency measurements of the system are taken as the interpolation points, cf.~\cite{Gillis2018, Cherifi2019, Benner2020, Schwerdtner2021}.
Instead of handling direct frequency interpolation it is also possible to start from the time domain.
The idea here is to perform a DFT on the time ``snapshots'', cf.~\cite{Peherstorfer2017, Cherifi2021, Cherifi2022}.
However, this approach still cannot directly operate on data alone, an intermediate LTI system (through the transfer function).
In contrast, the extension of DMD to pH proposed in~\cite{Morandin2022} can direclty run on system data.
While pH DMD does not provided many guarantees as of right now, it allows for easy application to our problem.

\subsection{Constructing the Combined Model}

We begin by noting that the properties of the Khatri-Rao product allow the following decomposition:
\begin{equation*}
    (H \dot{\mbf{X}}) \diamond (H \dot{\mbf{X}}) = (H \otimes H) (\dot{\mbf{X}} \diamond \dot{\mbf{X}}),
\end{equation*}
cf.~\cite[Equation~2.256]{Favier2021}.

We then combine both the linear model from Section~\ref{sec:lin-model} and the quadratic model from Section~\ref{sec:quad-model} by adding the output terms from both systems to the total output of the system.

\begin{equation*}
    \begin{bmatrix}
        \mbf{H} \dot{\mbf{X}}_{\textsc{lin}} \\
        (\mbf{H} \dot{\mbf{X}}_{\textsc{lin}}) \diamond (\mbf{H} \dot{\mbf{X}}_{\textsc{lin}}) \\
        - \mbf{Y}
    \end{bmatrix} = \begin{pmatrix}
        L_{11} & 0 & L_{12} \\
        0 & Q_{11} & Q_{12} \\
        L_{21} & Q_{21} & L_{22} + Q_{22}
    \end{pmatrix}
    \begin{bmatrix}
        \mbf{X}_{\textsc{lin}} \\
        \mbf{X}_{\textsc{lin}} \diamond \mbf{X}_{\textsc{lin}} \\
        \mbf{U}
    \end{bmatrix}
\end{equation*}
