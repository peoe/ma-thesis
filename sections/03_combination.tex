\section{Combining Both Approaches}

When constructing a quadratic manifold in a pH system, many approaches might be taken:
\begin{enumerate}
    \item ``quadraticize'' the state variables $\varPhi\trans \mbf{X}$ and $\varPhi\trans H \varPhi \varPhi\trans \dot{\mbf{X}}$,
    \item ``quadraticize'' the control variables $\mbf{U}$,
    \item ``quadraticize'' the output variables $\mbf{Y}$, or
    \item any viable combination of these.
\end{enumerate}

If we quadraticize the state variables we get the terms $\underline{\mbf{X}}, \underline{\dot{\mbf{X}}} \in \bb{R}^{r \times \frac{k (k + 1)}{2}}$.
Analogously, $\underline{\mbf{Y}}, \underline{\mbf{U}} \in \bb{R}^{m \times \frac{k (k + 1)}{2}}$.
We further define the concatenated vectors
\[
    \dot{\mcl{X}} =
    \begin{bmatrix}
        \varPhi\trans H \varPhi \varPhi\trans \mbf{\dot{X}} \\
        \underline{\dot{\mbf{X}}}
    \end{bmatrix},
    \mcl{X} =
    \begin{bmatrix}
        \varPhi\trans \mbf{X} \\
        \underline{\mbf{X}}
    \end{bmatrix},
    \mcl{Y} =
    \begin{bmatrix}
        \mcl{Y} \\
        \underline{\mcl{Y}}
    \end{bmatrix},
    \mcl{U} =
    \begin{bmatrix}
        \mbf{U} \\
        \underline{\mcl{U}}
    \end{bmatrix}.
\]

\subsection{Quadraticizing State Vectors}\label{subseq:quad-state-vecs}

If we only consider $\underline{\mbf{X}}$ and $\underline{\dot{\mbf{X}}}$ in the system we get
\begin{equation*}
    \mathcal{Z} =
    \begin{bmatrix}
        \underline{\dot{\mbf{X}}} \\
        -\mbf{Y}
    \end{bmatrix},
    \mathcal{T} =
    \begin{bmatrix}
        \underline{\mbf{X}} \\
        \mbf{U}
    \end{bmatrix} \in \bb{R}^{(r + m) \times \frac{k (k + 1)}{2}}
\end{equation*}
or
\begin{equation*}
    \mathcal{Z} =
    \begin{bmatrix}
        \varPhi\trans H \varPhi \varPhi\trans \mbf{\dot{X}} \\
        \underline{\dot{\mbf{X}}} \\
        -\mbf{Y}
    \end{bmatrix},
    \mathcal{T} =
    \begin{bmatrix}
        \varPhi\trans H \varPhi \varPhi\trans \mbf{X} \\
        \underline{\mbf{X}} \\
        \mbf{U}
    \end{bmatrix} \in \bb{R}^{(r + r + m) \times \frac{k (k + 1)}{2}}.
\end{equation*}
In both cases we would need to pad $\varPhi\trans H \varPhi \varPhi\trans \mbf{\dot{X}} \in \bb{R}^{r \times k}$ and $-\mbf{Y} \in \bb{R}^{m \times k}$ because the simple data dimensions no longer align with the quadratic ones.
It is also important to note that we somehow need to handle the quadraticized Hamiltonian.

\subsection{Quadraticizing Control and Output}

If we only consider $\underline{\mbf{U}}, \underline{\mbf{Y}}$, we basically run into the same issue as in Subsection~\ref{subseq:quad-state-vecs} though padding of the (normal) Hamiltonian.

\subsection{Quadraticizing State and Control/Output}

If we consider $\underline{\mbf{X}}, \underline{\dot{\mbf{X}}} \in \bb{R}^{r \times \frac{k (k + 1)}{2}}, \underline{\mbf{U}}, \underline{\mbf{Y}} \in \bb{R}^{m \times \frac{k (k + 1)}{2}}$ we can construct the following system:
\begin{equation*}
    \begin{bmatrix}
        \underline{\dot{\mbf{X}}} \\
        -\underline{\mbf{Y}}
    \end{bmatrix}
    = (\mcl{J} - \mcl{R})
    \begin{bmatrix}
        \underline{\mbf{X}} \\
        \underline{\mbf{U}}
    \end{bmatrix},
\end{equation*}
where $\mcl{J} = -\mcl{J}\trans, \mcl{R} \succcurlyeq 0 \in \bb{R}^{(r + m) \times (r + m)}$.

In particular, using the concatenated terms instead one can obtain
\begin{equation}\label{eq:concatenated-quadratic-ph-formulation}
    \begin{bmatrix}
        \dot{\mcl{X}} \\
        -\mcl{Y}
    \end{bmatrix}
    = (\mcl{J} - \mcl{R})
    \begin{bmatrix}
        \mcl{X} \\
        \mcl{U}
    \end{bmatrix}.
\end{equation}
This formulation allows for the direct recovery of the normal output variable $\mbf{Y}$ via correctly indexing the left hand side.

\subsection{Sequential vs.\ Simultaneaous Approximation}

The essential part of Equation~\eqref{eq:concatenated-quadratic-ph-formulation} is that this allows for both a sequential approach as well as a simultaneous approach:
\begin{itemize}
    \item In the simultaneous case, we just run phDMD~\cite{Morandin2022} on Equation~\eqref{eq:concatenated-quadratic-ph-formulation}.
    \item In the sequential approach, we first compute the solutions $\mcl{J}_\textsc{lin}, \mcl{R}_\textsc{lin}$ of Problem~\eqref{eq:phdmd-lst-problem}. Then plugging these into $\mcl{J} = \begin{bmatrix}
        \mcl{J}_\textsc{lin} & \mcl{J}_1 \\
        \mcl{J}_2 & \mcl{J}_3
    \end{bmatrix}, \mcl{R} = \begin{bmatrix}
        \mcl{R}_\textsc{lin} & \mcl{R}_1 \\
        \mcl{R}_2 & \mcl{R}_3
    \end{bmatrix}$, we compute the remaining matrices again using phDMD\@.
\end{itemize}

\itodo{non-quadratic Hamiltonians}