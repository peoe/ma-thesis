\section{Structure Preserving Linear MOR}\label{sec:structure-preserving-mor}

In Section~\ref{sec:system-mor} we covered some systemtheoretic linear MOR frameworks.
Now we want to cover some more aspects of linear MOR for \acp{LTI} by introducing structure preserving qualities into the Balanced Truncation and Interpolation-Based Reduction methods.
This is by no means easy because the restriction from the general set of matrices to more specific sets like the family of skew-symmetric or symmetric positive definite matrices causes the underlying optimization set to no longer be convex.
Thus, the corresponding optimization problems to find optimal systems become either hard to solve or the solutions may no longer be unique, adding locally optimal systems to the set of problems we have to face.

\subsection{Positive Real Balanced Truncation}\label{subsec:prbt}

The first algorithm we take a look at is positive real balanced truncation.
This method relies on the fact that positive real systems are closely linked to \ac{PH} systems by the fact that we can transform behaviourally controllable and behaviourally observable, that is minimal, positive real systems into \ac{PH} systems as we shall see later.

\begin{definition}[{Cf.~\cite[Definition~2.1]{Freund2004} or~\cite[Property~(PR)]{CGH2022}}]\label{def:positive-real}
    A \ac{LTI} system $\Sigma_\msc{lti}$ is said to be \emph{positive real} if its transfer function $\zeta$ has no poles in the right complex halfplane $\sigma \in \bb{C}, \re{\sigma} > 0$, for all complex numbers $\sigma \in \bb{C}$ it holds that $\conj{\zeta(\sigma)} = \zeta(\conj{\sigma})$, and the the matrix $\zeta(\sigma) + \zeta(\sigma)\herm$ is positive semidefinite for all complex numbers $\sigma \in \bb{C}, \re{\sigma} > 0$.
    Analogously, the transfer function is called \emph{strictly positive real} if in addition to the other two conditions the matrix sum is positive definite.
\end{definition}

To extend the standard balancing procedure to result in positive real systems we relate the passivity of an \ac{LTI} $(A, B, C, D, E)$ with its positive real transfer funciton $\zeta$ as indicated by~\cite[Corollary~2.7]{CGH2022}.
In addition, we can also formulate the condition $D + D\trans \succcurlyeq 0$ on the system's feedthrough term; cf.~\cite[Definition~5]{Gugercin2007}.
Similarly to the Lyapunov equations~\eqref{eq:lyapunov-equations} it can be shown that an \ac{LTI} $(A, B, C, D, \id)$ is strictly positive real if and only if there exist symmetric positive definite matrices $K, L \in \bb{C}^{n \times n}$ such that the following algebraic Riccati equations (AREs) are satisfied
\begin{equation}\label{eq:riccati-equations}
    \begin{aligned}
        A\trans K + K A + (K B - C\trans) (D + D\trans)\inv (K B - C\trans)\trans &= 0 \\
        A L + L A\trans + (L C\trans - B) (D + D\trans)\inv (L C\trans - B)\trans &= 0.
    \end{aligned}
\end{equation}
For positive real systems all solutions $K, L$ can be bounded by minimal and maximal solutions
\begin{equation*}
    K_{\msc{max}} \succcurlyeq K \succcurlyeq K_{\msc{min}} \succcurlyeq 0,\quad L_{\msc{max}} \succcurlyeq L \succcurlyeq L_{\msc{min}} \succcurlyeq 0
\end{equation*}
which has been shown by the connection to bounded real systems in~\cite[Proposition~5.1]{Ober1991}.
The solutions $K, L$ of~\eqref{eq:riccati-equations} are related by $K = L\inv$, thus further implying that
\begin{equation*}
    K_{\msc{min}} = L_{\msc{max}}\inv,\quad K_{\msc{max}} = L_{\msc{min}}\inv.
\end{equation*}
To balance the strictly positive real system we then perform the same calculations as in~\eqref{eq:balancing} on the minimal solutions $K_{\msc{min}}$ and $L_{\msc{min}}$.
Such a system is then called strictly positive real balanced if $K_{\msc{min}} = L_{\msc{min}}$ can be written as the diagonal matrix $\diag{\pi_1 \id[s_1], \dots, \pi_q \id[s_q]}$ with the $\pi_i$ ordered such that $0 < \pi_q < \cdots < \pi_1 \leq 1$ and $s_i$ the corresponding multiplicities summing up to $\sum s_i = n$.

\itodo{change complex numbers to real ones here, we're talking about pr systems and assumed real numbers for \ac{PH} systems! don't forget herm to trans!}

For the rest of this subsection we focus on demonstrating the necessary conditions such that a positive real \ac{LTI} can be represented in \ac{PH} form.
First off result, we require a few additional definitions which relate observability and controllability to the minimality of the \ac{LTI}.

\begin{definition}[{Cf.~\cite[Equation~(8)]{CGH2022} or~\cite[Theorem~6.2]{Freund2004}}]\label{def:behaviourally-controllable}
    An \ac{LTI} $(A, B, C, D, E)$ is said to be \emph{behaviourally controllable} if
    \begin{equation}\label{eq:behaviourally-controllable}
        \forall \lambda  \in \bb{C} \colon \rank{\begin{matrix}
            \lambda E - A & B
        \end{matrix}} = n.
    \end{equation}
\end{definition}

\begin{definition}[{Cf.~\cite[Equation~(8)]{CGH2022} or~\cite[Theorem~6.2]{Freund2004}}]\label{def:behaviourally-observable}
    An \ac{LTI} $(A, B, C, D, E)$ is said to be \emph{behaviourally observable} if
    \begin{equation}\label{eq:behaviourally-observable}
        \forall \lambda  \in \bb{C} \colon \rank{\begin{matrix}
            \lambda {(E - A)}\trans & C\trans
        \end{matrix}} = n.
    \end{equation}
\end{definition}

\begin{lemma}[{Cf.~\cite{Verghese1981, Dai1989}}]\label{lem:minimality-conditions}
    An \ac{LTI} descriptor system $(A, B, C, D, E)$ is minimal if and only if it fulfills Equations~\eqref{eq:behaviourally-controllable} and~\eqref{eq:behaviourally-observable} as well as the following conditions
    \begin{equation*}
        \rank{\begin{matrix}
            E & B
        \end{matrix}} = \rank{\begin{matrix}
            E \\
            C
        \end{matrix}} = n,\quad A \knl{E} \subseteq \range{E}.
    \end{equation*}
\end{lemma}

For the next step we need to compute a specific form of the transfer function's Laurent series.
In the same manner as~\cite[Equation~(37)]{CGH2022} we construct the Laurent expansion
\begin{equation}\label{eq:laurent-series}
    \zeta(s) = \sum\limits_{i = -\infty}^{k - 1} M_i s^i.
\end{equation}

\itodo{make sure you define what the $k$ is in this expansion, this might not be consistent with other sources!}

\begin{lemma}[{Adapted from~\cite[Lemma~5.1]{CGH2022}}]\label{lem:lti-laurent-series}
    Let $(A, B, C, D, E)$ be a positive real \ac{LTI} descriptor system with a transfer function $\zeta$ with its Laurent series as defined in Equation~\eqref{eq:laurent-series}.
    Then we can write the transfer function as a sum of a proper rational function $\zeta_{\msc{p}}$ and the matrix $M_1$
    \begin{equation}\label{eq:pr-lti-laurent-series}
        \zeta(s) = \zeta_{\msc{p}}(s) + M_1 s.
    \end{equation}
\end{lemma}

We now construct a minimal realization $(A_\msc{p}, B_\msc{p}, C_\msc{p}, D_\msc{p}, E_\msc{p})$ with $D_\msc{p} = M_0$ of $\zeta_\msc{p}$ from Lemma~\ref{lem:lti-laurent-series}.
This realization contains an invertible matrix $E_\msc{p}$ because by Lemma~\ref{lem:minimality-conditions} the realization satisfies the assumptions of~\cite[Theorem~6.3]{Freund2004}, and by extension from the positive realness of the complete system $(A, B, C, D, E)$ is also positive real.
We can apply~\cite[Proposition~5.4]{CGH2022} because we have defined the proper realization in such a way that $D_\msc{p} + D_\msc{p}\herm \succcurlyeq M_0 + M_0\herm$ is always true, therefore there exists a matrix $Q_\msc{p} \in \bb{C}^{n \times n}$ that solves the \acp{KYP} for the proper system $(A, B, C, D, E) = (A_\msc{p}, B_\msc{p}, C_\msc{p}, M_0, E_\msc{p})$
\begin{equation}\label{eq:kyp-lmi}
    \begin{pmatrix}
        -A\herm Q - Q\herm A & C\herm - Q\herm B \\
        C - B\herm Q & D + D\herm
    \end{pmatrix} \succcurlyeq 0,\quad E\herm Q = Q\herm E \succcurlyeq 0.
\end{equation}

\begin{lemma}[{Cf.~\cite[Proposition~3.1]{CGH2022}}]\label{lem:kyp-invertible-solution}
    If $\Sigma_\msc{lti}$ in the form~\eqref{eq:lti} has an invertible matrix $E$ and is behaviourally observable, meaning that it satisfies~\eqref{eq:behaviourally-observable}, and there exists a matrix $Q \in \bb{C}^{n \times n}$ solving the LMIs
    \begin{equation*}
        \begin{pmatrix}
            -A\herm Q - Q\herm A & C\herm - Q\herm B \\
            C - B\herm Q & D + D\herm
        \end{pmatrix} \succcurlyeq 0,\quad E\herm Q = Q\herm E,
    \end{equation*}
    then $Q$ is invertible.
\end{lemma}

Thus, by means of Lemma~\ref{lem:kyp-invertible-solution}, the matrix $Q_\msc{p}$ is invertible and by solving~\eqref{eq:kyp-lmi} it also satisfies $Q_\msc{p}\herm E_\msc{p} \succcurlyeq 0$.
This allows us to obtain the proper part of the final \ac{PH} system
\begin{equation}\label{eq:pr-proper-ph-part}
    \Sigma_1 \colon \left\lbrace
    \begin{pmatrix}
        J - R & G - P \\
        {(G + P)}\herm & D
    \end{pmatrix}
    \right. \coloneqq \begin{pmatrix}
        A_\msc{p} Q_\msc{p}\inv & B_\msc{p} \\
        C_\msc{p} Q_\msc{p}\inv & D_\msc{p}
    \end{pmatrix}.
\end{equation}
Additionally, we can form a minimal realization by creating the block matrices
\begin{equation*}
    E_\infty = \begin{pmatrix}
        M_1 & 0 \\
        0 & 0
    \end{pmatrix},\quad A_\infty = \begin{pmatrix}
        0 & -\id[m] \\
        \id[m] & 0
    \end{pmatrix},\quad B_\infty = \begin{pmatrix}
        0 \\
        \id[m]
    \end{pmatrix} = C_\infty\trans,\quad D_\infty = 0.
\end{equation*}
This indeed defines a realization when we compute $s M_1 = C_\infty {(s E_\infty - A_\infty)}\inv B_\infty = C_\infty \begin{pmatrix}
    0 & \id[m] \\
    -\id[m] & s M_1
\end{pmatrix} B_\infty$.
We apply the same arguments as for the proper rational part of the sum~\eqref{eq:pr-lti-laurent-series} to see that these matrices together with $Q_\infty = \id[2m]$ form a \ac{PH} realization
\begin{equation}\label{eq:pr-improper-ph-part}
    \Sigma_2 \colon \left\lbrace
    \begin{pmatrix}
        J - R & G - P \\
        {(G + P)}\herm & D
    \end{pmatrix}
    \right. \coloneqq \begin{pmatrix}
        A_\infty Q_\infty\inv & B_\infty \\
        C_\infty Q_\infty\inv & D_\infty
    \end{pmatrix}.
\end{equation}
Lastly, we combine the two \ac{PH} systems $\Sigma_1$ and $\Sigma_2$ from~\eqref{eq:pr-proper-ph-part} and~\eqref{eq:pr-improper-ph-part} by using~\cite[Lemma~5.6]{CGH2022}.
This surmounts to the final system matrices
\begin{equation*}
    E_\msc{ph} \coloneqq \begin{pmatrix}
        E_\msc{p} & 0 & 0 \\
        0 & M_1 & 0 \\
        0 & 0 & 0
    \end{pmatrix},\quad A_\msc{ph} \coloneqq \begin{pmatrix}
        A_\msc{p} & 0 & 0 \\
        0 & 0 & -\id[m] \\
        0 & \id[m] & 0
    \end{pmatrix},\quad B_\msc{ph} \coloneqq \begin{pmatrix}
        B_\msc{p} \\
        0 \\
        \id[m]
    \end{pmatrix},\quad C_\msc{ph} \coloneqq \begin{pmatrix}
        C_\msc{p} & 0 & \id[m]
    \end{pmatrix},\quad D_\msc{ph} \coloneqq M_0,\quad Q_\msc{ph} \coloneqq \begin{pmatrix}
        Q_\msc{p} & 0 \\
        0 & \id[2m]
    \end{pmatrix}.
\end{equation*}

\subsection{\acl{PHIRKA}}\label{subsec:ph-irka}

In Subsection~\ref{subsec:interpolation-reduction} we introduced the \ac{IRKA}.
In~\cite{Gugercin2012}, the authors propose an extended variant of the \ac{IRKA} algorithm that produces \ac{PH} reduced realizations of the transfer function $\zeta$.
When considering Algorithm~\ref{alg:irka} it is not inherently clear that the constructed subspaces $V_r$ and $W_r$ induce a passive system, however choosing $W_r = Q V_r {(V_r\trans Q V_r)}\inv$ generates a subspace which satisfies this exact constraint, thus realizing a \ac{PH} system; cf.~\cite[Section~2.4]{Breiten2022}.
The so-called \ac{PHIRKA} algorithm is shown in Algorithm~\ref{alg:ph-irka}.

\begin{algorithm}\label{alg:ph-irka}
    \caption{\ac{PHIRKA}; cf.~\cite[Algorithm~1]{Gugercin2012}}
    \KwData{Initial interpolation points $\sigma$ closed and initial tangent directions $r_1^{(0)}, \dots, r_r^{(0)}$ under complex conjugation, full order model $\Sigma$}
    $V_r^{(0)} \coloneqq \left( {(\sigma_1^{(0)} \id - (J - R) Q)}\inv B r_1^{(0)}, \dots, {(\sigma_r^{(0)} \id - (J - R) Q)}\inv B r_r^{(0)} \right)$\;
    $W_r^{(0)} \coloneqq Q V_r^{(0)} \left( {V_r^{(0)}}\trans Q V_r^{(0)} \right)\inv$\;
    \While{Not converged}{
        $J_r^{(k + 1)} \coloneqq {W_r^{(k)}}\trans J W_r^{(k)}, R_r^{(k + 1)} \coloneqq {W_r^{(k)}}\trans R W_r^{(k)}, Q_r^{(k + 1)} \coloneqq {V_r^{(k)}}\trans Q V_r^{(k)}, B_r^{(k + 1)} \coloneqq {W_r^{(k)}}\trans B$\;
        $A_r^{(k + 1)} \coloneqq (J_r^{(k + 1)} - R_r^{(k + 1)}) Q_r^{(k + 1)}$\;
        Compute left and right eigenpairs $(\lambda_i, x_i)$ and $(\lambda_i, y_i)$, such that $A_r^{(k + 1)} x_i = \lambda_i x_i, y_i\herm A_r^{(k + 1)} = \lambda_i y_i\herm, y_i\herm x_i = \delta_{i, j}$\;
        $\sigma_i^{(k + 1)} \coloneqq -\lambda_i, {r_i^{(k + 1)}}\trans \coloneqq y_i\herm B_r^{(k + 1)}$\;
        $V_r^{(k + 1)} \coloneqq \left( {(\sigma_i^{(k + 1)} \id - (J - R) Q)}\inv B r_1^{(k + 1)}, \dots, {(\sigma_r^{(k + 1)} \id - (J - R) Q)}\inv B r_r^{(k + 1)} \right)$\;
    }
\end{algorithm}

It is shown in~\cite[Theorem~11]{Gugercin2012} that any such reduced system is \ac{PH} and satisfies some of the necessary $\mcl{H}_2$ optimality conditions.
The remaining conditions can be satisfied if we compute
\begin{equation*}
    \zeta_r(s) \coloneqq \sum\limits_{i = 1}^r \frac{c_i b_i\trans}{s - \lambda_i}
\end{equation*}
and verify that the following equality is satisfied
\begin{equation*}
    \range{{(\lambda_1 \id - (J - R) Q)}\inv B b_1, \dots, {(\lambda_r \id - (J - R) Q)}\inv B b_r} = \range{{(\lambda_1 \id - (J - R) Q)}\inv B b_1, \dots, {(\lambda \id - (J - R) Q)}\inv B c_r}
\end{equation*}

\itodo{figure out how to include an E term in phirka}

\begin{remark}
    Additional \ac{PH} reduction techniques for example are contained in~\cite{Polyuga2010, Chaturantabut2016, Cherifi2019, BGD2020}.
\end{remark}
