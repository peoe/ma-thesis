\section{MOR with \aclp{NN}}\label{sec:nn-mor}

In recent years, \acp{NN} have been introduced into a variety of numerical applications to allow for nonlinear solutions to many problems, among them \ac{MOR} methods.
This section aims to explain how structures from deep \acp{NN} can be used in reducing \acp{FOM}.
Thus, we inherently rely on some basic knowledge in the structure of \acp{NN} and \acp{DCAE}.
For references on \acp{NN} and Deep Learning, we refer to sources such as~\cite{Goodfellow2016, Kubat2017, Sarker2021} as well as the other citations therein.

To approximate a function $f \colon \bb{R}^n \to \bb{R}$ or more generally an operator $A \colon \bb{R}^n \to \bb{R}^m$ with \acp{NN}, the general procedure consists of first constructing a parametrized network $A_\theta$ intended to approximate the operator $A$, training the network by successive updates to the parameter $\theta$ obtained by evaluating an approximate gradient of a loss function which ressembles a measured error between the output of the original operator $A$ and its computed counterpart as the output of $A_\theta$.
By the choice of nonlinear activation functions in the network's hidden layers, this method can produce very accurate nonlinear models while being extremely parallelizable.
However, it requires both large amounts of data as well as appropriate design choices in the structure of the \ac{NN} to be able to generalize from restrictive given input data to more complex verification data.

Besides the ability to mimic operator behaviour, \ac{NN} architectures can recreate effects akin to \ac{MOR}.
To this end, modern deep \ac{NN} techniques are employed to inherently produce \ac{NN} that in turn ressemble a low order network.
In recent publications such as~\cite{Lee2020, Salvador2021, Benner2022, Kim2022, Buchfink2023}, \acp{DCAE} are introduced.
These \acp{DCAE} consist of two parametrized components: an encoder $e_\theta \colon \bb{R}^n \to \bb{R}^r$ and a decoder $d_\theta \colon \bb{R}^r \to \bb{R}^n$.
Here, the encoder serves as a means to reduce the state order by computing $x_\msc{r} = e_\theta(x) \in \bb{R}^r$, and the decoder acts as the reconstruction of the reduced state back to the \ac{FOM} state $\hat{x} = d_\theta(x_\msc{r})$.
To train the encoder-decoder pair, we typically choose a loss function that corresponds to the projection error over a sample set $\mcl{S}$ of full order states
\begin{equation}\label{eq:reduction-loss}
    \mcl{L}_{\msc{data}}(\theta) \coloneqq \frac{1}{\abs{\mcl{S}}{}} \sum\limits_{x \in \mcl{S}} \norm{x - d_\theta\left( e_\theta(x) \right)}{}^2.
\end{equation}
In addition to this loss function, we can also express systemic constraints on the components of the \acp{ROM}.
As an example, the approach highlighted in~\cite{Buchfink2023} introduces a second loss function $\mcl{L}_{\msc{sympl}}(\theta)$ that weakly enforces the final \ac{ROM}'s symplecticity, a property closely related to the physically motivated nature of \ac{PH} systems.
Then, the combined loss function $\mcl{L} \coloneqq \alpha \mcl{L}_{\msc{data}} + (1 - \alpha) \mcl{L}_{\msc{sympl}}$ not only generates a reduced order model, but also weakly forces this reduced order model to have symplectic structure.

When we consider \ac{NN}-based approximation, we get a clear distinction between the offline and the online phases: We have to perform expensive calculations while training the model offline, and after the reduction we can efficiently evaluate the structure of the network to quickly get results in the online evaluation phase.
Unfortunately, the training process conceptually obfuscates how the low order state representation $e_\theta(x)$ is connected to the full order state $x$.
This results in sophisticated nonlinear models that are entirely unexplainable.
We aim to circumnavigate the unexplainability introduced in the \ac{NN} setting which means that we have to consider methods acting directly on the dynamical systems.
