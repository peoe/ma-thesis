\section{System Theoretic Linear MOR}\label{sec:system-mor}

\acl{MOR} for Systems Theory offers two main perspectives: reduction resulting from the system matrices of \ac{LTI} systems as in Definition~\ref{def:lti}, and reduction by means of the associated transfer function from Definition~\ref{def:transfer-function}.
We give an idea of truncation in Subsection~\ref{subsec:balanced-truncation} by explaining the method of \acf{BT}, a reduction method focussed on the system's matrices.
Afterwards, we discuss reduction by means of interpolating the transfer function, which leads us to the \acf{IRKA} in Subsection~\ref{subsec:interpolation-reduction}.

\subsection{Balanced Truncation}\label{subsec:balanced-truncation}

\ac{BT} is a two step procedure consisting of balancing and truncation.
Firstly, balancing an \ac{LTI} system in a heuristic way means transforming the system such that hard-to-reach states are simultaneously hard to observe.
Secondly, truncation of the balanced system constitutes itself by omission of the matrix blocks that correspond to hard-to-reach and hard-to-observe system states.
In this manner, we create a system that prioritizes easily controllable and easily observable states.

This procedure was originally described in~\cite{Mullis1976} and later extended by~\cite{Moore1981, Enns1984}.
Within the \ac{BT} methodology, several balancing options are available to cater to different system requirements.
Standard Lyapunov balancing often requires large computational times as~\cite{Gugercin2007} points out, because during this procedure one has to compute expensive dense matrix factorizations.
To mitigate this drawback, approximate balanced truncation methods have been developed.
These include, among others, stochastic balancing~\cite{Desai1984, Green1988}, frequency weighted balancing~\cite{Enns1984, Wang1999}, bounded real balancing~\cite{Opdenacker1988, Reis2010}, or positive real balancing~\cite{Desai1984, Ober1991}.
For our purposes, it suffices to consider Lyapunov \ac{BT} and \ac{PRBT} since we are not primarily interested in the theory of balancing \ac{MOR} methods.
\ac{PRBT} in particular plays an important role in the context of this thesis because it produces passive systems that are closely linked to \ac{PH} systems but we cover this connection later on in Subsection~\ref{subsec:prbt}.

We commence with standard Lyapunov \ac{BT} as described in~\cite{BB2017}.
Consider $(A, B, C, D, E)$ to be a realization of an \ac{LTI} system as in Definition~\ref{def:lti}.
It is important to note that some \ac{BT} literature employs \ac{LTI} systems that have the trivial matrix $E = \id$.
If $E$ is regular, then any system $(A, B, C, D, E)$ can easily be transformed into a system of the form $(\tilde{A}, \tilde{B}, \tilde{C}, \tilde{D}, \id)$ by applying $E\inv$ to the state equation in~\eqref{eq:lti}.
Henceforth, we refer to our \ac{LTI} system $\Sigma_\msc{lti}$ by the matrices $(A, B, C, D, \id)$.

\begin{definition}[{Adapted from~\cite[Definition~6.2]{BB2017}}]\label{def:controllability}
    The system $(A, B, C, D, \id)$ is controllable if for all states $x_0, x_1 \in \bb{R}^n$ and a final time $T \geq 0$ there exists an admissible control $u \colon \interval{0}{T} \to \bb{R}^q$ such that the system's solution $x(t)$ satisfies $x(0) = x_0$ and $x(T) = x_1$.
\end{definition}

\begin{definition}[{Adapted from~\cite[Definition~6.3]{BB2017}}]\label{def:observability}
    The state $x_1 \in \bb{R}^n$ of a system $(A, B, C, D, \id)$ is unobservable if the output of the system with the initial condition $x(0) = x_1$ is constant $y(t) \equiv 0$ for all points in time.
    In particular, a state $x_1$ being unobservable means that $x_1$ cannot be distinguished from the zero state no matter the control applied to the system.
    A system is observable if the set of all unobservable states consists of only the trivial zero state.
\end{definition}

To characterize which system states are observable and which system states are controllable, we define the system's infinite Controllability Gramian as given in~\cite[Equation~6.8]{BB2017}
\begin{equation}\label{eq:controllability-gramian}
    P = \int\limits_0^\infty \exp{(A t)} B B\trans \exp \left( A\trans t \right) \dif t \in \bb{R}^{n \times n}
\end{equation}
alongside the corresponding infinite Observability Gramian
\begin{equation}\label{eq:observability-gramian}
    Q = \int\limits_0^\infty \exp \left( A\trans t \right) C\trans C \exp{(A t)} \dif t \in \bb{R}^{n \times n}.
\end{equation}
For asymptotically stable and minimal systems as defined in Definitions~\ref{def:minimal-system} and~\ref{def:lti-stability}, these matrices satisfy the Lyapunov equalities
\begin{equation}\label{eq:lyapunov-equations}
    \begin{aligned}
        A P + P A\trans + B B\trans &= 0, \\
        A\trans Q + Q A + C\trans C &= 0
    \end{aligned}
\end{equation}
as has been shown in~\cite{Antoulas2005, Hinrichsen2005}.

The act of balancing now reduces to finding a transformation matrix $T \in \bb{R}^{n \times n}$ such that the two matrices $T P T\trans$ and $T\inv[T] Q T\inv$ are diagonal matrices.
One such balancing transformation can be constructed by computing the Cholesky decompositions
\begin{equation}\label{eq:balancing}
    P = S\trans S,\quad Q = R\trans R,\quad S R\trans = U \Sigma V\trans
\end{equation}
and then defining the transformation matrix as $T \coloneqq \Sigma\inv[\frac{1}{2}] V\trans R$.
With this $T$ we can transform the original \ac{LTI} system $(A, B, C, D, \id)$ into its balanced form $(T A T\inv, T B, C T\inv, D, \id)$.
To finally truncate the system, that is, to obtain a reduced order system, we split the balanced system into matrix blocks
\begin{equation*}
    T A T\inv = \begin{pmatrix}
        A_1 & A_2 \\
        A_3 & A_4
    \end{pmatrix},\quad T B = \begin{pmatrix}
        B_1 \\
        B_2
    \end{pmatrix},\quad C T\inv = \begin{pmatrix}
        C_1 & C_2
    \end{pmatrix}.
\end{equation*}
Thereafter, we define the balanced \ac{ROM} as $(A_1, B_1, C_1, D, \id)$.
This reduction even yields the following estimate of the $\mcl{H}_\infty$ projection error
\begin{equation*}
	\norm{\tfunc - \hat{\tfunc}}{\mcl{H}_\infty} \coloneqq \sup\limits_{\omega \in \bb{R}} \sigma_{\max} \left( \tfunc(\iu \omega) - \hat{\tfunc}(\iu \omega) \right).
\end{equation*}

\begin{theorem}[{Adapted from~\cite[Theorem~6.4]{BB2017}}]\label{thm:h-inf-error}
    Let $(A_\msc{b}, B_\msc{b}, C_\msc{b}, D_\msc{b}, \id)$ be a balanced realization of an asymptotically stable and minimal system as defined in Definitions~\ref{def:minimal-system} and~\ref{def:lti-stability} with an ordered sequence of real numbers $\sigma_1 > \cdots > \sigma_k > 0$ such that the Gramians read $P = Q = \diag{\sigma_1 \id, \dots, \sigma_k \id}$.
    Then the reduced system $\hat{\Sigma}_\msc{lti}$ obtained by truncating $\Sigma_\msc{lti}$ is asymptotically stable, minimal, balanced, and fulfills
    \begin{equation*}
        \norm{\tfunc - \hat{\tfunc}}{\mcl{H}_\infty} \leq 2 \sum\limits_{i = r + 1}^k \sigma_i
    \end{equation*}
    for the transfer functions $\tfunc$ and $\hat{\tfunc}$ corresponding to $\Sigma_\msc{lti}$ and $\hat{\Sigma}_\msc{lti}$.
\end{theorem}

\subsection{Interpolation-Based Reduction}\label{subsec:interpolation-reduction}

Whereas balanced \acp{ROM} try to minimize the $\mcl{H}_\infty$-norm, interpolation-based reduced models aim to find bestapproximations with respect to the $\mcl{H}_2$-norm as defined in~\cite[Section~3]{Gugercin2008} by
\begin{equation*}
	\norm{\tfunc}{\mcl{H}_2}^2 \coloneqq \frac{1}{2 \pi} \int\limits_{- \infty}^\infty \norm{\tfunc(\iu \omega)}{}^2 \dif \omega.
\end{equation*}
The problem with this type of optimization problem, as indicated in~\cite[Section~3.1]{Gugercin2008}, is that the set of stable \acp{LTI} does not form a subspace of $\mcl{H}_2$, hence the $\mcl{H}_2$ optimal approximation is not easy to characterize.
As a replacement for this optimization set, we instead consider the set of all proper rational transfer functions as defined in~\cite[Section~1.3.1]{Corless2003} with the simple poles $\mu \coloneqq {\sset{\mu_i}}_{i = 1}^r \subseteq \bb{C}^r$ in the open left half plane $\bb{C}_{-}$
\begin{equation*}
    \mcl{M}(\mu) \coloneqq \iset{\tfunc \text{ proper rational transfer function}}{\tfunc \text{ has simple poles at } \mu_i, i = 1, \dots, r}.
\end{equation*}
Following~\cite[Theorem~3.1]{Gugercin2008}, this results in an subset of $\mcl{H}_2$ with the properties that
\begin{enumerate}
    \item $H \in \mcl{M}(\mu)$ is the transfer function of a stable \ac{LTI} system with $\dim{(H)} = r$,
    \item $\mcl{M}(\mu)$ is an $(r - 1)$-dimensional subspace of $\mcl{H}_2$,
    \item $\tfunc_r \in \mcl{M}(\mu)$ is the unique bestapproximation of $\tfunc \in \mcl{H}_2$ if and only if it holds that $\inner{\tfunc - \tfunc_r}{H}{\mcl{H}_2} = 0$ for all other transfer functions $H \in \mcl{M}(\mu)$.
\end{enumerate}
As a consequence of~\cite[Theorem~3.1]{Gugercin2008}, if we interpolate the \ac{FOM} transfer function $\tfunc$ at the mirror images $\hat{\sigma} = -a + \iu b$ of its poles $\sigma = a + \iu b$, the subspace transfer function $\tfunc_\msc{r} \in \mcl{M}(\mu)$ is $\mcl{H}_2$-optimal among all \acp{ROM} with the same reduced poles $\mu$.

For any local minimizer we can formulate the following necessary optimality conditions of the $\mcl{H}_2$ bestapproximation problem with the interpolation points $\sigma_i \in \bb{C}$ and the corresponding tangential directions $b_i, c_i \in \bb{C}^{m}$ for an \ac{LTI} system with $m \in \bb{N}$ inputs and outputs
\begin{equation}\label{eq:h2-optimality-conditions}
    \tfunc_\msc{r}(- \sigma_i) b_i = \tfunc(- \sigma_i) b_i,\quad c_i\trans \tfunc_\msc{r}(- \sigma_i) = c_i\trans \tfunc(- \sigma_i),\quad c_i\trans \tfunc'_\msc{r}(- \sigma_i) b_i = c_i\trans \tfunc(- \sigma_i) b_i.
\end{equation}

While these conditions are very useful, solving the optimal approximation problem remains complicated.
To combat this difficulty, the authors of~\cite{Gugercin2008} propose an algorithm derived from the standard Newton update formula.
Let $\sigma \coloneqq {\sset{\sigma_i}}_{i = 1}^r \subset \bb{C}$ denote the set of interpolation points of the reduced model $\hat{\Sigma}_\msc{lti}$.
From this model we compute the transfer function $\tfunc_r$ and its poles $\lambda(\sigma) = {\sset{\lambda_i}}_{i = 1}^r$.
We now want to minimize the difference between the mirror images of the interpolation points and the poles $\lambda(\sigma)$.
To this end, we define the objective function $g(\sigma) \coloneqq \lambda(\sigma) + \sigma$, aiming to ultimately obtain an optimal set of interpolation points $\sigma_i$ such that $g\left( {\sset{\sigma_i}}_{i = 1}^r \right) = 0$.
This optimality condition directly corresponds to $\lambda(\sigma) = -\sigma$, thus resulting in the interpolation points matching the mirror images of the poles if we consider complex conjugate pairs of interpolation points.
The Newton step for this optimization in its full form is
\begin{equation}\label{eq:interpolation-points-newton-step}
    \sigma^{(k + 1)} = \sigma^{(k)} - (\id + J)\inv (\sigma^{(k)} + \lambda(\sigma^{(k)})),
\end{equation}
where $J$ is the Jacobian matrix in the sense that $J_{i, j} = \pd[\sigma_j]{} \lambda_i$.
In practice, the explicit compution of this Jacobian quickly becomes prohibitively expensive.
Therefore, we make the suitable substitution $J \approx 0$ because the Jacobian will be close to zero when the set of interpolation points is close to an optimal set.
Applying this assumption in~\eqref{eq:interpolation-points-newton-step}, the Newton update reduces to
\begin{equation*}
    \sigma^{(k + 1)} = - \lambda(\sigma^{(k)}).
\end{equation*}
Algorithm~\ref{alg:irka} shows how this Newton step is used in the \acf{IRKA} as described in~\cite[Algorithm~4.1]{Gugercin2008}.
This algorithm heavily relies on a realization-based formulation; however, there are some variants independent of specific realizations using the Loewner framework such as~\cite[Algorithm~7.2]{Beattie2017}.
It is further notable, that, due to the complex conjugates in the interpolation points, the updated subspaces $V_\msc{r}$ and $W_\msc{r}$ can be chosen as real-valued as explained in~\cite[Remark~4]{Gugercin2012}.

\begin{algorithm}\label{alg:irka}
    \caption{\ac{IRKA}, adapted from~\cite[Algorithm~7.1]{Beattie2017}}
    \KwData{Initial interpolation points ${\sset{\sigma_i^{(0)}}}_{i = 1}^r$ and initial tangent directions ${\sset{r_i^{(0)}}}_{i = 1}^r, {\sset{\ell_i^{(0)}}}_{i = 1}^r$ closed under complex conjugation, full order model $\Sigma$}
    \tcc{Initialize reduced subspaces}
    $V_r^{(0)} \coloneqq {\left( {(\sigma_i^{(0)} E - A)}\inv B r_i^{(0)} \right)}_{i = 1}^r,\quad W_r^{(0)} \coloneqq {\left( {(\sigma_i^{(0)} E - A\trans)}\inv C\trans \ell_i^{(0)} \right)}_{i = 1}^r$\;
    $k \coloneqq 0$\;
    \While{not converged}{
        $k \coloneqq k + 1,\quad \tilde{A} \coloneqq W\trans A V,\quad \tilde{E} \coloneqq W\trans E V,\quad \tilde{B} \coloneqq W\trans B,\quad \tilde{C} = C V$\;
        \tcc{Compute pole-residue expansion}
        $\tfunc_r(s) = \tilde{C} {(s \tilde{E} - \tilde{A})}\inv \tilde{B} = \sum\limits_{i = 1}^r \frac{\tilde{\ell}_i \tilde{r}_i\trans}{s - {\lambda(\sigma^{(k)})}_i}$\;
        \tcc{Update interpolation points and tangential directions}
        $\sigma_i^{(k)} \coloneqq - \lambda_i(\sigma^{(k - 1)}),\quad r_i^{(k)} \coloneqq \tilde{r}_i,\quad \ell_i^{(k)} \coloneqq \tilde{\ell}_i$\;
        \tcc{Update reduced subspaces}
        $V_\msc{r}^{(k)} \coloneqq {\left( {(\sigma_i^{(k)} E - A)}\inv B r_i^{(k)}\right)}_{i = 1}^r,\quad W_\msc{r}^{(k)} \coloneqq {\left( {(\sigma_i^{(k)} E - A\trans)}\inv C\trans \ell_i^{(k)} \right)}_{i = 1}^r$\;
    }
\end{algorithm}
