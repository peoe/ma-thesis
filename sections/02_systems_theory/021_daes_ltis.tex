\section{\aclp{LTI}}\label{sec:ltis}

As detailed in~\cite{Kunkel2006}, a \acf{DAE} can be expressed as
\begin{equation}\label{eq:general-dae}
    F(t, x, \dot{x}) = 0,
\end{equation}
where for natural numbers $n, m$ the function $F \colon I \times \Omega_x \times \Omega_{\dot{x}} \to \bb{C}^m$ operates on an interval $I \sse \bb{R}_{\ge 0}$ and two open spatial domains $\Omega_x, \Omega_{\dot{X}} \sse \bb{C}^n$.
The meaning of $\dot{x}$ in~\cite{Kunkel2006} is purposefully ambiguous, allowing $\dot{x}$ to refer to either the derivative of a function with respect to time $x$ or as an independent variable in the formulation.
For the purposes of this thesis, however, we only consider $\dot{x}$ to be the time derivative $\frac{\dif}{\dif t} x$.

In the case of linear \acp{DAE} with constant coefficients, we can express the implicit formulation of Equation~\eqref{eq:general-dae} as the explicit formula
\begin{equation}\label{eq:lin-const-coeff-dae}
    E \dot{x}(t) = A x(t) + f(t)
\end{equation}
in terms of matrices $A, E \in \bb{C}^{n \times n}$ and a time dependent function $f \colon I \to \bb{C}^n$.
We further assume that the function $f$ serves as an influence on the solution of the \ac{DAE}.
In recollection of the \ac{PH} framework we want to define, this influence reflects the external control variable acting on the system over time.
We explicitly formulate the control in terms of yet another linear matrix product
\begin{equation}\label{eq:control-substitution}
    f(t) = B u(t)
\end{equation}
for some natural number $q$, a matrix $B \in \bb{C}^{n \times q}$, and a control variable $u \colon I \to \bb{C}^q$.
Lastly, we introduce an output component $y$ to our modelled system.
This output too takes the shape of a time dependent function $y \colon I \to \bb{C}^q$ and, keeping in line with Equations~\eqref{eq:lin-const-coeff-dae} and~\eqref{eq:control-substitution}, we write it as the linear combination of the state $x$ and the control $u$ with the matrices $C \in \bb{C}^{q \times n}, D \in \bb{C}^{q \times q}$
\begin{equation}\label{eq:output-definition}
    y(t) = C x(t) + D u(t).
\end{equation}

\begin{definition}\label{def:lti}
    A \acl{LTI} $\Sigma_{\msc{lti}}$ is a system of equations
    \begin{equation}\label{eq:lti}
        \Sigma_{\msc{lti}} \colon \left\lbrace
        \begin{aligned}
            E \dot{x}(t) &= A x(t) + B u(t), \\
            y(t) &= C x(t) + D u(t)
        \end{aligned}
        \right.
    \end{equation}
    with the matrices $A, E \in \bb{C}^{n \times n}, B \in \bb{C}^{n \times q}, C \in \bb{C}^{q \times n}, D \in \bb{C}^{q \times q}$ constant in time as well as state functions $x, \dot{x} \colon I \to \bb{C}^n$ and interaction variables $u, y \colon I \to \bb{C}^q$ which vary over time.
\end{definition}

\begin{remark}
    Definition~\ref{def:lti} constitutes a continuous dynamical system.
    If we instead sample the trajectory of $x$ at fixed sample points $0 \leq t_1 \dots \leq t_k = T < \infty$, then we can rephrase $\Sigma_\msc{lti}$ in discrete form
    \begin{equation*}
        \Sigma_\msc{lti}^\msc{d} \colon \left\lbrace
        \begin{aligned}
            E^\msc{d} x_{i + 1} &= A^\msc{d} x_i + B^\msc{d} u_i, \\
            y_i &= C^\msc{d} x_i + D^\msc{d} u_i,
        \end{aligned}
        \right.
    \end{equation*}
    where $x_i \coloneqq x(t_i)$ and $u_i \coloneqq u(t_i), y_i \coloneqq y(t_i)$ are the state, control and output variables sampled at the time points $t_i, i = 1, \dots, k$, and $E^\msc{D}, A^\msc{D}, B^\msc{D}, C^\msc{D}$ and $D^\msc{D}$ are the discrete time equivalents of $E, A, B, C$ and $D$.
    These systems mostly come up when either the derivative data $\dot{x}$ is not available to us or when we discretize a continuous system using a time stepping procedure.
\end{remark}

\begin{remark}
    In contrast to systems of \acp{ODE}, \acp{DAE} can include equations that are not differential in nature.
    In this sense, the system
    \begin{align}
        \dot{x} &= A x + B u, \label{eq:dae-ode-difference-state}\\
        \dot{y} &= C x + D u\label{eq:dae-ode-difference-ode-io}
    \end{align}
    classifies as both a system of \acp{ODE} as well as a \ac{DAE}.
    In contrast, if Equation~\eqref{eq:dae-ode-difference-ode-io} read
    \begin{equation}\label{eq:dae-ode-difference-dae-io}
        y = C x + D u
    \end{equation}
    instead, then the system consisting of Equations~\eqref{eq:dae-ode-difference-state} and~\eqref{eq:dae-ode-difference-dae-io} no longer is a system of \acp{ODE} but only a \ac{DAE}.
\end{remark}

Any \ac{LTI} system in the form~\eqref{eq:lti} is said to act in the time domain, meaning that the inputs of both the states $x, \dot{x}$, the control $u$, and output variable $y$ are functions of time.
From a Systems Theory point of view, however, it is necessary to consider the system also from within the frequency domain.
There, we no longer consider how the system's state along with the control and the output change over time, but rather how their behaviour over time can be represented by oscillations depending on frequency.
We compute this frequency formulation by applying the Laplace transform
\begin{equation*}
    \mcl{L}[f](\omega) = \int\limits_0^\infty \exp{(- \omega t)} f(t) \dif t,\quad \omega \in \bb{C},
\end{equation*}
to the states, the control, and the output in the \ac{LTI} system~\eqref{eq:lti}; cf.~\cite{Arendt2011}.
We then define the following functions as variables of the complex frequency $\omega$ such that
\begin{equation}
    X(\omega) \coloneqq \mcl{L}[x](\omega),\quad \dot{X}(\omega) \coloneqq \mcl{L}[\dot{x}](\omega),\quad U(\omega) \coloneqq \mcl{L}[u](\omega),\quad Y(\omega) \coloneqq \mcl{L}[y](\omega).
\end{equation}
Next, we use the crucial property that the Laplace transform of a function's derivative can be expressed in terms of the function itself with the equation
\begin{equation*}
    \mcl{L}[\dot{x}](\omega) = \omega \mcl{L}[x](\omega) - \lim\limits_{t \searrow 0} x(t);
\end{equation*}
cf.~\cite[Theorem~9.1]{Doetsch1974}.
Therefore, under the assumption that the matrix pencil $\omega E - A$ is regular, we compute from the \ac{LTI} system's first equation
\begin{alignat*}{3}
     & & E \dot{X}(\omega) &= A X(\omega) + B U(\omega) \\
    \implies & & E (\omega X(\omega) - x(0)) &= A X(\omega) + B U(\omega) \\
    \implies & & \omega E X(\omega) - A X(\omega) &= E x(0) + B U(\omega) \\
    \implies & & (\omega E - A) X(\omega) &= E x(0) + B U(\omega) \\
    \implies & & X(\omega) &= (\omega E - A)\inv E x(0) + (\omega E - A)\inv B U(\omega).
\end{alignat*}
We apply this result to the second equation of the \ac{LTI} system~\eqref{eq:lti}, thus describing the output in terms of the frequency-based control $Y$ by the equation
\begin{equation}\label{eq:frequency-output}
    \begin{aligned}
        Y(\omega) &= C X(\omega) + D U(\omega) \\
        &= C (\omega E - A)\inv B U(\omega) + D U(\omega) + C (\omega E - A)\inv E x(0).
    \end{aligned}
\end{equation}
Notably, the last equality of~\eqref{eq:frequency-output} contains a summand that only depends on the initial state $x_0 = x(0)$ of the system.
For ease of notation, we assume that this time domain state value $x(t)$ initially is zero: $x(0) = x_0 = 0$.
This restriction is a sensible assumption because it corresponds to a system that initially contains no energy and can only be interacted with through the control.
Equation~\eqref{eq:frequency-output} thus shows how we can directly compute the output of a system under the influence of a control without having to consider the internal state representation.
More general definitions can also incorporate the initial state dependency in the transfer function; however, this is not necessary in the context of this thesis, thus we refer the interested reader to the introductory chapter in~\cite{Benner2017} for a detailed explanation.
From now on, we access the frequency domain control-output relation by defining the intermediate $\tfunc$ such that $Y(\omega) = \tfunc(\omega) U(\omega)$.

\begin{definition}\label{def:transfer-function}
    The transfer function $\tfunc \colon \bb{C} \to \bb{C}^{q \times q}$ of an \ac{LTI} system~\eqref{eq:lti} is defined by
    \begin{equation}\label{eq:transfer-function}
        \tfunc(\omega) = C (\omega E - A)\inv B + D
    \end{equation}
    for the system matrices $A, E \in \bb{C}^{n \times n}, B \in \bb{C}^{n \times q}, C \in \bb{C}^{q \times n}$ and $D \in \bb{C}^{q \times q}$, and frequency domain control function $U \colon \bb{C} \to \bb{C}^{q}$.
\end{definition}

\begin{remark}
    The matrices $(A, B, C, D, E)$ used in Definition~\ref{def:transfer-function} are not unique.
    For any transfer function $\tfunc$, multiple tuples of system matrices can exist that produce the same output $\tfunc(\omega)$ for identical frequencies $\omega \in \bb{C}$.
    Any such tuple is called a realization of the transfer function.
    We can create an arbitrary amount of realizations by considering the coordinate transformation matrices $T \in \bb{C}^{n \times n}$.
    If $T$ is regular, then we can obtain a different realization of $\tfunc$ by substituting $\tilde{E} \coloneqq T E T\inv, \tilde{A} \coloneqq T A T\inv, \tilde{B} \coloneqq T B, \tilde{C} \coloneqq C T\inv, \tilde{D} \coloneqq D$.
    This is evident if we compute
    \begin{align*}
        &\tilde{C} {\left( \omega \tilde{E} - \tilde{A} \right)}\inv \tilde{B} + \tilde{D} \\
        = &C T\inv {\left( \omega T E T\inv - T A T\inv \right)}\inv T B + D \\
        = &C T\inv {\left( T (\omega E - A) T\inv \right)}\inv T B + D \\
        = &C T\inv T {\left( \omega E - A \right)}\inv T\inv T B + D \\
        = &C {(\omega E - A)}\inv B + D \\
        = &\tfunc(\omega).
    \end{align*}
\end{remark}

Next, we consider what it means that a dynamical system is minimal.
This directly leads us to structure-preserving \ac{MOR} algorithms.
In order to discuss these algorithms, we yet require a notion of a transfer function's poles and their relation to the stability of the underlying dynamical system.

\begin{definition}[Cf.~\cite{CGH2022}]\label{def:minimal-system}
    The realization $(A, B, C, D, E)$ of an \ac{LTI} system with its associated transfer function $\tfunc$ is said to be minimal if the state dimension $n \in \bb{N}$ in the realization is minimal.
\end{definition}

\begin{definition}[{Cf.~\cite[Section~2]{Benner2017}}]\label{def:transfer-function-poles}
    Given a transfer function $\tfunc(\omega)$, we decompose it as the rational function $\tfunc(\omega) = \frac{N(\omega)}{d(\omega)}$, where $N \colon \bb{C} \to \bb{C}^{q \times n}$ is a matrix-valued polynomial and $d \colon \bb{C} \to \bb{C}$ is a scalar polynomial representing the least common denominator of $\tfunc(\omega)$'s $q \cdot m$ entries.
    We then define the transfer function's poles as the set of complex numbers
    \begin{equation*}
        \mcl{P}(\tfunc) \coloneqq \iset{\sigma \in \bb{C}}{d(\sigma) = 0}.
    \end{equation*}
\end{definition}

\begin{remark}
    In general, the set of a transfer function's poles $\mcl{P}(\tfunc)$ is a subset of the set of the generalized eigenvectors of the matrix pencil $\omega E - A$ defined by
    \begin{align*}
        {(\omega E - A)}^m x_m = 0,\quad {(\omega E - A)}^{m - 1} x_m \neq 0.
    \end{align*}
    However, if the realization $(A, B, C, D, E)$ is minimal as defined in Definition~\ref{def:minimal-system}, then both sets are equal, cf.~\cite[Section~2]{Benner2017}.
\end{remark}

\begin{definition}[{Cf.~\cite[Section~2]{Benner2017}}]\label{def:lti-stability}
    Let $(A, B, C, D, E)$ be a realization of the \ac{LTI} system $\Sigma_\msc{lti}$.
    Then, $\Sigma_\msc{lti}$ is stable if all finite eigenvalues of the matrix pencil $\omega E - A$ lie within the open left complex half-plane $\bb{C}_{-} \coloneqq \iset{\sigma \in \bb{C}}{\fk{Re}(\sigma) < 0}$.
    Any stable system under the control $u(t) \equiv 0$ tends to the zero state for $t \to \infty$ and is henceforth asymptotically stable.
\end{definition}
