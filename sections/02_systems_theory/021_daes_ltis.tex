\section{Linear Time-Invariant Systems}\label{sec:ltis}

As detailed in~\cite{Kunkel2006}, a \emph{differential-algebraic equation} can be expressed as
\begin{equation}\label{eq:general-dae}
    F(t, x, \dot{x}) = 0,
\end{equation}
where for natural numbers $n, m$ the function $F \colon I \times \Omega_x \times \Omega_{\dot{x}} \to \bb{C}^m$ operates on an interval $I \sse \bb{R}$, and two open spatial domains $\Omega_x, \Omega_{\dot{X}} \sse \bb{C}^n$.
The source we state purposefully keeps the meaning of $\dot{x}$ ambiguous, referring to either the derivative with respect to time of a function $x$ or as an independent variable in the formulation, however we choose to only consider the time derivative $\dot{x} = \frac{\dif}{\dif t} x$ here.

In the case of linear DAEs with constant coefficients, we can express the implicit formulation of Equation~\eqref{eq:general-dae}
\begin{equation}\label{eq:lin-const-coeff-dae}
    E \dot{x}(t) = A x(t) + f(t)
\end{equation}
in terms of matrices $A, E \in \bb{C}^{n \times n}$ and a time dependent function $f \colon I \to \bb{C}^n$.

We further assume that the function $f$ serves as an influence on the solution of the DAE.
The interpretation here is that $f$ is a \emph{control} acting on the system over time.
We explicitly formulate the control in terms of yet another linear matrix product
\begin{equation}\label{eq:control-substitution}
    f(t) = B u(t)
\end{equation}
for some natural number $q$, a matrix $B \in \bb{C}^{n \times q}$, and a control variable $u \colon I \to \bb{C}^q$.
Lastly, we introduce an output component $y$ to our modelled system.
This output again takes the shape of a time dependent function $y \colon I \to \bb{C}^q$, and, keeping in line with Equaions~\eqref{eq:lin-const-coeff-dae} and~\eqref{eq:control-substitution}, we formulate it as a linear combination
\begin{equation}\label{eq:output-definition}
    y(t) = C x(t) + D u(t)
\end{equation}
for two matrices $C \in \bb{C}^{q \times n}, D \in \bb{C}^{q \times q}$.

\begin{definition}\label{def:lti}
    A \emph{linear time-invariant (descriptor) system} $\Sigma_{\textsc{LTI}}$ is a system of equations
    \begin{equation}\label{eq:lti}
        \Sigma_{\textsc{LTI}} \colon \left\lbrace
        \begin{aligned}
            E \dot{x}(t) &= A x(t) + B u(t), \\
            y(t) &= C x(t) + D u(t)
        \end{aligned}
        \right.
    \end{equation}
    with matrices $A, E \in \bb{C}^{n \times n}, B \in \bb{C}^{n \times q}, C \in \bb{C}^{q \times n}, D \in \bb{C}^{q \times q}$, and time varying functions $x, \dot{x} \colon I \to \bb{C}^n, u, y \colon I \to \bb{C}^q$.
\end{definition}

Any LTI system in this form is said to act in the time domain, that is the inputs of both the \emph{states} $x, \dot{x}$, and the control $u$ and output variable $y$ are time.
From a system theoretic point of view however we must also consider the system from the \emph{frequency domain}.
We thus describe the system not for a point in time, but rather for a certain frequency with which an input acts on it.
This frequency formulation is obtained by applying the \emph{Laplace transform}
\begin{equation*}\label{eq:laplace-trafo}
    \mcl{L}[f](\omega) = \int\limits_0^\infty \exp{(- \omega t)} f(t) \dif t
\end{equation*}
to the individual components in the LTI system; cf.~\cite{Arendt2011}.
\begin{equation}
    X = \mcl{L}[x],\quad \dot{X} = \mcl{L}[\dot{x}],\quad U = \mcl{L}[u],\quad Y = \mcl{L}[y].
\end{equation}

The Laplace transform of a derivative can be expressed as
\begin{equation*}
    \mcl{L}[\dot{x}](\omega) = \omega \mcl{L}[x](\omega) - \lim\limits_{t \searrow 0} x(t);
\end{equation*}
cf.~\cite[Theorem~9.1]{Doetsch1974}.
Therefore, under the assumption that the matrix pencil $\omega E - A$ is regular we compute from the first equation in~\eqref{eq:lti}
\begin{alignat*}{3}
     & & E \dot{X}(\omega) &= A X(\omega) + B U(\omega) \\
    \implies & & E (\omega X(\omega) - x(0)) &= A X(\omega) + B U(\omega) \\
    \implies & & \omega E X(\omega) - A X(\omega) &= E x(0) + B U(\omega) \\
    \implies & & (\omega E - A) X(\omega) &= E x(0) + B U(\omega) \\
    \implies & & X(\omega) &= (\omega E - A)\inv E x(0) + (\omega E - A)\inv B U(\omega).
\end{alignat*}
We further apply this result to the output part of the LTI system~\eqref{eq:lti}
\begin{alignat*}{2}
    Y(\omega) &= C X(\omega) + D U(\omega) \\
     &= C (\omega E - A)\inv B U(\omega) + D U(\omega) + C (\omega E - A)\inv E x(0).
\end{alignat*}
Notably, the last equation contains a summand which only depends on the initial state of the system.
We can assume this initial value of the time domain function $x(t)$ to be zero $x(0) = x_0 = 0$.
This is a sensible assumption because it corresponds to a system that contains no energy and can only be acted upon through the outside control.
This reformulation thus shows how we can directly map the control acting on the system onto the produced output without having to consider the internal state representation.
More general definitions can also incorporate the initial state dependency in the transfer function, however we refer the reader to the introductory chapter in~\cite{Benner2017}.

\begin{definition}\label{def:transfer-function}
    The \emph{transfer function} $\zeta \colon \bb{C} \to \bb{C}^{q \times q}$ of an LTI system~\eqref{eq:lti} is defined by the expression
    \begin{equation}\label{eq:transfer-function}
        \zeta(\omega) = C (\omega E - A)\inv B + D
    \end{equation}
    for the system matrices $A, E \in \bb{C}^{n \times n}, B \in \bb{C}^{n \times q}, C \in \bb{C}^{q \times n}, D \in \bb{C}^{q \times q}$ and some frequency domain control $U \colon \bb{C} \to \bb{C}^{q}$.
\end{definition}

\begin{remark}
    The matrices used in the definition of the transfer function are not unique.
    For any transfer function multiple tuples of matrices may exist which induce the same transfer function.
    Any such tuple is called a \emph{realization} of the transfer function.
    Across all realizations, a tuple of matrices $(A, B, C, D, E)$ is called minimal if the matrices $A$ and $E$ have the smallest possible dimension amongst all possible realizations.
\end{remark}

The transfer function of an LTI system is an essential tool for constructing reduced order models from it.
We will discuss this further in Chapter~\ref{chap:model-order-reduction}, however we now introduce a key benchmark for this later topic: the poles of the transfer function.
Following~\cite{Benner2017}, an LTI system is called \emph{stable} if all the finite generalized eigenvalues of the matrix pencil $\omega E - A$ are contained in the left half of the complex plain $\bb{C}_{-} = \iset{\sigma \in \bb{C}}{\fk{Re}(\sigma) < 0}$.
Stable in this context refers to the fact that a solution for any stable system tends to zero for $t \rightarrow \infty$ as long as no control is applied ($u(t) \equiv 0$).
This property guarantees that a system is asymptotically stable.
Very closely connected to the eigenvalues of $\omega E - A$ are the poles of the transfer function associated with the realization $(A, B, C, D, E)$.
According to~\cite[Section~2]{Benner2017}, the transfer function $\zeta$ may be written as the rational expression
\begin{equation*}
    \zeta(\omega) = \frac{P(\omega)}{d(s)},
\end{equation*}
where $N$ is a matrix polynomial mapping to $\bb{C}^{q \times n}$ and $d$ is a scalar polynomial representing the least common denominator of all the $q \cdot n$ entries of $\zeta(\omega)$.
The poles of $\zeta$ then are the roots $\iset{\sigma \in \bb{C}}{d(\sigma) = 0}$ of $d$.
In general, the set of poles of the transfer function is a subset of the set of generalized eigenvectors of the matrix pencil $\omega E - A$, however if the realization $(A, B, C, D, E)$ is minimal, both sets are equal.

\itodo{add remarks on differences continuous/discrete time systems}
