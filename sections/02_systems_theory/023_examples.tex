\section{Examples of \texorpdfstring{\ac{LTI}}{LTI} and \texorpdfstring{\ac{PH}}{PH} Systems}

To begin this section, we showcase a few \ac{LTI} models.
Firstly, Example~\ref{ex:heating} models a simple controlled heat equation.
Thereafter we discuss a more complex advection-reaction-diffusion equation in the context of air quality simulations.
Lastly, we demonstrate how higher order systems can be transformed to first order \ac{LTI} systems in Example~\ref{ex:casting-second-order}.

\begin{example}[Controlled Heating]\label{ex:heating}
    Let $\Omega \subseteq \bb{R}^n$ be a closed domain with sufficiently smooth boundary $\Gamma \in \subseteq \bb{R}^{n - 1}$.
    We consider a linear heat equation, and apply a control $u$ on the boundary $\Gamma$ ressembling an exterior heating element to get
    \begin{equation*}
        \pd[t]{} x = \lapl{x} + b \cdot \res{u}{\Gamma}.
    \end{equation*}
    The output $y$ is the measured heat distribution at a set of sensor points $\mcl{S} \subseteq \Omega$ which we compute from
    \begin{equation*}
        y = C \res{x}{\mcl{S}}.
    \end{equation*}
    With this equation, we can immediately transform this problem into the \ac{LTI} formulation~\eqref{eq:lti} with the system matrices $A = \lapl{}, B = b, C, D = 0, E = \id$.
    Figure~\ref{fig:lti-heating} depicts a simple example of this model for a rectangular 2D-domain.
    For this illustration, the sample points are labelled $\mcl{S} = \sset{s_1, \dots, s_4}$.

    \begin{figure}[h]
        \centering
        \begin{tikzpicture}[scale=0.6, auto, swap]
            \node[ghost] (sw) at (0, 0) {};
            \node[ghost] (ne) at (11, 6) {};
            \node[bounded domain, fit=(sw.north east) (ne.south west)] (test) {$\Omega$};

            \node[ghost] (alignmenthelper) at (-3, 0) {};

            \node[ghost] (boundarynode) at (11, 4.75) {};
            \node[barlabel] (boundarylabel) at (14, 6) {$\Gamma$};
            \path[labelline] (boundarylabel) -- (boundarynode);

            \foreach \pos / \name in {%
                {(0, 3.25)/$s_1$}, {(11, 3.25)/$s_2$},%
                {(5.5, 1.5)/$s_3$}, {(8, 5)/$s_4$}%
            }%
                \node[sensor, label=below right:\name] (\name) at \pos {};
        \end{tikzpicture}
        \caption{A 2D rectangular domain with sample points $s_1, \dots, s_4$ for a controlled heating problem.}%
        \label{fig:lti-heating}
    \end{figure}
\end{example}

\begin{example}[{Controlled Air Quality Simulations, adapted from~\cite[Section~2.3]{Antoulas2005}}]
    Modern city scapes have to account for local air quality for several reasons such as health issues, policy-making, and economic factors.
    Thus, being able to simulate different configurations in cheap models before running expensive real-world trials is important for many different interest groups.
    A variety of models can be used to run simulations of the concentrations $c_i \in \bb{R}, i = 1, \dots, S$, of $S \in \bb{N}$ individual chemicals in an open domain $\Omega \subseteq \bb{R}^n$ depending on the velocity of the wind $v \in \bb{R}^n$, diffusion through turbulence $K \in \bb{R}^{n \times n}$, and reactions occuring at rates $R_i \in \bb{R}$.
    Put together, we can express this model using the advection-reaction-diffusion equation
    \begin{equation}\label{eq:air-quality-state-equation}
        \pd[t]{} c_i(x, t) + \dvg \left( v(x, t) c_i(x, t) \right) - \dvg \left( K(x, t) \nabla c_i(x, t) \right) - R_i(c, x, t) = u_i(x, t),\quad i = 1, \dots, S.
    \end{equation}
    In contrast to the controlled heating model from Example~\ref{ex:heating}, we do not use a control variable on the boundary of the domain, but rather at specific places of interest such as factories, roads, residential zones, or forests.
    This control $u_i \colon \Omega \times I \to \bb{R}$ acts in $M_i \in \bb{N}$ individual points for each chemical, and models both chemical emissions as well as localized air purification measures.
    Discretizing~\eqref{eq:air-quality-state-equation} yields the state equation of an \ac{LTI} system $\Sigma_{\msc{lti}}$.
    For the output of the system $\Sigma_{\msc{lti}}$, we consider a set of sensors $\mcl{S} \subseteq \Omega$ placed at locations that may differ from the positions of the control points.
    This can then be codified in the usual manner a linear combination of the internal states $c_i$ and the controls $u_i$ in the equation
    \begin{equation*}
        y_i(t) = C c_i(x, t) + D u_i(x, t).
    \end{equation*}
    We showcase a domain similar to the one from to Example~\ref{ex:heating} for the 2D setting in Figure~\ref{fig:air-quality}.

    \begin{figure}[h]
        \centering
        \begin{tikzpicture}[scale=0.6, auto, swap]
            \node[ghost] (sw) at (0, 0) {};
            \node[ghost] (ne) at (11, 6) {};
            \node[bounded domain, fit=(sw.north east) (ne.south west)] (test) {$\Omega$};

            \foreach \pos / \name in {%
                {(2, 1)/$u_1$}, {(3.5, 5)/$u_2$},%
                {(8, 5)/$u_3$}%
            }%
                \node[control, label=below right:\name] (\name) at \pos {};

            \foreach \pos / \name in {%
                {(0, 3.25)/$s_1$}, {(9, 2.25)/$s_2$},%
                {(5.5, 1.5)/$s_3$}%
            }%
                \node[sensor, label=below right:\name] (\name) at \pos {};
        \end{tikzpicture}
        \caption{A 2D rectangular domain with controls $u_1, u_2, u_3$, and sample points $s_1, s_2, s_3$ for an air quality control problem with a single chemical of interest.}%
        \label{fig:air-quality}
    \end{figure}
\end{example}

\begin{example}[Casting Second Order Systems as \acp{LTI}]\label{ex:casting-second-order}
    We consider a second order model with the matrices $M, E, A \in \bb{R}^{n \times n}, B \in \bb{R}^{n \times m}, C \in \bb{R}^{m \times n}, D \in \bb{R}^{m \times m}$ in the extended \ac{LTI} form
    \begin{align*}
        M \ddot{q} + E \dot{q} &= A q + B u, \\
        y &= C q + D u.
    \end{align*}
    To express this as a first order \ac{LTI} system, we apply the transformation $x = \begin{pmatrix}
        q \\
        \dot{q}
    \end{pmatrix}$
    and compute the transformation relations $\dot{x_1} = x_2, M \dot{x}_2 = A x_1 - E x_2 + Bu$.
    These in term can be put into \ac{LTI} form
    \begin{equation*}
        \Sigma_{\msc{lti}} \colon \left\lbrace
        \begin{aligned}
            \begin{pmatrix}
                \id & 0 \\
                0 & M
            \end{pmatrix} x &= \begin{pmatrix}
                0 & \id \\
                A & -E
            \end{pmatrix} x + \begin{pmatrix}
                0 \\
                B
            \end{pmatrix} u, \\
            y &= \begin{pmatrix}
                C & 0
            \end{pmatrix} x + D u.
        \end{aligned}
        \right.
    \end{equation*}
\end{example}

To illustrate the abstract port-based modelling discussed at the end of Section~\ref{sec:phltis}, we showcase a few examples of problems in the \ac{PH} framework.
In Example~\ref{ex:ms-system}, we give a direct application of classical mechanical physics adapted from~\cite[Example~2.1]{VanDerSchaft2014}.
Finally, we highlight the approach to construct a \ac{PH} formulation with a cubic storage function for the inviscid Burgers' equation adapted from~\cite[Example~2.1]{Maschke2005} in Example~\ref{ex:inviscid-burgers}.

\begin{example}[Mass-Spring-Damper System]\label{ex:ms-system}
    We consider a \ac{MSD} that consists of three components: a mass $m$, a linear damper with damping coefficient $c$, and a linear spring with spring constant $k$ as depicted in Figure~\ref{fig:single-msd}.
    Under the assumption that the mass moves in only one direction and without any friction, we can express the movement of the mass in terms of its kinetic energy $\frac{1}{2m} p^2$ and its potential energy $\frac{1}{2} k q^2$ with the system's states $p \in \bb{R}$ and $q \in \bb{R}$.
    To this end, we apply Hooke's law and calculate the following effort and flow variables $e_m, e_s$ and $f_m, f_s$ respectively
    \begin{equation*}
        f_m = -\dot{p},\quad f_s = -\dot{q},\quad e_m = \frac{\dif}{\dif p} \left( \frac{1}{2m} p^2 \right) - c \cdot e_s,\quad e_s = \frac{\dif}{\dif q} \left( \frac{1}{2} k q^2 \right).
    \end{equation*}
    All that is left now, is to formulate the interconnection between both systems.
    For this, we have to join the effort exerted at the end of the spring $e_s$ to the force applied to the point mass $-f_m$, and furthermore also tie the mass' velocity effort $e_m$ to the force of the spring end point $f_s$.
    This results in the interconnection
    \begin{equation*}
        -f_s = e_m,\quad f_m = e_s.
    \end{equation*}
    All in all, we obtain the storage function $\mcl{H}(p, q) = \frac{1}{2m} p^2 + \frac{1}{2} k q^2$.
    Therefore, the corresponding system in \ac{PH} formulation takes the form
    \begin{equation*}
        \begin{pmatrix}
            \dot{p} \\
            \dot{q}
        \end{pmatrix} =
        \begin{pmatrix}
            -c & -1 \\
            1 & 0
        \end{pmatrix}
        \begin{pmatrix}
            \pd[p]{} \mcl{H}(p, q) \\
            \pd[q]{} \mcl{H}(p, q)
        \end{pmatrix} =
        \begin{pmatrix}
            -k q \\
            \frac{1}{m} p
        \end{pmatrix}.
    \end{equation*}

    \begin{figure}[h]
        \centering
        \begin{tikzpicture}[scale=0.6, auto, swap,%
            damper/.pic = {
                \coordinate (-east);
                \coordinate[left=1mm of -east] (-west);
                \draw[very thick] ($(-east)+(0,2mm)$) -- ++ (0,-4mm);
                \draw[semithick]  ($(-east)+(0,3mm)$) -| ++ (-1mm,-6mm) -- ++ (1mm,0);
            }]
            \foreach \pos / \name in {%
                {(.5, 0)/wallbottom}, {(.5, 4)/walltop},%
                {(0, .5)/wall0}, {(0, 1.5)/wall1},%
                {(0, 2.5)/wall2}, {(0, 3.5)/wall3},%
                {(0, 4.5)/wall4}, {(.5, 1)/wallpoint1},%
                {(.5, 2)/wallpoint2}, {(.5, 3)/wallpoint3},%
                {(3, 2.75)/springmasspoint}, {(.5, 2.75)/springwallpoint},%
                {(1.75, 1.25)/damperpoint}, {(3, 1.25)/dampermasspoint},%
                {(.5, 1.25)/damperwallpoint},%
                {(3, .5)/mass0}, {(7, .5)/mass1},%
                {(7, 3.5)/mass2}, {(3, 3.5)/mass3},%
                {(9.5, 2)/controlpoint}, {(7, 2)/controlmasspoint}%
            }%
                \node[ghost] (\name) at \pos {};

            \path[labelline] (wall0) -- (wallbottom) -- (walltop) -- (wall4);
            \path[labelline] (wall1) -- (wallpoint1);
            \path[labelline] (wall2) -- (wallpoint2);
            \path[labelline] (wall3) -- (wallpoint3);

            \draw[decoration={aspect=0.3, segment length=1.5mm, amplitude=2mm,coil},decorate] (springwallpoint) -- (springmasspoint);
            \pic (damper) at (damperpoint) {damper};
            \path[labelline] (damper-east) -- (dampermasspoint);
            \path[labelline] (damper-west) -- (damperwallpoint);

            \path[labelline] (mass0) -- (mass1);
            \path[labelline] (mass1) -- (mass2);
            \path[labelline] (mass2) -- (mass3);
            \path[labelline] (mass3) -- (mass0);
            \node[ghost] (masscenter) at (5, 2) {$m$};

            \path[force, label=test] (controlpoint) -- (controlmasspoint);
            \node[ghost, label=above:$u$] (controlghost) at (8.25, 2) {};
        \end{tikzpicture}
        \caption{A \acl{MSD} system with one element.}%
        \label{fig:single-msd}
    \end{figure}
\end{example}

\begin{example}[{Burgers' Equation adapted from~\cite[Example~2.1]{Maschke2005}}]\label{ex:inviscid-burgers}
    To demonstrate non-quadratic Hamiltonians, we first consider the inviscid homogeneous Burgers' equation on a one-dimensional spatial domain $\interval[open]{a}{b} \subset \bb{R}$ in the form of
    \begin{equation}\label{eq:inviscid-burgers}
        \pd[t]{} u + \pd[x]{} u^2 = 0.
    \end{equation}
    In this case, we can express the associated Hamiltonian as
    \begin{equation}\label{eq:burgers-hamiltonian}
        \mcl{H}(u) = \int\limits_a^b \frac{u^3}{6} \dif x.
    \end{equation}
    Applying the variational derivative $\delta_u H(u) = \frac{u^2}{2}$, we compute the reformulation of Equation~\eqref{eq:inviscid-burgers} in terms of the Hamiltonian~\eqref{eq:burgers-hamiltonian} as the following equation
    \begin{equation*}
        \pd[t]{} u = - \pd{} \left( \delta_u \mcl{H}(u) \right) = -\pd{} \frac{u^2}{2}.
    \end{equation*}
    In this case, the time derivative of the Hamiltonian $\mcl{H}$ is conserved by
    \begin{equation*}
        \frac{\dif}{\dif t} \mcl{H}(u) = - \int\limits_a^b \delta_u \mcl{H}(u) \cdot \pd{} \left( \delta_u \mcl{H}(u) \right) \dif x = \frac{1}{2} \left( {u(b)}^4 - {u(a)}^4 \right),
    \end{equation*}
    where we computed the last equality using integration by parts.
\end{example}
