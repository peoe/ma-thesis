\section{Examples of LTI and pH Systems}

To begin this section we showcase a few models in the LTI setting.
Firstly, Example~\ref{ex:heating} models a controlled heat eqaution which might appear in cases of optimal heating.

\begin{example}[Controlled Heating]\label{ex:heating}
    We consider a standard heat equation, apply a control to it amounting to a heating surface, and measure the heat distribution within the system at some finite set of sensor points as the output.
    We commence from the standard formulation and add in a boundary control term
    \begin{equation*}
        \pd[t]{} x = \lapl{x} + b \res{u}{\Gamma}
    \end{equation*}
    such that $\Gamma$ is a subset of the domain boundary.
    The output $y$ is directly coupled to a set of sensor points $\mcl{S}$
    \begin{equation*}
        y = c \res{x}{\mcl{S}} + d \res{u}{\Gamma \cap \mcl{S}}.
    \end{equation*}
    We can immediately put this problem into the LTI formulation with $A u = \lapl{u}, B u = b \res{u}{\Gamma}, C x = c \res{x}{\mcl{S}}, D u = d \res{u}{\Gamma \cap \mcl{S}}$.
\end{example}

\begin{example}[Casting Second Order Systems as LTIs]
    Consider a general description of a second order model
    \begin{align*}
        M \ddot{q} + E \dot{q} &= A q + B u \\
        y &= C q + D u.
    \end{align*}
    To express this as a first order LTI we apply the transformation $x = \begin{pmatrix}
        q \\
        \dot{q}
    \end{pmatrix}$
    and compute the constituent relations $M \dot{x}_2 = A x_1 - E x_2 + Bu, \dot{x_1} = x_2$.
    These in term can be put into LTI form
    \begin{equation*}
        \Sigma_{\textsc{lti}} \colon \left\lbrace
        \begin{aligned}
            \begin{pmatrix}
                \id & 0 \\
                0 & M
            \end{pmatrix} x &= \begin{pmatrix}
                0 & \id \\
                A & -E
            \end{pmatrix} x + \begin{pmatrix}
                0 \\
                B
            \end{pmatrix} u, \\
            y &= \begin{pmatrix}
                C & 0
            \end{pmatrix} x + D u.
        \end{aligned}
        \right.
    \end{equation*}
\end{example}

\begin{example}[{Air Quality Simulations; cf.~\cite[Section~2.3]{Antoulas2005}}]
    Modern city scapes have to account for local air quality for several reasons such as health issues, policy-making, and economic factors.
    Thus, being able to simulate different settings in cheap models before running expensive trials is important for many different interest groups.
    A variety of models can be used to run simulations of the concentrations $c_i$ of individual chemicals, among them conservation laws such as
    \begin{equation}\label{eq:air-quality-state-equation}
        \pd[t]{} c_i + \nabla \cdot v(x, t) c_i - \nabla \cdot (K(x, t) \nabla c_i) = R_i(c) + u_i,\quad i = 1, \dots, S
    \end{equation}
    incorporating advection along the wind velocity vector $v(x, t)$, diffusion via turbulence $K(x, t)$, reaction terms $R_i(c)$ indicating the rate of change of the concentration of each chemical, and $U_i$ the control term acting as sources or sinks for the chemicals.
    Discretizing~\eqref{eq:air-quality-state-equation} yields the state equation of an LTI system $\Sigma_{\textsc{lti}}$.
    For the output of $\Sigma_{\textsc{lti}}$ we can imagine a set of sensors placed around locations of interest such as factories, roads, or residential zones.
    These can be codified in the usual manner as
    \begin{equation*}
        y_i = C c_i + D u_i.
    \end{equation*}

    \itodo{make sure this really only is advection reaction}
\end{example}

To illustrate the abstract port-based modelling discussed at the end of Section~\ref{sec:ltis} we state a few examples of problems stated in the potr-Hamiltonian framework.
In Example~\ref{ex:ms-system} we give a direct application of classical mechanical physcs adapted from~\cite[Example~2.1]{VanDerSchaft2014}.
Example~\ref{ex:burgers} highlights the approach for the viscuous Burgers' equation resulting in a cubic Hamiltonian; cf.~\cite[Example~2.1]{Maschke2005}.
Finally, in an effort to illustrate the capability to model veen more complex systems, we rephrase Maxwell's equations as a port-Hamiltonian system; cf.~\cite[Section~2]{Haine2022}.

\itodo{MHD equadtions too?}

\begin{example}[Mass Spring System]\label{ex:ms-system}
    Consider a system made up from two components: a point mass and a linear spring.
    Under the assumptions that the mass moves in only one direction and without friction we can express the movement of the mass in terms of its kinteic energgy $\frac{1}{2m} p^2$
    \begin{equation*}
        \dot{p} = -f_m,\quad e_m = \frac{\dif}{\dif p} \left( \frac{1}{2m} p^2 \right).
    \end{equation*}
    In a similar way we apply Hooke's law to compute the spring's potential energy $\frac{1}{2} k q^2$
    \begin{equation*}
        \dot{q} = -f_s,\quad e_s = \frac{\dif}{\dif q} \left( \frac{1}{2} k q^2 \right).
    \end{equation*}
    All that is left now is to state the interconnection between both systems.
    To this end, we have to link the force exerted at the end of the spring $e_s$ to the force applied to the point mass $-f_m$, and necessarily also connect the mass' velocity $e_m$ to the velocity of the spring end point $e_s$.
    This results in the interconnection
    \begin{equation*}
        -f_s = e_m,\quad f_m = e_s.
    \end{equation*}
    All in all we obtain the Hamiltonian $\mcl{H}(p, q) = \frac{1}{2m} p^2 + \frac{1}{2} k q^2$, and the corresponding system in port-Hamiltonian formulation
    \begin{equation*}
        \begin{pmatrix}
            \dot{p} \\
            \dot{q}
        \end{pmatrix} =
        \begin{pmatrix}
            0 & -1 \\
            1 & 0
        \end{pmatrix}
        \begin{pmatrix}
            \pd[p]{} \mcl{H}(p, q) \\
            \pd[q]{} \mcl{H}(p, q)
        \end{pmatrix} =
        \begin{pmatrix}
            -k q \\
            \frac{1}{m} p
        \end{pmatrix}.
    \end{equation*}

    \itodo{insert image?}
\end{example}

\begin{example}[{Burgers' Equation; cf.~\cite[Example~2.1]{Maschke2005}}]\label{ex:burgers}
    Consider the inviscid homogeneous Burgers' equation on a one-dimensional spatial domain $\interval[open]{a}{b}$
    \begin{equation*}
        \pd[t]{} u + \pd[x]{} u^2 = 0.
    \end{equation*}
    In this case we can express the associated Hamiltonian as
    \begin{equation*}
        \mcl{H}(u) = \int\limits_a^b \frac{u^3}{6} \dif x.
    \end{equation*}
    Applying the variational derivative $\delta_u H(u) = \frac{u^2}{2}$ we compute the port-Hamiltonian reformulation
    \begin{equation*}
        \pd[t]{} u = - \pd{} \delta_u H(u) = -\pd{} \frac{u^2}{2}.
    \end{equation*}
    In this case the time derivative of the Hamiltonian $\mcl{H}$ is constrained by
    \begin{equation*}
        \frac{\dif}{\dif t} \mcl{H}(u) = - \int\limits_a^b \delta_u \mcl{H}(u) \cdot \pd{} \delta_u \mcl{H}(u) \dif x = \frac{1}{2} \left( {u(b)}^4 - {u(a)}^4 \right)
    \end{equation*}
    where we computed the last equality using integration by parts.
\end{example}

\begin{example}[Maxwell's Equations]\label{ex:maxwell}
    For an electric field $E$, a magnetic field $B$, auxiliary fields $D$ and $H$, a free charge density $\rho$, and a free current density $j$ Maxwell's equations are given by
    \begin{equation*}
        \begin{aligned}
            \pd[t]{} D - \curl{H} &= -j, &\pd[t]{} B + \curl{E} &= 0, \\
            \dvg{D} &= \rho, &\dvg{B} &= 0.
        \end{aligned}
    \end{equation*}
    On the boundary a linear time-invariant impedance bonudary condition
    \begin{equation*}
        \pi_t(E) = k * \left( \res{H}{\Gamma} \times n \right)
    \end{equation*}
    is imposed for the unit outer normal $n$, the time-domain convolution $*$, and a causal kernel $k$, meaning that its Laplace tranfsorm is a positive-real function.
    The derivation of the power balance is quite involved so we refer the reader to $\cdot$.
    Initially, we use the integral equality
    \begin{equation*}
        \inner{E}{\pd[t]{} D}{\Omega} + \inner{H}{\pd[t]{} B}{\Omega} = - \inner*{\res{H}{\Gamma} \times n}{\pi_t(E)}{\Gamma} - \inner{j}{E}{\Omega}.
    \end{equation*}
    Combined with the following three constitutive relations
    \begin{equation*}
        D(t, x) = \eps(x) E(t, x),\quad H(t, x) = \frac{1}{\mu(x)} B(t, x),\quad j(t, x) = \sigma(x) E(t, x)
    \end{equation*}
    for $\eps$ the dielectric permittivity, $\mu$ the magnetic permittivity, and $\sigma \geq 0$ the conductivity, this results in the Hamiltonian
    \begin{equation*}
        \mcl{H} = \frac{1}{2} \inner{E}{D}{\Omega} + \frac{1}{2} \inner{H}{B}{\Omega}.
    \end{equation*}
    With the energy variables $\alpha_D = D, \alpha_B = B, f_j = E$ and the coenergy varables $e_D = \delta_D \mcl{H} = E, e_B = \delta_B \mcl{H} = H, e_j = j$ we deduce
    \begin{equation*}
        \mcl{H} = \frac{1}{2} \inner*{\frac{1}{\eps} \alpha_D}{\alpha_D}{\Omega} + \frac{1}{2} \inner*{\frac{1}{\mu} \alpha_B}{\alpha_B}{\Omega}
    \end{equation*}
    as well as with the output $\inner{y}{u}{\Gamma} = - \inner{\res{H}{\Gamma} \times n}{\pi_t(E)}{\Gamma}$ the final port-Hamiltonian system is
    \begin{equation*}
        \begin{pmatrix}
            \pd[t]{} \alpha_D \\
            \pd[t]{} \alpha_B \\
            f_j
        \end{pmatrix} =
        \begin{pmatrix}
            0 & \curl & -\id \\
            -\curl & 0 & 0 \\
            \id & 0 & 0
        \end{pmatrix}
        \begin{pmatrix}
            e_D \\
            e_B \\
            e_j
        \end{pmatrix}.
    \end{equation*}

    \itodo{check for the definition of $\pi_t$ in Buffa, Costabel, Sheen; 2002: On traces for $H(curl, \Omega)$ in Lipschitz domains}
\end{example}
