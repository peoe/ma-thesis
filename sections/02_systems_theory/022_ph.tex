\section{Port-Hamiltonian Systems}

In the previous section we saw how we can model dynamical systems through the lens of considering the output of a system for a specific control acting on it.
This idea can be further extended upon by incorporating physical properties into the description of the underlying dynamical systems through so-called ports, entities that describe how certain parts of the model are interacting with each other.
We first give a description of linear time-invariant port-Hamiltonian DAEs before mentioning more generalized versions.
We also define the property of passivity for such dynamical systems.
Furhter, we explicitly note that from now on we consider the system matrices to be real-valued, which is in contrast to the complex-valued definitions in the previous section.

These first remarks are closely adapted from~\cite[Definition~4.9]{Mehrmann2022}.
In port-Hamiltonian systems, we split the state interaction matrix $A \in \bb{R}^{m \times n}$ into two matrices $J, R \in \bb{R}^{m \times m}$ and a state transformation matrix $Q \in \bb{R}^{m \times n}$.
The $J$ matrix is responsible for preserving energy in the system, reflected in the skew-symmetry of the matrix $-J = J\trans$.
The $R$ matrix on the other hand dissipates energy through the internal states.
We consider $R$ to be a symmetric positive semidefinite matrix, denoted $R \in \mcl{S}_{\curlyeqsucc}^{n \times n}(\bb{R})$.
We then couple these matrices
\begin{equation*}
    A = (J - R)Q
\end{equation*}
such that the energy dissipation matrix takes the correct sign when putting physical formulations into the port-Hamiltonian framework.
Apart from the matrices coupling the internal states we apply similar splits to the matrices relating to the controls and outputs of the system
\begin{equation*}
    B = G - P,\quad C = (G + P)\trans Q,\quad D = S - N.
\end{equation*}
Additionally, we require that all these matrices conform to the structural constraints
\begin{equation*}
    \begin{pmatrix}
        Q & 0 \\
        0 & \id
    \end{pmatrix}\trans
    \begin{pmatrix}
        R & P \\
        P\trans & S
    \end{pmatrix}
    \begin{pmatrix}
        Q & 0 \\
        0 & \id
    \end{pmatrix}
    \in \mcl{S}_{\curlyeqsucc}^{(n + q) \times (n + q)},\quad -N = N\trans \in \bb{R}^{q \times q},\quad G \in \bb{R}^{n \times q}.
\end{equation*}

We could now rephrase~\eqref{eq:lti} in terms of these matrices, however there is one more aspect to the port-Hamiltonian formulation we yet lack: the \emph{Hamiltonian}.
The Hamiltonian is motivated in Physics to model the total amount of energy within a system.
In this case we restrict ourselves to a quadratic Hamiltonian formulation
\begin{equation}\label{eq:quadratic-hamiltonian}
    \mcl{H} \colon I \times \bb{R}^{n} \to \bb{R},\quad \mcl{H}(t, x) \mapsto \frac{1}{2} x(t)\trans Q\trans E x(t).
\end{equation}

\begin{definition}\label{def:phlti}
    A \emph{port-Hamiltonian linear time-invariant (descriptor) system with quadratic Hamiltonian} is a system of equations
    \begin{equation}\label{eq:phlti}
        \Sigma_{\msc{phlti}} \colon \left\lbrace
        \begin{aligned}
            E \dot{x}(t) &= (J - R) Q x(t) + (G - P) u(t), \\
            y(t) &= (G + P)\trans Q x(t) + (S - N) u(t)
        \end{aligned}
        \right.
    \end{equation}
    together with a Hamiltonian $\mcl{H}(t, x) = \frac{1}{2} x(t)\trans Q\trans E x(t)$ if the system matrices $E, Q \in \bb{R}^{m \times n}, J, R \in \bb{R}^{m \times m}$, $G, P \in \bb{R}^{m \times q}$, and $S, N \in \bb{R}^{q \times q}$ satify the properties
    \begin{equation}\label{eq:ph-matrix-structure}
        -J = J\trans,\quad -N = N\trans,\quad
        \begin{pmatrix}
            Q & 0 \\
            0 & \id
        \end{pmatrix}\trans
        \begin{pmatrix}
            R & P \\
            P\trans & S
        \end{pmatrix}
        \begin{pmatrix}
            Q & 0 \\
            0 & \id
        \end{pmatrix}
        \in \mcl{S}_{\curlyeqsucc}^{(n + q) \times (n + q)}.
    \end{equation}
\end{definition}

\begin{remark}
    Because we inherently derived this representation from the matrices $(A, B, C, D, E)$, we can in turn describe any linear time-invariant port-Hamiltonian system $\Sigma_{\msc{phlti}}$ as an LTI system $\Sigma_{\msc{lti}}$ by defining its system matrices to be
    \begin{equation*}
        A \coloneqq (J - R) Q,\quad B \coloneqq G - P,\quad C = (G + P)\trans Q, D = S - N.
    \end{equation*}
\end{remark}

\begin{remark}
    In Definition~\ref{def:phlti} we stated the general definition of a port-Hamiltonian system.
    In practice, it is common to transform the system such that it no longer contains the $Q$ matrix.
    This is achieved by multiplying the first equation in~\eqref{eq:phlti} by $Q\trans$ from the left and defining
    \begin{equation*}
        H \coloneqq Q\trans E,\quad \tilde{J} \coloneqq Q\trans J Q,\quad \tilde{R} \coloneqq Q\trans R Q, \tilde{G} \coloneqq Q\trans G, \tilde{P} \coloneqq Q\trans P.
    \end{equation*}
    Using this substitution we can write our original pH system as
    \begin{equation*}
        \Sigma \colon \left\lbrace
        \begin{aligned}
            H \dot{x}(t) &= (\tilde{J} - \tilde{R}) x(t) + (\tilde{G} - \tilde{P}) u(t), \\
            y(t) &= (\tilde{G} + \tilde{P})\trans x(t) + (S - N) u(t)
        \end{aligned}
        \right.
    \end{equation*}
    in which the matrix $H$ directly represents the Hamiltonian function by inserting $H = Q\trans E$ into~\eqref{eq:quadratic-hamiltonian}.
\end{remark}

\begin{remark}
    For a nonzero direct feedthrough term $D \coloneqq S - N$ we can reformulate any pH LTI system $\Sigma_{\msc{phlti}}$ such that we eliminate the feedthrough term.
    We begin by decomposing $D$ according to~\cite[Theorem~2.13]{Mehrmann2022} into $D = U_D \begin{pmatrix}
        D_1 & 0 \\
        0 & 0
    \end{pmatrix} U_D\trans$ for some orthogonal matrix $U_D$ and the regular component $D_1$.
    We then partition the control-state components along the corresponding coordinates such that
    \begin{equation*}
        (G - P) U_D = \begin{pmatrix}
            G_1 - P_1 & G_2 - P_2
        \end{pmatrix},\quad U_D\trans u = \begin{pmatrix}
            u_1 \\
            u_2
        \end{pmatrix},\quad U_D\trans y = \begin{pmatrix}
            y_1 \\
            y_2
        \end{pmatrix}.
    \end{equation*}
    Hereafter we calculate that we can substitute a new state variable $z \coloneqq D_1 u_1 + P_1\trans x$ into the system
    \begin{equation*}\label{eq:feedthrough-elimination}
        \Sigma_{\msc{phlti}} \colon \left\lbrace
        \begin{aligned}
            \begin{pmatrix}
                E & 0 \\
                0 & 0
            \end{pmatrix} \begin{pmatrix}
                x \\
                z
            \end{pmatrix} &= \begin{pmatrix}
                J - R & 0 \\
                D_1\inv P_1\trans & -D_1\inv
            \end{pmatrix} \begin{pmatrix}
                x \\
                z
            \end{pmatrix} + \begin{pmatrix}
                G_1 - P_1 \\
                \id
            \end{pmatrix} u_1 + \begin{pmatrix}
                G_2 \\
                0
            \end{pmatrix} u_2, \\
            y_1 &= \begin{pmatrix}
                G_1\trans & \id
            \end{pmatrix} \begin{pmatrix}
                x \\
                z
            \end{pmatrix}, \\
            y_2 &= G_2\trans x.
        \end{aligned}
        \right.
    \end{equation*}
    As a last step we multiply the state equation in~\eqref{eq:feedthrough-elimination} from the left with the matrix $\begin{pmatrix}
        \id & P_1 \\
        0 & \id
    \end{pmatrix}$ and finally partition the state block matrix appropriately such that~\cite[Theorem~4.12]{Mehrmann2022} guarantees that the new system is once again port-Hamiltonian.
\end{remark}

Because port-Hamiltonian systems are directly motivated by the energy contained within them we are interested in estimating the change in total energy $\mcl{H}(t)$ at any point in time $t$.
Considering the physical motivation, it makes sense to bound $\frac{\dif}{\dif t} \mcl{H}$ by the control $u$ and the output $y$: the system itself preserves energy, and thus a change in energy can only take place by influx through the control or dispersal by the system's output.
In fact, we can derive the following \emph{dissipation inequality} thanks to the positive semidefiniteness of the system matrices as specified in~\eqref{eq:ph-matrix-structure} as done in~\cite[Theorem~6.1]{Mehrmann2022}
\begin{equation}\label{eq:dissipation-inequality}
    \mcl{H}(x(t_1)) - \mcl{H}(x(t_0)) \leq \int\limits_{t_0}^{t_1} u(\tau)\trans y(\tau) \dif \tau.
\end{equation}
If the state $x(t_0)$ contains no energy, e.\ g.\ $x(t_0) = 0$ we can trivially transform the inequality to
\begin{equation*}
    0 \leq \mcl{H}(x(t_1)) \leq \int\limits_0^t u(\tau)\trans y(\tau) \dif \tau
\end{equation*}
because the energy of the state $x(t_1)$ cannot be less than zero.
Additionally, while we motivated the dissipation inequality as a property of port-Hamiltonian systems, it is by no means restricted to port-Hamiltonian systems, in and of itself only requiring the existence of an energy storage function $\mcl{H}$ to serve as a proxy for the total system energy.

\begin{definition}\label{def:passivity}
    A system $\Sigma$ with an energy storage function $\mcl{H}$ is defined to be \emph{passive} if it satisfies the dissipation inequality~\eqref{eq:dissipation-inequality} for any two points $t_0, t_1, t_0 \leq t_1$ in time.
\end{definition}

\begin{remark}
    Passivity is a very important property for port-based modelling of dynamical systems because it allows for the conjunction of any two passive systems through a power-conserving or dissipative interconnection; cf.~\cite{Mehrmann2022, Morandin2022}.
\end{remark}

\begin{remark}
    Any passive LTI system $\Sigma_{\msc{lti}}$ given by a realization $(A, B, C, D, E)$ with a quadratic Hamiltonian $\mcl{H}(x) = x\trans H x$ allows for a port-Hamiltonian $\Sigma_{\msc{phlti}}$ if the matrix $H$ is positive semidefinite and it satisfies $\knl{H} \subseteq \knl{A}$; cf.~\cite[Theorem~7.1]{VanDerSchaft2014}.
\end{remark}

Generally speaking, Definition~\ref{def:phlti} is too restrictive because the system~\eqref{eq:phlti} only allows for coefficients constant in time.
In the final part of this chapter we highlight a three approaches to consider more general formulations.

\begin{remark}
    The first of these constitutes itself in the application of linear time-varying operators.
    This results in the following linear port-Hamiltonian system
    \begin{equation}\label{eq:lph}
        \Sigma_{\msc{lph}} \colon \left\lbrace
        \begin{aligned}
            E(t) \dot{x}(t) + E(t) K(t) x(t) &= (J(t) - R(t)) Q(t) x(t) + (G(t) - P(t)) u(t), \\
            y(t) &= (G(t) + P(t))\trans Q(t) x(t) + (S(t) - N(t)) u(t)
        \end{aligned}
        \right.
    \end{equation}
    with the quadratic Hamiltonian $\mcl{H}(t, x) \coloneqq \frac{1}{2} x\trans Q(t)\trans E(t) x(t)$.
    The port-Hamiltonian constraints on the dissipative elements $P, R$, and $S$ stays the same as for a time-invariant system $\Sigma_{\msc{phlti}}$, however the condition for the previously simply skew-symmetric part shifts to the need for a skew-adjoint differential operator; cf.~\cite[Definition~4.8]{Mehrmann2022}.
\end{remark}

\begin{remark}
    The second approach is very similar to~\eqref{eq:lph}, however in addition to the time dependency we also allow for the operators to change based on the system state.
    This then results in a nonlinear port-Hamiltonian system such as
    \begin{equation}\label{eq:nlph}
        \Sigma_{\msc{nlph}} \colon \left\lbrace
        \begin{aligned}
            E(t, x) \dot{x}(t) + r(t, x) &= (J(t, x) - R(t, x)) \eta(t, x) + (G(t, x) - P(t, x)) u(t), \\
            y(t) &= (G(t, x) + P(t, x))\trans \eta(t, x) + (S(t, x) - N(t, x)) u(t)
        \end{aligned}
        \right.
    \end{equation}
    with the Hamiltonian implicitly defined through the differential equaionts
    \begin{equation*}
        \pd{} \mcl{H}(t, x) = E(t, x)\trans \eta(t, x),\quad \text{ and }\quad \pd[t]{} \mcl{H}(t, x) = \eta(t, x)\trans r(t, x),
    \end{equation*}
    alongside similar constraints on the energy-conservative and dissipative parts of the port-Hamiltonian system as in the linear time-invariant case; cf.~\cite[Definition~4.1]{Mehrmann2022}.
\end{remark}

\begin{remark}
    For the most general representation of port-Hamiltonian systems we have to take a large conceptual leap, for a more detailed discussion we refer to~\cite[Section~2.2]{VanDerSchaft2014}.
    Consider a finite-dimensional vector space $\mcl{F}$ and denote its dual space $\mcl{E} = \mcl{F}'$ endowed with the usual dual product ${\langle e, f \rangle}_{\mcl{E} \times \mcl{F}} = e(f)$ for some elements $f \in \mcl{F}, e \in \mcl{E}$.
    We then call a subspace $\mcl{D} \sse \mcl{F} \times \mcl{E}$ a \emph{Dirac structure} if $\mcl{D}$ satisfies
    \begin{enumerate}
        \item ${\langle e, f \rangle}_{\mcl{E} \times \mcl{F}}$ for all pairs $(f, e) \in \mcl{D}$, and
        \item $\dim{(\mcl{D})} = \dim{(\mcl{F})}$.
    \end{enumerate}
    In this setting we can thus define a port-Hamiltonian system to be the Dirac structure
    \begin{equation*}
        \mcl{D} \sse \mcl{T}_x \mcl{X} \times \mcl{T}'_x \mcl{X} \times \mcl{F}_R \times \mcl{E}_R \times \mcl{F}_P \times \mcl{E}_P,
    \end{equation*}
    where $\mcl{X}$ denotes an abstract state space, $\mcl{T}_x \mcl{X}$ and $\mcl{T}'_x \mcl{X}$ the respective tangent and dual tangent spaces of $\mcl{X}$, a resistive Dirac structure $\mcl{F}_R \times \mcl{E}_R$, as well as an external port Dirac structure $\mcl{F}_P \times \mcl{E}_P$.
    For a Hamiltonian function $\mcl{H} \colon \mcl{X} \to \bb{R}$, the first two factors can be characterized by the inner state energy preservig relation
    \begin{equation*}
        \frac{\dif}{\dif t} H(x) = \pd{} \mcl{H}(x)\trans \dot{x} = 0.
    \end{equation*}
    Similarly, the resistive (dissipative) structure and the external port satisfy
    \begin{equation*}
        \frac{\dif}{\dif t} \mcl{H}(x) \leq 0,\quad \frac{\dif}{\dif t} \mcl{H}(x) \leq e_p\trans f_p,\quad 0 = \pd{} \mcl{H}(x)\trans \dot{x} + e_R\trans f_R + e_P\trans f_P.
    \end{equation*}
\end{remark}
