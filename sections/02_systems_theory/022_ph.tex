\section{Port-Hamiltonian Systems}

In the previous section we saw how we can model dynamical systems through the lens of considering the output of a system for a specific control acting on the system.
This idea can be taken furhter by incorporating inherent physical properties into the description of dynamical systems.
We first give a description of linear time-invariant port-Hamiltonian DAEs before mentioning more generalized versions.
We also define the properties of stability and passivity of such systems.
Furhter, we explicitly note that from now on we consider the system matrices to be real-valued, which is in contrast to the complex-valued definitions in the previous section.

These first remarks stick close to~\cite[Definition~4.9]{Mehrmann2022}.
In port-Hamiltonian systems, we split the state interaction matrix $A \in \bb{R}^{m \times n}$ into two matrices $J, R \in \bb{R}^{m \times m}$ and a state transformation matrix $Q \in \bb{R}^{m \times n}$.
The $J$ matrix is responsible for preserving energy in the system, reflected in the skew-symmetry of the matrix $-J = J\trans$.
The $R$ matrix on the other hand dissipates energy through the internal states.
We consider $R$ to be a symmetric positive semidefinite matrix, denoted $R \in \mcl{S}_{\curlyeqsucc}^{n \times n}(\bb{R})$.
We then couple these matrices
\begin{equation*}
    A = (J - R)Q
\end{equation*}
such that the energy dissipation matrix takes the correct sign when putting physical formulations into the port-Hamiltonian framework.
Apart from the matrices coupling the internal states we perform similar splits to the matrices relating to the controls and outputs of the system
\begin{equation*}
    B = G - P,\quad C = (G + P)\trans Q,\quad D = S - N.
\end{equation*}
Additionally, we require that all these matrices conform to the following structural constraints
\begin{equation}\label{eq:ph-matrix-structure}
    \begin{pmatrix}
        Q & 0 \\
        0 & \id
    \end{pmatrix}\trans
    \begin{pmatrix}
        R & P \\
        P\trans & S
    \end{pmatrix}
    \begin{pmatrix}
        Q & 0 \\
        0 & \id
    \end{pmatrix}
    \in \mcl{S}_{\curlyeqsucc}^{(n + q) \times (n + q)},\quad -N = N\trans \in \bb{R}^{q \times q},\quad G \in \bb{R}^{n \times q}.
\end{equation}

We could now rephrase~\eqref{eq:lti} in terms of these matrices, however there is one more aspect to the port-Hamiltonian formulation we yet lack: the \emph{Hamiltonian}.
The Hamiltonian is motivated in Physics to model the total amount of energy within a system.
In this case we restrict ourselves to a quadratic Hamiltonian formulation
\begin{equation}\label{eq:quadratic-hamiltonian}
    \mcl{H} \colon I \times \bb{R}^{n} \to \bb{R},\quad \mcl{H}(t, x) \mapsto \frac{1}{2} x(t)\trans Q\trans E x(t).
\end{equation}

\begin{definition}\label{def:phlti}
    A \emph{port-Hamiltonian linear time-invariant (descriptor) system with quadratic Hamiltonian} is a system of equations
    \begin{equation}\label{eq:phlti}
        \Sigma_{\textsc{PHLTI}} \colon \left\lbrace
        \begin{aligned}
            E \dot{x}(t) &= (J - R) Q x(t) + (G - P) u(t), \\
            y(t) &= (G + P)\trans Q x(t) + (S - N) u(t)
        \end{aligned}
        \right.
    \end{equation}
    together with a Hamiltonian $\mcl{H}(t, x) = \frac{1}{2} x(t)\trans Q\trans E x(t)$ if the system matrices $E, Q \in \bb{R}^{m \times n}, J, R \in \bb{R}^{m \times m}$, $G, P \in \bb{R}^{m \times q}$, and $S, N \in \bb{R}^{q \times q}$ satify the properties
    \begin{equation*}
        -J = J\trans,\quad -N = N\trans,\quad
        \begin{pmatrix}
            Q & 0 \\
            0 & \id
        \end{pmatrix}\trans
        \begin{pmatrix}
            R & P \\
            P\trans & S
        \end{pmatrix}
        \begin{pmatrix}
            Q & 0 \\
            0 & \id
        \end{pmatrix}
        \in \mcl{S}_{\curlyeqsucc}^{(n + q) \times (n + q)}.
    \end{equation*}
\end{definition}

\itodo{mention representaiton as classical LTI system}

At this point it is useful to introduce the property of \emph{passivity}.
Passivity is related to the term stability introduced at the end of Section~\ref{sec:ltis}.
According to~\cite[Section~3.1.2]{Benner2017}, an LTI system $\Sigma_\textsc{LTI}$ is passive if it holds that
\begin{equation*}
    0 \leq \int\limits_0^t u(\tau)\trans y(\tau) \dif \tau,
\end{equation*}
though a more involved derivation in~\cite[Section~3.4]{Mehrmann2022} yields the definition by a \emph{dissipation inequality}
\begin{equation}\label{eq:dissipation-inequality}
    \mcl{H}(x(t_1)) - \mcl{H}(t_0) \leq \int\limits_{t_0}^{t_1} u(\tau)\trans y(\tau) \dif \tau,
\end{equation}
which we stick with for the remainder of this thesis.
Considering~\cite[Theorem~6.1]{Mehrmann2022}, the dissipation inequality~\eqref{eq:dissipation-inequality} is directly related to the positive semidefiniteness of $\begin{pmatrix}
    R & P \\
    P\trans & S
\end{pmatrix}$.
In a physical interpretation, the total systems energy is always related to the amount of energy inserted through the control and extracted through the output.
A very useful result of passivity is that any two passive systems can be coupled by a power-conserving or a dissipative interconnection and the combined system remains passive; cf.~\cite{Mehrmann2022, Morandin2022}.

\itodo{add removal of $Q$ and direct feedthrough terms}

Generally speaking, Definition~\ref{def:phlti} is too restrictive because the system~\eqref{eq:phlti} only allows for linear and constant coefficients.
In the final part of this chapter we highlight a three approaches to consider more general formulations.

The first of these constitutes itself in the application of linear time-varying operators.
This results in the following linear port-Hamiltonian system
\begin{equation}\label{eq:lph}
    \Sigma_{\textsc{LPH}} \colon \left\lbrace
    \begin{aligned}
        E(t) \dot{x}(t) + E(t) K(t) x(t) &= (J(t) - R(t)) Q(t) x(t) + (G(t) - P(t)) u(t), \\
        y(t) &= (G(t) + P(t))\trans Q(t) x(t) + (S(t) - N(t)) u(t)
    \end{aligned}
    \right.
\end{equation}
with the quadratic Hamiltonian $\mcl{H}(t, x) \coloneqq \frac{1}{2} x\trans Q(t)\trans E(t) x(t)$.
The port-Hamiltonian constraints on the dissipative elements $P, R$, and $S$ stays the same as for a time-invariant system $\Sigma_{\textsc{PHLTI}}$, however the condition for the previously simply skew-symmetric part shifts to the need for a skew-adjoint differential operator; cf.~\cite[Definition~4.8]{Mehrmann2022}.

The second approach is very similar to~\eqref{eq:lph}, however in addition to the time dependency we also allow for the operators to change based on the system state.
This then results in a nonlinear port-Hamiltonian system such as
\begin{equation}\label{eq:nlph}
    \Sigma_{\textsc{NLPH}} \colon \left\lbrace
    \begin{aligned}
        E(t, x) \dot{x}(t) + r(t, x) &= (J(t, x) - R(t, x)) \eta(t, x) + (G(t, x) - P(t, x)) u(t), \\
        y(t) &= (G(t, x) + P(t, x))\trans \eta(t, x) + (S(t, x) - N(t, x)) u(t)
    \end{aligned}
    \right.
\end{equation}
with the Hamiltonian implicitly defined through the differential equaionts
\begin{equation*}
    \pd{} \mcl{H}(t, x) = E(t, x)\trans \eta(t, x),\quad \text{ and }\quad \pd[t]{} \mcl{H}(t, x) = \eta(t, x)\trans r(t, x),
\end{equation*}
alongside similar constraints on the energy-conservative and dissipative parts of the port-Hamiltonian system as in the linear time-invariant case; cf.~\cite[Definition~4.1]{Mehrmann2022}.

As a final remark on the most general representation of port-Hamiltonian systems we have to take a massive conceptual leap, for a more detailed discussion we refer to~\cite[Section~2.2]{VanDerSchaft2014}.
Consider a finite-dimensional vector space $\mcl{F}$ and denote its dual space $\mcl{E} = \mcl{F}'$ endowed with the usual dual product ${\langle e, f \rangle}_{\mcl{E} \times \mcl{F}} = e(f)$ for some elements $f \in \mcl{F}, e \in \mcl{E}$.
We then call a subspace $\mcl{D} \sse \mcl{F} \times \mcl{E}$ a \emph{Dirac structure} if $\mcl{D}$ satisfies
\begin{enumerate}
    \item ${\langle e, f \rangle}_{\mcl{E} \times \mcl{F}}$ for all pairs $(f, e) \in \mcl{D}$, and
    \item $\dim{(\mcl{D})} = \dim{(\mcl{F})}$.
\end{enumerate}
In this setting we can thus define a port-Hamiltonian system to be the Dirac structure
\begin{equation*}
    \mcl{D} \sse \mcl{T}_x \mcl{X} \times \mcl{T}'_x \mcl{X} \times \mcl{F}_R \times \mcl{E}_R \times \mcl{F}_P \times \mcl{E}_P,
\end{equation*}
where $\mcl{X}$ denotes an abstract state space, $\mcl{T}_x \mcl{X}$ and $\mcl{T}'_x \mcl{X}$ the respective tangent and dual tangent spaces of $\mcl{X}$, a resistive Dirac structure $\mcl{F}_R \times \mcl{E}_R$, as well as an external port Dirac structure $\mcl{F}_P \times \mcl{E}_P$.
For a Hamiltonian function $\mcl{H} \colon \mcl{X} \to \bb{R}$, the first two factors can be characterized by the inner state energy preservig relation
\begin{equation*}
    \frac{\dif}{\dif t} H(x) = \pd{} \mcl{H}(x)\trans \dot{x} = 0.
\end{equation*}
Similarly, the resistive (dissipative) structure and the external port satisfy
\begin{equation*}
    \frac{\dif}{\dif t} \mcl{H}(x) \leq 0,\quad \frac{\dif}{\dif t} \mcl{H}(x) \leq e_p\trans f_p,\quad 0 = \pd{} \mcl{H}(x)\trans \dot{x} + e_R\trans f_R + e_P\trans f_P.
\end{equation*}

\itodo{take notion of port variable from van der Schaft, page 185 and add it to the beginning of port-based modelling approach description}
