\section{Systemtheoretic Linear MOR}\label{sec:system-mor}

System-theoretic model order reduction offers two main perspectives: reduction resulting from the notion of LTIs as in Definition~\ref{def:lti}, and reduction by means of the transfer function from Definition~\ref{def:transfer-function}.
We give an idea of matrix-based reduction in Subsection~\ref{subsec:balanced-truncation} explaining the method of Balanced Truncation.
Afterwards, we discuss reduction by means of interpolating the transfer function in Subsection~\ref{subsec:interpolation-reduction}.

\subsection{Balanced Truncation}\label{subsec:balanced-truncation}

\emph{Balanced Truncation} is a method consisting of a two step procedure: Firstly, balance the LTI system such that all states that one can only reach with difficulty are equally hard to observe, and secondly truncate the resulting system matrices such that only a reduced system of easily observable and reachable states remains.
The procedure was originally described in~\cite{Mullis1976} and later extended by~\cite{Moore1981}.
Within the balanced truncation framework, several options are available to cater to different systems.
Standard balancing (``Lyapunov'' balancing in~\cite{Gugercin2007}) often requires large computational time budgets because during this procedure one has to compute dense matrix factorizations.
To mitigate this drawback, approximate balanced truncation methods have been developped.
Other balancing methods exist such as stochastic balancing, frequency weighted balancing, or positive real balancing.
For our purposes, ordinary balanced truncation and positive real balances truncation suffice.
Positive real balanced truncation in particular is interesting in the context of this thesis because it produces passive systems which are closely linked to port-Hamiltonian systems.

\itodo{sources: approximate bt, stoch. bt, weighted bt, prbt}

We commence with standard balanced truncatoin; cf.~\cite{BB2017}.
Consider $(A, B, C, D, E)$ to be an LTI system as in Definition~\ref{def:lti}.
It is important to note that most balanced truncation literature employs LTIs that have $E = \id$.
If $E$ is regular any system $(A, B, C, D, E)$ can easily be transformed into a system of the form $(\tilde{A}, \tilde{B}, C, D, \id)$ by appliying $E\inv$ to the first equation in~\eqref{eq:lti}.
We make use of abuse of notation and henceforth simply refer to our system by the matrices $(A, B, C, D, \id)$.
To tell which system states are observable and which system states are controllable, we define the system's \emph{Gramians}
\begin{align}
    P &= \int\limits_0^\infty \exp{(A t)} B B\trans \exp{(A\trans t)} \dif t,\label{eq:controllability-gramian} \\
    Q &= \int\limits_0^\infty \exp{(A\trans t)} C\trans C \exp{(A t)} \dif t.\label{eq:observability-gramian}
\end{align}
In this context, $P$ reflects the controllability of the system, describing the ability to reach any state using the right control, and we call $Q$ the observability Gramian, signifying the set of points which cannot be differentiated from the initial point for some control producing a constantly zero ouput.

\itodo{make descriptions mre mathematical, cf.~\cite[Definitions~6.2 and 6.3]{BB2017}}

\itodo{add more sources about bt}

For asymptotically stable and minimal systems, these matrices satisfy the Lyapunov equalities
\begin{align*}
    A P + P A\trans + B B\trans &= 0, \\
    A\trans Q + Q A + C\trans C &= 0.
\end{align*}

\itodo{add source; see in Benner, Breiten}

Balancing now constitutes finding a transformation matrix $T$ such that the two matrices $T P T\trans, T\inv[T] Q T\inv$ are diagonal matrices.
Such a balancing transformation can be constructed by computing the decompositions
\begin{equation*}
    P = S\trans S,\quad Q = R\trans R,\quad S R\trans = U \Sigma V\trans
\end{equation*}
and then putting $T = \Sigma\inv[\frac{1}{2}] V\trans R$

\subsection{Interpolation-Based Reduction}\label{subsec:interpolation-reduction}

qwer
