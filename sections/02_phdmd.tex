\section{DMD for pH Systems}

\itodo{write on~\cite{Morandin2022}}

\subsection{pH Systems}

For the definition of pH Systems we take a cue from~\cite{Duindam2009, VanDerSchaft2014, Mehrmann2022}.
pH Systems can be stated in many different forms.
In particular, the Input-State-Output form in Equation~\eqref{eq:input-state-output}, cf.~\cite[Section~2.4]{VanDerSchaft2014} and~\cite[Section~2.2.3]{Duindam2009}, and the Input-State-Output form with direct feed-through in Equation~\eqref{eq:input-state-output-direct-feed}, cf.~\cite[Section~2.2.4]{Duindam2009} are of interest:
\begin{equation}\label{eq:input-state-output}
    \begin{aligned}
        \dot{x} &= (J(x) - R(x)) \partial_x H(x) + G(x) u, \\
        y &= G(x)\trans \partial_x H(x),
    \end{aligned}
\end{equation}
and
\begin{equation}\label{eq:input-state-output-direct-feed}
	\begin{aligned}
        \dot{x} &= (J(x) - R(x)) \partial_x H(x) + (G(x) - P(x)) u , \\
        y &= {(G(x) + P(x))}\trans \partial_x H(x) + (S(x) - N(x)) u.
    \end{aligned}
\end{equation}
In these equations, $x$ represents the internal state, $y$ is the output, $u$ stands in for the control, and $H$ is the Hamiltonian representing the total energy of the described system.

When dealing with passive models it is beneficial to consider the \emph{dissipation inequality}
\begin{equation}
    \partial_t H(x(t)) \leq y(t)\trans u(t).
\end{equation}
If a model satisfies this inequality, then the coupling of it to another passive model will also be a passive system.
Further, the stability of inferred systems can be guaranteed
\itodo{properties of matrices}
\itodo{find source/description for passive models}
\itodo{nonlinear dissipative pH descriptor system~\cite[Section~4]{Mehrmann2022}}

\subsection{phDMD for linear pH Systems}

For linear pH Systems with a Hamiltonian of the form $H(x) = \frac{1}{2} x\trans H x$ we can equivalently write Equation~\eqref{eq:input-state-output-direct-feed} as
\begin{equation}\label{eq:lin-ph-system}
    \begin{aligned}
        H \dot{x} &= (J - R) x + (G - P) u, \\
        y &= {(G + P)}\trans x + (S - N) u.
    \end{aligned}
\end{equation}
This system is then discretized using the implicit midpoint rule and $u((i + \frac{1}{2}) \Delta t) \approx \frac{u_{i + 1} + u_i}{2}$, cf.~\cite{Morandin2022}:
\begin{equation}\label{eq:discrete-lin-ph-system}
    \begin{aligned}
        H \frac{x_{i + 1} - x_i}{\Delta t} &= (J - R) \frac{x_{i + 1} + x_i}{2} + (G - P) \frac{u_{i + 1} + u_i}{2}, \\
        \frac{y_{i + 1} + y_i}{2} &= {(G + P)}\trans \frac{x_{i + 1} + x_i}{2} + (S - N) \frac{u_{i + 1} + u_i}{2}.
    \end{aligned}
\end{equation}
Here, $J = -J\trans, N = -N\trans, \mcl{R} = \begin{bmatrix}
    R & P \\
    P\trans & S
\end{bmatrix} \succcurlyeq 0$.
Equivalently we get
\begin{equation}\label{eq:disc-lin-ph-block-structure}
    \begin{bmatrix}
        H \frac{x_{i + 1} - x_i}{\Delta t} \\
        -\frac{y_{i + 1} + y_i}{2}
    \end{bmatrix}
    = (\mcl{J} - \mcl{R}) \begin{bmatrix}
        \frac{x_{i + 1} + x_i}{2} \\
        \frac{u_{i + 1} + u_i}{2}
    \end{bmatrix},
\end{equation}
with the discrete Hamiltonian $\Delta H_i = {\left( \frac{y_{i + 1} + y_i}{2} \right)}\trans H \left( \frac{x_{i + 1} - x_i}{\Delta t} \right)$.
Then, the \emph{discrete-time dissipation inequality} holds (cf.~\cite[Lemma~2.6]{Morandin2022}):
\begin{equation}
    \Delta H_i \leq {\left( \frac{y_{i + 1} + y_i}{2} \right)}\trans \left( \frac{u_{i + 1} + u_i}{2} \right).
\end{equation}

From here, we consider the data matrices
\begin{align*}
    \dot{X} &= \frac{1}{\Delta t} {\left( x_{i + 1} - x_i \right)}_{i = 0}^{k - 1} \in \bb{R}^{n \times k}, X = \frac{1}{2} {\left( x_{i + 1} + x_i \right)}_{i = 0}^{k - 1} \in \bb{R}^{n \times k}, \\
    U &= \frac{1}{2} {\left( u_{i + 1} + u_i \right)}_{i = 0}^{k - 1} \in \bb{R}^{m \times k}, Y = \frac{1}{2} {\left( y_{i + 1} + y_i \right)}_{i = 0}^{k - 1} \in \bb{R}^{m \times k}. \\
\end{align*}
Given any matrix $H \succ 0$, and a reduced basis $\varPhi \in \bb{R}^{n \times r}$, we want to find matrices $\mcl{J} = -\mcl{J}\trans, \mcl{R} \in \bb{R}^{k \times k}$ that minimize the norm
\begin{equation}\label{eq:phdmd-lst-problem}
    \norm[F]{\mcl{Z} - (\mcl{J} - \mcl{R})\mcl{T}},
    \mcl{Z} =
    \begin{bmatrix}
        \varPhi\trans H \varPhi \varPhi\trans \dot{X} \\
        - Y
    \end{bmatrix},
    \mcl{T} =
    \begin{bmatrix}
        \varPhi\trans X \\
        U
    \end{bmatrix}.
\end{equation}

If we know either $\mcl{J}$ or $\mcl{R}$, we can simply solve for the other matrix.
This fives rise to an iterative solution algorithm for this problem in~\cite[Section~3.2]{Morandin2022}.
The authors note that the initialization of both matrices drastically impacts performance and thus derive a similar but additionally weighted problem.
The solutions of this can then be used as either initialization for $\mcl{J}, \mcl{R}$ or by themselves, cf.~\cite[Section~3.4]{Morandin2022}.